\documentclass{acm_proc_article-sp}
%\documentclass{sig-alternate}
\usepackage{stmaryrd,amsmath,amssymb,newlfont,graphicx,verbatim}
\usepackage[ruled,lined,boxed,commentsnumbered,linesnumbered]{algorithm2e}
\usepackage{subfigure,url}
\usepackage{fancyvrb}


%\newtheorem{notation}[theorem]{Notation}

\newcounter{thedef} \setcounter{thedef}{1}
\newenvironment{definition}[1][]{\medskip\noindent {\textbf{Definition
  \arabic{thedef} #1} \stepcounter{thedef} \rm}}{\medskip}

\newcounter{theass} \setcounter{theass}{1}
\newenvironment{assumption}[1][]{\medskip\noindent {\textbf{Assumption
  \arabic{theass} #1} \stepcounter{theass} \rm}}{\medskip}

\newcounter{theeg} \setcounter{theeg}{1}
\newenvironment{example}[1][]{\medskip\noindent {\textbf{Example
  \arabic{theeg} #1} \stepcounter{theeg} \rm}{\medskip}}



\newcommand{\dom}{\mathrm{dom}}
\newcommand{\len}{\mathit{len}}
\newcommand{\poly}{\mathsf{poly}}
\newcommand{\flow}{\mathsf{flow}}
\newcommand{\jump}{\mathsf{jump}}
\newcommand{\inv}{\mathsf{inv}}
\newcommand{\init}{\mathsf{init}}
\newcommand{\guard}{\mathsf{guard}}
\newcommand{\reset}{\mathsf{reset}}
\newcommand{\reach}{\mathsf{Reach}}
\newcommand{\goal}{\mathsf{goal}}
\newcommand{\safe}{\mathsf{safe}}
\newcommand{\p}{\mathsf{P}}
\newcommand{\np}{\mathsf{NP}}
\newcommand{\R}{\mathbb{R}}
\newcommand{\lrf}{\mathcal{L}_{\mathbb{R}_{\mathcal{F}}}}
\newcommand{\citep}{\cite}

\newcommand{\hide}[1]{}
\newcommand{\eg}{{\em e.g.}}
\newcommand{\ie}{{\em i.e.}}
\newcommand{\enforce}{\mathsf{enforce}}




\begin{document}
\conferenceinfo{HSCC}{'14 Seattle, Washington, USA}


% a specific title
%\title{Identifying Personalized Hormone Therapy Schedules for Prostate Cancer Through Delta-Reachability Analysis
% a general title 
\title{Towards Personalized Cancer Therapy Using Delta-Reachability Analysis
\titlenote{This work has been partially supported by award N00014-13-1-0090 of the US Office of Naval Research and award CNS0926181 of the National Science foundation (NSF).}}

% alternative titles:
% Identifiying Optimized Intermittent Androgen Suppression Therapy Schedules for Prostate Cancer Using $\delta$-Decisions 
% Hybrid Modeling of Prostate Cancer Progression Reveals Optimized Angrogen Ablation Therapy Schedules
% Delta-Decision Based Analysis of a Hybrid Model for Prostate Cancer Progression

\numberofauthors{6} 
\author{
% 1st. author
\alignauthor
Bing Liu\\
       \affaddr{Computer Science Dept.}\\
       \affaddr{Carnegie Mellon University}\\
       \affaddr{Pittsburgh, PA, USA}\\
       \email{liubing@cs.cmu.edu}
% 2nd. author
\alignauthor
Soonho Kong\\
       \affaddr{Computer Science Dept.}\\
       \affaddr{Carnegie Mellon University}\\
       \affaddr{Pittsburgh, PA, USA}\\
       \email{soonhok@cs.cmu.edu}
% 3rd. author
\alignauthor 
Sicun Gao\\
       \affaddr{Computer Science Dept.}\\
       \affaddr{Carnegie Mellon University}\\
       \affaddr{Pittsburgh, PA, USA}\\
       \email{sicung@cs.cmu.edu}
\and  % use '\and' if you need 'another row' of author names
% 4th. author
\alignauthor 
Paolo Zuliani\\
       \affaddr{School of Computer Science}\\
       \affaddr{Newcastle University}\\
       \affaddr{Newcastle, UK}\\
       \email{paolo.zuliani@ncl.ac.uk}
% 5th. author
\alignauthor 
Edmund M. Clarke\\
       \affaddr{Computer Science Dept.}\\
       \affaddr{Carnegie Mellon University}\\
       \affaddr{Pittsburgh, PA, USA}\\
       \email{emc@cs.cmu.edu}
}

\date{20 October 2014}


\maketitle

\begin{abstract}
Recent clinical studies suggest that the efficacy of hormone therapy for prostate cancer depends on the characteristics of individual patients. In this paper, we develop a computational framework for identifying patient-specific androgen ablation therapy schedules for postponing the potential cancer relapse. We model the population dynamics of heterogeneous prostate cancer cells in response to androgen suppression as a nonlinear hybrid automaton. We estimate personalized kinetic parameters to characterize patients and employ a delta-reachability analysis based approach to predict patient-specific therapeutic strategies. The results show that our framework is promising and may lead to a prognostic tool for personalized cancer therapy.
\end{abstract}


% Meta info
% [todo]

\category{D.2.4}{Software Engineering}{Software/Program Verification}[Model checking]
\category{J.3}{Life and Medical Sciences}{Biology and genetics}

\terms{Theory\\ Verification}

\keywords{hybrid systems, delta-reachability, systems biology, prostate cancer, personalized therapy}


\section{Introduction}\label{sec:intro}

% Need a paragrapgh or two to explain why the tool is interesting and
% significant should be provided.

\dReach{} is a bounded model checker for hybrid systems. It encodes
bounded reachability problems of hybrid systems as first-order
formulas over the real numbers, and solves them using
$\delta$-decision procedures in the SMT solver \dReal{}. \dReach{} is
able to handle a wide range of highly nonlinear hybrid systems. It has
scaled well on various realistic nonlinear models from biomedical and
robotics applications~\cite{}.

It is well-known that the standard bounded reachability problems for
simple hybrid systems are already highly
undecidable~\cite{DBLP:conf/rex/AlurD91,DBLP:conf/hybrid/AlurCHH92}. In
previous work~\cite{}, we have defined the notion of
$\delta$-reachability problem of hybrid systems. In this new
framework, we have shown that bounded $\delta$-reachability is
decidable for a wide range of hybrid systems, with reasonable
complexity bounds~\cite{}. We give a brief review of the framework in
Section~\ref{sec:delta-reachability}.

Realistic hybrid systems involves nonlinear ODEs with transcendental
functions. \dReach{} allows users to specify a hybrid system in a
nonlinear signature as it is without linearizing or overapproximating
it. Users can provide the tool with a numerical error bound $\delta$,
a bounded time horizon $[0, T]$, and a maximum number of mode switches
$k$ for the analysis. As a result of analysis, \dReach{} will return
either \textbf{$\delta$-sat} with a concrete counterexample, or
\textbf{unsat} which does not involve numerical errors. We also
provide a visualization for the $\delta$-sat case to help
understand the analysis result.
\begin{figure}[!t]
  \subfloat[An example of nonlinear hybrid system model: off-treatment
  mode of the prostate cancer treatement model~\cite{}\label{subfig-1:prostate}]{
    \includegraphics[width=0.48\textwidth]{images/prostatebw-mode2.pdf}
  }
  \hfill
  \subfloat[Visualization of a concrete counterexample generated from
  dReach for the prostate cancer treatment model.]{%
    \includegraphics[width=0.48\textwidth]{images/prostate}
  }
  \caption{An example of nonlinear hybrid system model: Prostate
    cancer model.}
  \label{fig:prostate-example}
\end{figure}
For instance, figure~\ref{fig:prostate-example} shows a part of a
prostate cancer treatment model that contains nonlinear ODEs and a
visualization of a generated concrete counterexample.

\paragraph{Related Work}
%reachable set computation tools: flow star, SpaceX, Phaver,
%theorem provers:
%similar tools: iSAT, RSolver -- emphasize on the nonlinearity that we can handle.

The paper is structured as follows.


%%% Local Variables:
%%% mode: latex
%%% TeX-master: "main"
%%% End:

\section{Model description}\label{apndx:model}

\paragraph{Atrial Fibrillation.} The model has four discrete control locations, four state variables, and nonlinear ODEs. A typical set of ODEs in the model is:
\begin{eqnarray*}
\frac{du}{dt} &=& e + (u-\theta_v)(u_u-u ) v g_{fi} + wsg_{si}-g_{so}(u)\\
\frac{ds}{dt} &=& \displaystyle\frac{g_{s2}}{(1+\exp(-2k(u-us)))} -  g_{s2}s\\
\frac{dv}{dt} &=& -g_v^+\cdot v \hspace{1cm} \frac{dw}{dt} = -g_w^+\cdot w
\end{eqnarray*}
The exponential term on the right-hand side of the ODE is the sigmoid function, which often appears in modelling biological switches.
\paragraph{Prostate Cancer Treatment.} The Prostate Cancer Treatment model exhibits more nonlinear ODEs. The reachability questions are
\begin{eqnarray*}
\frac{dx}{dt} &=& (\alpha_x
(k_1+(1-k_1)\frac{z}{z+k_2}-\beta_x( (1-k_3)\frac{z}{z+k_4}+k_3)) - m_1(1-\frac{z}{z_0}))x + c_1 x\\
\frac{dy}{dt} &=& m_1(1-\frac{z}{z_0})x+(\alpha_y (1- d\frac{z}{z_0}) - \beta_y)y+c_2y\\
\frac{dz}{dt} &=& \frac{-z}{\tau} + c_3z\\
\frac{dv}{dt} &=& (\alpha_x
(k_1+(1-k_1)\frac{z}{z+k_2}-\beta_x(k_3+(1-k_3)\frac{z}{z+k_4}))\\
& &- m_1(1-\frac{z}{z_0}))x + c_1 x + m_1(1-\frac{z}{z_0})x+(\alpha_y (1- d\frac{z}{z_0}) - \beta_y)y+c_2y
\end{eqnarray*}
\paragraph{Electronic Oscillator.} The 3dOsc model represents an electronic oscillator model that contains nonlinear ODEs such as the following:
\begin{eqnarray*}
\frac{dx}{dt} &=& - ax \cdot sin(\omega_1 \cdot \tau)\\
\frac{dy}{dt} &=& - ay \cdot sin( (\omega_1 + c_1) \cdot \tau) \cdot sin(\omega_2)\cdot 2\\
\frac{dz}{dt} &=& - az \cdot sin( (\omega_2 + c_2) \cdot \tau) \cdot cos(\omega_1)\cdot 2\\
\frac{\omega_1}{dt} &=& - c_3\cdot \omega_1\ \ \ \frac{\omega_2}{dt} = -c_4\cdot\omega_2\ \ \ \frac{d\tau}{dt} = 1
\end{eqnarray*}
\paragraph{Quadcopter Control.} We developed a model that contains the full dynamics of a quadcopter. We use the model to solve control problems by answering reachability questions. A typical set of the differential equations are the following:
\begin{eqnarray*}
\frac{\mathrm{d}\omega_x}{\mathrm{d}t} &=& L\cdot k\cdot (\omega_1^2 - \omega_3^2)(1/I_{xx})-(I_{yy} - I_{zz})\omega_y\omega_z/I_{xx}\\
\frac{\mathrm{d}\omega_y}{\mathrm{d}t} &=& L\cdot k\cdot(\omega_2^2 - \omega_4^2)(1/I_{yy})-(I_{zz} - I_{xx})\omega_x\omega_z/I_{yy}\\
\frac{\mathrm{d}\omega_z}{\mathrm{d}t} &=& b\cdot(\omega_1^2 - \omega_2^2 + \omega_3^2 - \omega_4^2)(1/I_{zz})-(I_{xx} - I_{yy})\omega_x\omega_y/I_{zz}\\
\frac{\mathrm{d}\phi}{\mathrm{d}t} &=& \omega_x + \displaystyle{\frac{\sin\left(\phi\right) \sin\left(\theta\right)}{{\left(\frac{\sin\left(\phi\right)^{2} \cos\left(\theta\right)}{\cos\left(\phi\right)} + \cos\left(\phi\right) \cos\left(\theta\right)\right)} \cos\left(\phi\right)}}\omega_y + \displaystyle\frac{\sin\left(\theta\right)}{\frac{\sin\left(\phi\right)^{2} \cos\left(\theta\right)}{\cos\left(\phi\right)} + \cos\left(\phi\right) \cos\left(\theta\right)}\omega_z\\
\frac{\mathrm{d}\theta}{\mathrm{d}t} &=& -(\displaystyle\frac{\sin\left(\phi\right)^{2} \cos\left(\theta\right)}{{\left(\frac{\sin\left(\phi\right)^{2} \cos\left(\theta\right)}{\cos\left(\phi\right)}\omega_y + \cos\left(\phi\right) \cos\left(\theta\right)\right)} \cos\left(\phi\right)^{2}} + \frac{1}{\cos\left(\phi\right)})\omega_y\\
& &\hspace{5cm}-\displaystyle\frac{\sin\left(\phi\right) \cos\left(\theta\right)}{{\left(\frac{\sin\left(\phi\right)^{2} \cos\left(\theta\right)}{\cos\left(\phi\right)} + \cos\left(\phi\right) \cos\left(\theta\right)\right)} \cos\left(\phi\right)}\omega_z \\
\frac{\mathrm{d}\psi}{\mathrm{d}t} &=& \displaystyle\frac{\sin\left(\phi\right)}{{\left(\frac{\sin\left(\phi\right)^{2} \cos\left(\theta\right)}{\cos\left(\phi\right)} + \cos\left(\phi\right) \cos\left(\theta\right)\right)} \cos\left(\phi\right)}\omega_y + \displaystyle\frac{1}{\frac{\sin\left(\phi\right)^{2} \cos\left(\theta\right)}{\cos\left(\phi\right)} + \cos\left(\phi\right) \cos\left(\theta\right)}\omega_z\\
\frac{\mathrm{d}{xp}}{\mathrm{d}t} &=& (1/m)(\sin(\theta)\sin(\psi)k(\omega_1^2 + \omega_2^2 +\omega_3^2+\omega_4^2) - k\cdot d\cdot{xp})\\
\frac{\mathrm{d}{yp}}{\mathrm{d}t} &=& (1/m)(-\cos(\psi)\sin(\theta)k(\omega_1^2 + \omega_2^2 +\omega_3^2+\omega_4^2) - k\cdot d\cdot{yp})\\
\frac{\mathrm{d}{zp}}{\mathrm{d}t} &=& (1/m)(-g-\cos(\theta)k(\omega_1^2 + \omega_2^2 +\omega_3^2+\omega_4^2) - k\cdot d\cdot{zp}\\
\frac{\mathrm{d}x}{\mathrm{d}t} &=& {xp}, \frac{\mathrm{d}y}{\mathrm{d}t} = {yp}, \frac{\mathrm{d}z}{\mathrm{d}t} = {zp}
\end{eqnarray*}

\section{Methods}
\subsection{$\lrf$-formulas and $\delta$-decisions over the reals}
%\section{$\delta$-decisions}

We first briefly review our framework of $\delta$-decision problems for first-order sentences over the reals with computable real functions.
%
The notion of computable functions over the real numbers are developed in Computable Analysis~\citep{CAbook}. In our recent work~\citep{gao12a,gao12b}, we developed a theory of decision problems over the reals with computable functions. %We do not go into the details of the definitions of computable functions here, which is covered in the references.
It suffices to note that most common continuous real functions are computable, such as addition, multiplication, absolute value, $\min$, $\max$, $\exp$, $\sin$ and solutions of Lipschitz-continuous ordinary differential equations. Compositions of computable functions are computable. In fact, the notion of computability of real functions directly corresponds to whether they can be numerically simulated. We write $\mathcal{F}$ to denote an arbitrary collection of
symbols representing computable functions over $\mathbb{R}^n$ for various $n$. We consider the first-order formulas with a signature $\lrf = \langle 0,1,\mathcal{F},>\rangle$. Note that constants are seen as 0-ary functions in $\mathcal{F}$. $\lrf$-formulas are evaluated in the standard way over the corresponding structure $\mathbb{R}_{\mathcal{F}}= \langle \mathbb{R}, \mathcal{F}, >\rangle$.
%It is not hard to see that we only need to use atomic formulas of the form
We use atomic formulas of the form $t(x_1,...,x_n)>0$ or $t(x_1,...,x_n)\geq 0$, where $t(x_1,...,x_n)$ are built
up from functions in $\mathcal{F}$. To avoid extra preprocessing of formulas, we give an explicit definition of $\lrf$-formulas as follows.
%\begin{definition}[$\lrf$-Formulas]

\paragraph{$\lrf$-Formulas}
Let $\mathcal{F}$ be a collection of Type 2 functions, which contains at least
$0$, unary negation -, addition $+$, and absolute value $|\cdot|$. We define:
\begin{align*}
t& := x \; | \; f(t(\vec x)), \mbox{ where }f\in \mathcal{F}\mbox{, possibly
constant};\\
\varphi& := t(\vec x)> 0 \; | \; t(\vec x)\geq 0 \; | \; \varphi\wedge\varphi
\; | \; \varphi\vee\varphi \; | \; \exists x_i\varphi \; |\; \forall x_i\varphi.
\end{align*}
In this setting $\neg\varphi$ is regarded as an inductively defined operation
which replaces atomic formulas $t>0$ with $-t\geq 0$, atomic formulas $t\geq 0$
with $-t>0$, switches $\wedge$ and $\vee$, and switches $\forall$ and $\exists$.
Implication $\varphi_1\rightarrow\varphi_2$ is defined as
$\neg\varphi_1\vee\varphi_2$.
%\end{definition}

%\begin{definition}[Bounded Quantifiers]
We define
\begin{align*}
\exists^{[u,v]}x.\varphi &=_{df}\exists x. ( u \leq x \land x \leq v \wedge
\varphi),\\
\forall^{[u,v]}x.\varphi &=_{df} \forall x. ( (u \leq x \land x \leq v)
\rightarrow \varphi),
\end{align*}
where $u$ and $v$ denote $\lrf$ terms whose variables only
contain free variables in $\varphi$, excluding $x$. It is easy to check that
$\exists^{[u,v]}x. \varphi \leftrightarrow \neg \forall^{[u,v]}x. \neg\varphi$.
%\end{definition}
We say a sentence is bounded if it only involves bounded quantifiers.
%\begin{definition}[Bounded $\lrf$-Sentences]
A {\em bounded $\lrf$-sentence} is
$$Q_1^{[u_1,v_1]}x_1\cdots Q_n^{[u_n,v_n]}x_n\;\psi(x_1,...,x_n).$$
$Q_i^{[u_i,v_i]}$s are bounded quantifiers, and $\psi(x_1,...,x_n)$ is a
quantifier-free $\lrf$-formula.
%\end{definition}
We write $\psi(x_1,...,x_n)$ as $\psi[t_1(\vec
x)>0,...,t_k(\vec x)>0; t_{k+1}(\vec x)\geq 0,...,t_m(\vec
x)\geq 0]$ to emphasize that $\psi(\vec x)$ is a Boolean
combination of the atomic formulas shown.
%\begin{definition}[$\delta$-Variants]\label{variants}
\paragraph{$\delta$-Variants}
Let $\delta\in \mathbb{Q}^+\cup\{0\}$, and $\varphi$ an
$\lrf$-formula of the form
$$\varphi: \ Q_1^{I_1}x_1\cdots Q_n^{I_n}x_n\;\psi[t_i(\vec x, \vec y)>0;
t_j(\vec x, \vec
y)\geq 0],$$ where $i\in\{1,...k\}$ and $j\in\{k+1,...,m\}$. The {\em
$\delta$-weakening} $\varphi^{\delta}$ of $\varphi$ is
defined as the result of replacing each atom $t_i > 0$ by $t_i >
-\delta$ and $t_j \geq 0$ by $t_j \geq -\delta$. That is,
$$\varphi^{\delta}:\ Q_1^{I_1}x_1\cdots Q_n^{I_n}x_n\;\psi[t_i(\vec x, \vec
y)>-\delta; t_j(\vec x,
\vec y)\geq -\delta].$$
%\end{definition}

We then have the following main decidability result.
%\begin{theorem}[$\delta$-Decidability]
\paragraph{$\delta$-Decidability}
Let $\delta\in\mathbb{Q}^+$ be arbitrary. There is an algorithm which, given any bounded $\varphi$, correctly returns one of the following two answers:
\begin{itemize}
\item ``$\delta$-$\mathsf{True}$'': $\varphi^{\delta}$ is true.
\item ``$\mathsf{False}$'': $\varphi$ is false.
\end{itemize}
Note when the two cases overlap, either answer is correct.
%\end{theorem}

We call this new decision problem the $\delta$-decision problem for
$\lrf$-sentences.
%\begin{definition}[$\delta$-Complete Decision Procedures]
If an algorithm solves the $\delta$-decision problem correctly for a set $S$ of $\lrf$-sentences, we say it is $\delta$-complete for $S$.
%\end{definition}
From $\delta$-decidability, $\delta$-complete decision procedures always exists for bounded $\lrf$-formulas. In practice, we have shown that the combination of the DPLL(T) framework and Interval Constraint Propagation (ICP) indeed gives us a $\delta$-complete decision procedure. We implemented such procedures in our tool dReal~\citep{dreal}, which solves formulas containing transcendental functions and ordinary differential equations. In what follows we will see how $\delta$-complete decision procedures provide the engine for parameter synthesis of biological hybrid systems.





\subsection{Parameterized $\lrf$-representations of hybrid automata}

We now describe hybrid automata using $\lrf$-formulas, and define parameterization and perturbations on them.
%
A hybrid system~\citep{henzinger96} is a tuple $H = \langle X$, $Q$, $\flow$, $\guard$, $\reset$, $\inv$, $\init\rangle$
where $X\subseteq \mathbb{R}^n$ specifies the range of the {\em continuous variables}  $\vec x$ of the system. $Q=\{q_0,...,q_m\}$ is a finite set of discrete {\em control modes}. $\flow \subseteq Q\times X\times \R \times X$ specifies the {\em continuous dynamics} for each mode. The $\flow$ predicate is usually defined either as explicit mappings from $\vec a_0$ and $t$ to $\vec a_t$,  or as solutions of systems of differential equations/inclusions that specify the derivative of $\vec x$ over time. $\jump\subseteq Q\times X\times Q\times X$ specifies the {\em jump conditions} between modes. $\inv \subseteq Q\times X$ defines the {\em invariant conditions} for the system to stay in a control mode. $\init \subseteq Q\times X$ defines the set of {\em initial configurations} of the system. Without loss of generality we always assume that $q_0$ is the only intial mode, and $\init_{q_0}\subseteq X$ denotes the initial values for the continuous variables.
%We now $\lrf$-representations of hybrid automata.
%\begin{definition}[$\lrf$-Representations]\label{lrf-definition}
%\index{$\lrf$-Representation}
\paragraph{$\lrf$-representations of hybrid automata}
Let $H = \langle X$, $Q$, $\flow$, $\jump$, $\inv$, $\init\rangle$ be an $n$-dimensional hybrid automaton.  Let $\mathcal{F}$
be a set of real functions, and $\mathcal{L}_{\mathbb{R}_{\mathcal{F}}}$ the corresponding first-order language. We say that $H$ has an $\lrf$-representation, if for every $q,q'\in Q$, there exists  quantifier-free
$\lrf$-formulas $$\phi^q_{\flow}(\vec x, \vec x_0, t), \phi^{q\rightarrow
q'}_{\jump}(\vec x,
\vec x'), \phi^{q}_{\inv}(\vec x), \phi^q_{\init}(\vec x)$$
such that for all
$\vec a ,\vec a'\in \mathbb{R}^n$,
$t\in\mathbb{R}$:
\begin{itemize}
\item $\mathbb{R}\models \phi^q_{\flow}(\vec a, \vec a', t)$ iff $(q, \vec
a,
\vec a', t)\in \flow$.
\item $\mathbb{R}\models \phi^{q\rightarrow q'}_{\jump}(\vec a, \vec a')$ iff
$(q, q', \vec a, \vec a')\in \jump$.
\item $\mathbb{R}\models \phi^q_{\inv}(\vec a)$ iff $(q, \vec a)\in \inv.$
\item $\mathbb{R}\models \phi^q_{\init}(\vec a)$ iff $q = q_0$ and $\vec a\in
\init_{q_0}$.
\end{itemize}
%\end{definition}
We can write $H = \langle X, Q, \phi_{\flow}, \phi_{\jump}, \phi_{\inv},\phi_{\init}\rangle$ to emphasize that $H$ is $\lrf$-represented. But from now on we simply write $\flow, \jump, \inv, \init$ to denote these logic formulas, so that we can use $H = \langle X, Q, \flow, \jump, \inv, \init\rangle$ directly to denote the $\lrf$-representation of $H$.
%\begin{definition}[Invariant-Free Automata]\index{Invariant-free Hybrid Automata}
%A hybrid automaton is {\em invariant-free} its invariant formula $\inv$ is $\top$, and we say its invariant is {\em
%trivial} in this case.
%\end{definition}

%\begin{definition}[Computable Representation]
We say a hybrid automaton $H$ has a {\em computable representation}, if $H$ has
an $\lrf$-representation, where $\mathcal{F}$ is an arbitrary set of computable
functions.
%\end{definition}
Combining continuous and discrete behaviors, the trajectories of hybrid systems are {\em piecewise continuous}. This motivates a two-dimensional structure of time, with which we can keep track of both the discrete changes and the duration of each continuous flow.
%\begin{definition}[Hybrid time domain]
A {\em hybrid time domain} $T$ is a subset of $\mathbb{N}\times \mathbb{R}$ of the form
$T_m=\{(i, t): i<m \mbox{ and } t\in [t_i, t_i']\mbox{ or }[t_i, +\infty)\},$
where $m\in \mathbb{N}\cup\{+\infty\}$, $\{t_i\}_{i=0}^m$ is an increasing sequence in $\mathbb{R}^+$, $t_0= 0$, and $t_i'=t_{i+1}$.
%\end{definition}
We write the set of all hybrid time domains as $\mathbb{H}$.
%\begin{definition}[Hybrid Trajectories]
Suppose $X\subseteq\mathbb{R}^n$ and $T_m$ is a hybrid time domain. A {\em hybrid trajectory} is any continuous function $\xi: T_m\rightarrow X.$
%\end{definition}
We write $\Xi_X$ to denote the set of all possible hybrid trajectories from $\mathbb{H}$ to $X$.
We can now define trajectories of a given hybrid automaton. The intuition behind the following definition is straightforward. The labeling function $\sigma_{\xi}^H(i)$ is used to map a step $i$ to the corresponding discrete mode in $H$. In each mode, the system flows continuously following the dynamics defined by $\flow(q, \vec x_0, t)$. Note that $(t-t_k)$ is the actual duration in the $k$-th mode. When a switch between two modes is performed, it is required
that $\xi(k+1, t_{k+1})$ is updated from the exit value $\xi(k, t_k')$ in the previous mode, following the jump conditions.

%\begin{definition}[Trajectories of a Hybrid Automaton]\label{trajec}
%Let $H$ be a hybrid automaton, and $\xi: T_m\rightarrow X$ a hybrid trajectory.
We say that $\xi: T_m\rightarrow X$ is {\em a trajectory of $H$ of discrete depth $m$}, if there
exists a {\em labeling function} $\sigma^H_{\xi}: \mathbb{N}\rightarrow Q$ such
that:
\begin{itemize}
\item $\sigma^H_{\xi}(0) = q_0$ and
$\mathbb{R}_{\mathcal{F}}\models \init_{q_0}(\xi(0,0))$.
\item For any $(i, t)\in T_m$,
$\mathbb{R}_{\mathcal{F}}\models \inv_{\sigma^H_{\xi}(i)} (\xi(i,t))$.
\item When $i=0$, $\mathbb{R}_{\mathcal{F}}\models\flow_{q_0}(\xi(0,0), \xi(0,t), t)$.
\item When $i = k+1$, where $0< k+1 <m$,
\begin{eqnarray*}
\mathbb{R}_{\mathcal{F}}&\models&\flow_{\sigma^H_{\xi}(k+1)}(\xi(k+1, t_{k+1}),
\xi(k+1, t), (t - t_{k+1}))\mbox{ and }\\
\mathbb{R}_{\mathcal{F}}&\models& \jump_{(\sigma^H(k)\rightarrow
\sigma^H(k+1))}(\xi(k, t_k'), \xi(k+1,t_{k+1})).
\end{eqnarray*}
\end{itemize}
%\end{definition}
We write $\llbracket H\rrbracket$ to denote the set of all possible trajectories
of $H$.

%\begin{definition}[Reachability Properties]
\paragraph{Reachability Properties}
%Let $H$ be a hybrid automaton and $U\subseteq X\times Q$ be a subset of its state space.
Let $U\subseteq X\times Q$ be a subset of the state space of $H$. $H$ reaches $U$ if there exists $\xi\in \llbracket H\rrbracket$ such that there exists $t\in \mathbb{R}$ and $n\in\mathbb{N}$ satisfying
$$(\xi(t,n), \sigma_{\xi}^{H}(n))\in U.$$
%\end{definition}
%Let $H$ be a hybrid system.
%Parameter synthesis for reachability properties asks for a set of parameters such that some mode can be reached.
%
%\begin{definition}[Parameterized Hybrid Automaton]
%We say a hybrid automaton $H$ is parameterized by $\vec p$, if
We say $H$ is parameterized by $\vec p = (p_1,...,p_m)$, if
$$H(\vec p) = \langle X, Q, \flow(\vec p), \jump(\vec p), \inv(\vec p), \init(\vec p)\rangle,$$
where $\vec p$ are among the free variables in the $\lrf$-representation of $H$.
%\end{definition}
%\begin{definition}[Parameter Synthesis for Reachability Properties]
%Let $H(\vec p)$ be a hybrid automaton parameterized by variables $\vec p = (p_1,...,p_m)$, and $U\subseteq X\times Q$ a subset of its state space.
Thus, the parameter synthesis problem for reachability asks for an assignment for $\vec a\in \mathbb{R}^m$ such that $H(\vec a)$ reaches $U$.
%\end{definition}

%\begin{example}
%
%\end{example}




\subsection{Synthesizing parameters with $\delta$-decisions}

We now show how to encode parameter synthesis problems for $\lrf$-represented hybrid systems using $\lrf$-formulas. %The encoding is mostly standard bounded model checking, as pointed out in~\cite{}.
%However, when the invariant of a mode is nontrivial, we need to make sure that a trajectory satisfy the invariant throughout the continuous flow in the mode. This requires nested quantifiers, which do not occur in other verification domains, and has not been observed previously. We will consider systems with trivial invariant conditions first, and then handle the general case.
Throughout the following two definitions, let $H = \langle X$, $Q$, $\flow$, $\jump$, $\init\rangle$ be an $n$-dimensional $\lrf$-represented hybrid system with $|Q|=m$, and $\unsafe$ an $\lrf$-formula that encodes a subset $U\subseteq X\times Q$. Let $k\in \mathbb{N}$ and $M\in \mathbb{R}$ be the bounds on steps and time respectively. Recall that $q_0\in Q$ always denotes the starting mode.

%The encoding is straightforward.
$\reach_{H,q'}^k(\vec x_k^t)$ defines the states that $H$ can reach, if after $k$ steps of discrete changes it is in mode $q'$. From there, if $H$ makes a $\jump$ from mode $q'$ to $q$, then the states have
the make a discrete change following $\jump_{q'\rightarrow q}(\vec x_k^t, \vec
x_{k+1})$. As last, in mode $q'$, any state $\vec x_{k+1}^t$ that $H$ can reach
should satisfy the $\flow$ conditions $\flow_q(\vec x_{k+1}^t, \vec x_{k+1}, t)$
in mode $q$. Note that after each discrete jump, a new time variable $t_k$ is
introduced and independent from the previous ones.
%
%
%\begin{definition}[Invariant-Free Case]
%Let $H$ be invariant-free. Then $(k,M)$-reachability encoding of $H$ and $U$, written as
%$\reach^{k,M}(H,U)$, is defined as:
%\begin{eqnarray*}
%& & \exists \vec a \exists^X \vec x_0 \exists^X\vec x_0^t\cdots \exists^X\vec
%x_k\exists^X \vec x_k^t \exists^{[0,M]}t_0\cdots \exists^{[0,M]}t_k\\
%& &\Big(\ \init_{q_0}(\vec a, \vec x_0)\wedge \flow_{q_0}(\vec a, \vec x_0, \vec x_0^t, t_0)\\
%& &\wedge
%\bigvee_{i=0}^{k-1}\Big( \bigvee_{q, q'\in Q} \Big(\jump_{q\rightarrow q'}(\vec a, \vec x_i^t, \vec x_{i+1})\wedge \flow_{q'}(\vec a, \vec x_{i+1}, \vec
%x_{i+1}^t, t_{i+1})\Big)\Big)\\
%& &\wedge\ \goal(\vec x_{k}^t)\Big).
%\end{eqnarray*}
%\end{definition}
%
%Next, we define the encoding for general hybrid systems with nontrivial
%invariants.
%\begin{definition}[General case]
%Let $H$ be invariant-free.
The $(k,M)$-reachability encoding of $H$ and $U$, $\reach^{k,M}(H,U)$, is defined as:
\begin{flalign*}
&\exists \vec a \exists^X \vec x_0 \exists^X\vec x_0^t\cdots \exists^X\vec
x_k\exists^X \vec x_k^t \exists^{[0,M]}t_0\cdots \exists^{[0,M]}t_k&\\
&\Big(\ \init_{q_0}(\vec x_0)\wedge \flow_{q_0}(\vec a, \vec x_0, \vec x_0^t,
t_0)&\\
&\wedge \forall^{[0,t_0]}t\forall^X\vec x\;(\flow_{q_0}(\vec a, \vec x_0, \vec x,
t)\rightarrow \inv_{q_0}(\vec a, \vec x))&\\
&\wedge
\bigvee_{i=0}^{k-1}\Big( \bigvee_{q, q'\in Q} \Big(\jump_{q\rightarrow q'}(\vec a, \vec
x_i^t, \vec x_{i+1})\wedge \flow_{q'}(\vec a, \vec x_{i+1}, \vec
x_{i+1}^t, t_{i+1})&\\
&\wedge \forall^{[0,t_0]}t\forall^X\vec x\;(\flow_{q'}(\vec a, \vec x_{i+1}, \vec x,
t)\rightarrow \inv_{q_0}(\vec a, \vec x)) )\Big)\Big)&\\
&\wedge\ \unsafe(\vec a, \vec x_{k}^t)\Big).&
\end{flalign*}
%\end{definition}
%
%\begin{remark}[{\bf TODO}]
%The restriction on invariant can be relaxed. usually there is only a big
%invariant.
%\end{remark}
%We have explained the intuition behind the encoding of $\reach^{k,M}(H,U)$. A
%formal proof of the correctness, as stated in the following proposition, is
%contained in the Appendix.
%\begin{proposition}
$H$ reaches $U$ in $k$ steps of discrete jumps with time duration less than $M$ for each state, if and only if, $\reach^{k,M}(H,U)$ is true.
%\end{proposition}

%\begin{remark}
%Note that the solutions have errors.
%\end{remark}

%%% Local Variables:
%%% TeX-master: "main"
%%% End:

\section{Case Studies}

We have developed an open-source tool dReach using OCaml to perform $\delta$-complete reachability analysis for hybrid systems. dReach is built upon our SMT solver dReal \citep{dreal} that implements a $\delta$-complete decision procedure. All the experiments reported below were done using a machine with two Intel Xeon E5-2650 2.00GHz processors and 64GB RAM.

\subsection{Hormone therapy for prostate cancer}
Prostate cancer is the second leading cause of cancer-related deaths among men in United States \citep{cancerstat}. Hormone therapy in the form of androgen deprivation has been a cornerstone of the management of advanced prostate cancer for several decades. However, controversy remains regarding its optimum application \citep{nru}. Continuous androgen suppression (CAS) therapy has many side effects including anemia, osteoporosis, impotence, etc. Further, most patients experience a relapse after a median duration of 18-24 months of CAS treatment, due to the proliferation of androgen-independent (AI) cancer cells.

In order to reduce side effects of CAS and to delay the time to relapse, intermittent androgen suppression (IAS) was proposed aiming to limit the duration of androgen-poor conditions and avoid emergence of AI cells \citep{bruchovsky95}. In details, IAS therapy switches between on-treatment and off-treatment modes by monitoring the serum level of a tumor marker called prostate-specific antigen (PSA):  (i) when the PSA level decreases and reaches a lower threshold value $r_0$, androgen suppression is suspended; (ii) when the PSA level increases and reaches a upper threshold value $r_1$, androgen suppression is resumed by the administration of medical agents.

Recent clinical phase II and III trials confirm that IAS has significant advantages in terms of quality of life and cost. However, with respect to time to relapse and cancer-specific survival, the clinical trials suggest that to what extent IAS is superior to CAS depends on the individual patient and the on- and off-treatment scheme \citep{bruchovsky06,bruchovsky07,book13}. Thus, a crucial unsolved problem is how to design a personalized treatment scheme for each individual to achieve maximum therapeutic efficacy.

To answer this question, mathematical models have been developed to study the dynamics of prostate cancer under androgen suppression \citep{jackson04,ideta08, hirata10,pnas11}. Recently, attempts have been made to computationally classify patients and obtain the optimal treatment scheme \citep{chaos10,suzuki10}. However, these results relied on simplifying nonlinear hybrid dynamical systems to more manageable versions such as piecewise linear models \citep{chaos10} and piecewise affine systems \citep{suzuki10}. In this section, we show that our $\delta$-decision based parameter synthesis approach can help to design personalized treatment scheme based on nonlinear hybrid systems with arbitrary computable real functions. Here we focus on the hybrid model presented by \cite{ideta08}, which describes the growth of a prostate tumor as the dynamics of a mixed population of androgen-dependent (AD) and androgen-independent (AI) cells. 

The model has two modes which are shown in Figure \ref{pmodel}. $x(t)$, $y(t)$, and $z(t)$ represent the population of AD cells, the population of AI cells, and the serum androgen concentration, respectively. The growth dynamics of AD and AI cells are governed by their proliferation rate, apoptosis rate and mutation rate from AD to AI phenotype, depending on androgen concentration $z(t)$. The PSA level $v$ (ng ml$^{-1}$) is defined as $v(t)=x(t)+y(t)$. The treatment is suspended or restarted according to the value of $v$ and ${dv}/{dt}$. In mode $2$ (off-treatment), the androgen concentration is maintained at the normal level $z_0$ by homeostasis. In mode $1$ (on-treatment), the androgen is cleared at a rate $1/\tau$. Table \ref{prostate} lists the values of model parameters.

\begin{figure}[htb]
\centering
\includegraphics[scale=0.5]{fig-prostate}
\caption{The prostate cancer treatment model.}
\label{pmodel}
 \vspace{-0.7cm}
\end{figure}



\begin{table}[htb]
\processtable{Prostate cancer model parameter values\label{prostate}}
{\begin{tabular}{lll}\toprule
Parameter  & Bone metastasis & Lymph node metastasis  \\\midrule
$\alpha_x$ & 0.0204 d$^{-1}$ & 0.0168 d$^{-1}$  \\
$\alpha_y$ & 0.0242 d$^{-1}$ & 0.0277 d$^{-1}$  \\
$\beta_x$  & 0.0076 d$^{-1}$ & 0.0085 d$^{-1}$  \\
$\beta_y$  & 0.0168 d$^{-1}$ & 0.0222 d$^{-1}$  \\
$k_1$     & 0.0 nM & 0.0 nM \\
$k_2$     & 2.0 & 2.0  \\
$k_3$     & 8.0 nM & 8.0 nM \\
$k_4$     & 0.5 & 0.5  \\
$m_1$     & 0.00005 d$^{-1}$ & 0.00005 d$^{-1}$  \\
$z_0$     & 20.0 nM & 20.0 nM  \\
$\tau$     & 62.5 d & 62.5 d \\
\botrule
\end{tabular}}{}
 \vspace{-0.7cm}
\end{table}

\subsubsection{Model selection}
Based on different assumptions of the proliferation dynamics of AI cells, the above model has three variations, denoted as $H_1$, $H_2$, and $H_3$, which are discriminated by the value of $d$, i.e.:
\begin{itemize}
\item $H_1$: AI cells grow at the constant rate independent of the androgen level ($d=0$)\\
\item $H_2$: AI cells do not grow when the androgen level is normal ($d=1-\beta_2/\alpha_2$)\\
\item $H_3$: AI cells decrease when the androgen level is normal ($d=1)$
\end{itemize}

In order to perform model selection using $\delta$-decision procedures, we specified the cancer relapse as a state with ``$v>30$'', since the PSA level $v$ reflects the total number of tumor cells. We then checked whether each of the model candidates can reach a relapse state within a bounded time of $1000$ days. Here the treatment scheme threshold parameters were fixed as $r_0=4$ (ng ml$^{-1}$) and $r_1=10$ (ng ml$^{-1}$). The range of the initial concentration of androgen was given as $[10, 20]$ (nM).

Given the invariant $v \in [0,30]$, $H_1$ and $H_2$ are unable to reach a state with $t=1000$. In other words, $H_1$ and $H_2$ will always lead to cancer relapse state no matter which initial androgen concentration was chosen. This is conflict with the clinical observations by \cite{bruchovsky06,bruchovsky07}. In contrast, $H_3$ is able to avoid the relapse state and reproduce the experimental observation (see Figure \ref{prostate-fig1}). Thus, we completed the model selection process by ruling out $H_1$ and $H_2$ and choose $H_3$ for further analysis.


\begin{figure}[htb]
\centering
\includegraphics[scale=0.5]{fig-prostatetraj}
\caption{Simulated time profiles of $H_3$ model.}
\label{prostate-fig1}
 \vspace{-0.7cm}
\end{figure}

\subsubsection{Personalized therapy design}
We next apply our parameter synthesis method to selecting suitable therapy and designing personalized treatment scheme for individual patients. Note that the parameter values in Table \ref{prostate} were estimated from clinical data of hundreds of patients \citep{bruchovsky07}. Among patients, the values of some parameters vary, which causes the variability in responsiveness to hormone therapy. We select a set of ``personalized parameters'' including $\alpha_y$ (the proliferation rate of AI cells), $\beta_y$ (the apoptosis rate of AI cells), $m_1$ (the mutation rate from AD to AI cells), and $z(0)$ (the initial androgen level). The values of these parameters can be either experimentally measured \citep{berges95} or computationally determined from PSA time serials data \citep{hirata10}.

Figure \ref{patients}(a-c) illustrates the PSA dynamics of $3$ patients with different personalized parameters under the same IAS treatment scheme ($r_0=4$, $r_1=10$). IAS prevents the relapse for Patient\#1 and delays the relapse for Patient\#2, but does not help Patient\#3. Figure \ref{patients}(d) shows that, by modifying the IAS scheduling parameters $r_0$ and $r_1$, the relapse of Patient\#3 can be avoided or delayed. Thus, we can formulate the personalized therapy design problem as a parameter synthesis procedure: (i) fill in parameter values of a patient to $H_3$; (ii) set the ranges of scheduling parameters as $r_0 \in [0,8)$ (nM) and $r_1 \in [8,15]$; (iii) check if $H_3$ can reach a state with $t=1000$ given the invariant $v \in [0,30]$ (i.e. no relapse). If the $\delta$-decision procedure returns $\mathsf{False}$, it means that androgen suppression therapy is not suitable for the patient. Otherwise, a treatment scheme containing feasible values of $r_0$ and $r_1$ will be returned, which could prevent or delay the relapse of the patient. Note that if $r_0=0$ is returned, it refers to the CAS scheme.

\begin{figure}[htb]
\centering
\includegraphics[scale=0.55]{fig-prostatetraj2}
\caption{Simulated PSA profiles of patients with different parameters. (a) Patient\#1: $\alpha_y=0.0242$, $\beta_y=0.0168$, $m_1=0.00005$, $z(0)=12$, $r_0=4$, $r_1=10$ (b) Patient\#2: $\alpha_y=0.24$, $\beta_y=0.13$, $z(0)=13$, $m_1=0.0001$, $r_0=4$, $r_1=10$ (c) Patient\#3: $\alpha_y=0.35$, $\beta_y=0.187$, $m_1=0.00005$, $z(0)=10$, $r_0=4$, $r_1=10$ (d) Patient\#3: $\alpha_y=0.035$, $\beta_y=0.187$, $m_1=0.00005$, $z(0)=10$, $r_0=6$, $r_1=15$.}
\label{patients}
 \vspace{-0.7cm}
\end{figure}


We tested our method on real patients data collected by \cite{bruchovsky07}. The values of $\alpha_y$, $\beta_y$, $m_1$, and $z(0)$ for each selected patient were estimated by fitting the model to the PSA time serials data under the first $1.5$ cycles of IAS therapy (data available at http://www.nicholasbruchovsky.com/clinicalResearch.html). Table \ref{prostate2} summarized the suggested treatment scheme for selected patients. The in silico validation results are shown in Supplementary Materials.

\begin{table}[htb]
\processtable{Personalized hormone therapy scheme for selected patients\label{prostate2}}
{\begin{tabular}{llllll}\toprule
Patient ID  & $\alpha_y$  & $\beta_y$ & $m_1$ & $z(0)$ & Suggested scheme  \\\midrule
\#8 & 0.025 & 0.021  & 3.0E-5 & 8.23 & $r_0=5.0$, $r_1=11.2$ \\
\#10 & 0.019 & 0.009  & 5.9E-5 & 9.44 & $r_0=4.1$, $r_1=9.4$ \\
\#45 & 0.012  & 0.041  & 1.0E-5 & 12.61 & $r_0=3.8$, $r_1=12.2$ \\
\#97 & 0.031  & 0.015  & 2.3E-5 & 10.61 & $-$ \\
\botrule
\end{tabular}}{}
 \vspace{-0.7cm}
\end{table}

\begin{figure*}[t]
\centering
\includegraphics[scale=0.65]{fig-cardiac}
\caption{The minimal resistor model of cardiac cells.}
\label{mrm}
 \vspace{-0.7cm}
\end{figure*}


\subsection{Parameter identification for cardiac disorders}

Mathematical modeling the dynamics of cardiac cells is important in understanding the mechanisms of cardiac disorders. \cite{orovio08} has developed an extremely versatile electrical model for cardiac cells, referred as minimum resistor model (MRM), which reproduces experimentally measured characteristics of human ventricular cell dynamics. Identifying the parameter ranges for which the MRM accurately reproduces cardiac abnormalities will benefit the development of the treatment of cardiac disorders. For instance, improper functioning of the cardiac cell ionic channels may cause the cells to lose excitability. Unexcitable cells can induce ventricular tachycardia or fibrillation by blocking propagating electrical waves. In order to identify parameter ranges for which cardiac cells loss excitability, \cite{grosu11} linearized and transformed MRM into a piecewise-multiaffine system called MHA so that parameters synthesis process can be performed using the method proposed in \cite{rovergene}. However, due the the simplification, MRM and MHA have different sets of parameters. In this section, we show how we identify MRM parameter ranges for cardiac disorders without the help of linear approximation.



MRM contains $4$ state variables and $26$ parameters. An action potential (AP) is a change in the cell's transmembrane potential $u$, as a response to an external stimulus (current) $\epsilon$. The flow of total currents is controlled by a fast channel gate $v$ and two slow gates $w$ and $s$. %Figure \ref{mrm} shows the $4$ modes associated with MRM. The parameter values can be found in Supplementary Materials.
%
%
%\begin{figure}[h]
%\centering
%\includegraphics[scale=1]{fig-cardiactraj}
%\caption{The simulated time profile of MRM.}
%\label{ctrace}
% \vspace{-0.7cm}
%\end{figure}
%
In Mode $1$, gates $v$ and $w$ are open and gate $s$ is closed. The transmembrane potassium current causes the decay of $u$. The cell is resting and waiting for stimulation. We assume external stimulus $\epsilon$ equals to $1$ and lasts for $1$ millisecond. The stimulation causes $u$ increase which may trigger $\jump_{1 \rightarrow 2}: u \geq \theta_o$. In Mode 2, $v$ starts closing. The decay rate of $u$ changes. The systems will jump to Mode 3 if $u \geq \theta_w$. In Mode 3, $w$ is also closing. $u$ is governed by the potassium current and the calcium current. When $u \geq \theta_v$, Mode 4 can be reached which means a successful AP initiation. In Mode 4, $u$ reaches its peak due to the fast opening of sodium channel. The cardiac muscle contracts and $u$ starts decreasing. Figure \ref{ctrace} shows a witness trace computed by dReal when Mode 4 is reachable. The stimulus $\epsilon$ was reset every $500$ milliseconds.







When the system can not reach a state in Mode 4, the cardiac cell loses the excitability, which might lead to tachycardia and fibrillation. Starting with Mode 1, we then synthesized parameters using which the system will never go into Mode 4. We obtained the following results (see Supplementary Materials for more details):
$$\tau_{o1} \in (0,0.006)\vee \tau_{o2} \in (0,0.13)\vee 6.2 \cdot \tau_{so1} + \tau_{so2} \ge 9.9$$

The results suggest that when $\tau_{o1} \in (0, 0.006)$, the system will always stay in Mode 1. When $\tau_{o2} \in (0, 0.13)$, a state in Mode 3 can not be reached. Furthermore, whether the system can jump from Mode 3 to Mode 4 depends on the interplay between $\tau_{so1}$ and $\tau_{so2}$.  Figure \ref{cresults} visualizes these results by showing the simulated time profiles with different parameter values.



\begin{figure}[h]
\centering
\includegraphics[scale=0.6]{fig-cardiactraj2}
\caption{Simulation results with different parameter settings. (a) Original parameters (b) $\tau_{o1}=0.0055$ (c) $\tau_{o2} = 0.125$ (d) $\tau_{so1} =1.2$, $\tau_{so2} =1.0$ }
\label{cresults}
 \vspace{-0.7cm}
\end{figure}

%\paragraph{Mode 1 (recovering resting mode)}
%The $\flow_1$ condition:
%\begin{eqnarray*}
%\frac{du}{dt}&=& \epsilon - \frac{u}{\tau_{o1}}\\
%\frac{dv}{dt}&=& \frac{1 - v}{\tau_{v1}^-}\\
%\frac{dw}{dt}&=&  \displaystyle\frac{1 -  \displaystyle\frac{u}{\tau_w^{\infty}}-w}{\tau_{w1}^{-}+(\tau_{w2}^{-}-\tau_{w1}^{-}) \displaystyle\frac{1}{1+e^{-2k_{w}^{-}(u-u_{w}^{-})}}}\\
%\frac{ds}{dt}&=& \Big(\frac{1}{1+e^{-2k_{s}(u-u_{s})}}-s\Big)\frac{1}{\tau_{s1}}
%\end{eqnarray*}
%The jump condition $\jump_{1 \rightarrow 2}: u \geq \theta_o$
%
%\paragraph{Mode 2 (recovering resting mode)}
%The $\flow_2$ condition:
%\begin{eqnarray*}
%\frac{du}{dt}&=& \epsilon - \frac{u}{\tau_{o2}}\\
%\frac{dv}{dt}&=& \frac{- v}{\tau_{v2}^-}\\
%\frac{dw}{dt}&=&  \displaystyle\frac{w_{\infty}^{*}  -  w}{\tau_{w1}^{-}+(\tau_{w2}^{-}-\tau_{w1}^{-}) \displaystyle\frac{1}{1+e^{-2k_{w}^{-}(u-u_{w}^{-})}}}\\
%\frac{ds}{dt}&=& \Big(\frac{1}{1+e^{-2k_{s}(u-u_{s})}}-s\Big)\frac{1}{\tau_{s1}}
%\end{eqnarray*}
%The jump condition $\jump_{2 \rightarrow 1}: u < \theta_o$, $\jump_{2 \rightarrow 3}: u \geq \theta_w$
%
%\paragraph{Mode 3 (recovering resting mode)}
%The $\flow_3$ condition:
%\begin{eqnarray*}
%\frac{du}{dt}&=& \epsilon - \displaystyle\frac{1}{\tau_{so1}+(\tau_{so2}-\tau_{so1})\displaystyle\frac{1}{1+e^{-2k_{so}(u-u_{so})}}}+w\frac{s}{\tau_{si}}\\
%\frac{dv}{dt}&=& \frac{- v}{\tau_{v2}^-}\\
%\frac{dw}{dt}&=&  \frac{-w}{\tau_w^+}\\
%\frac{ds}{dt}&=& \Big(\frac{1}{1+e^{-2k_{s}(u-u_{s})}}-s\Big)\frac{1}{\tau_{s1}}
%\end{eqnarray*}
%The jump condition $\jump_{3 \rightarrow 2}: u < \theta_w$, $\jump_{3 \rightarrow 4}: u \geq \theta_v$
%
%
%\paragraph{Mode 4 (recovering resting mode)}
%The $\flow_4$ condition:
%\begin{eqnarray*}
%\frac{du}{dt}&=& \epsilon + v (u-\theta_v)(u_u-u) \frac{1}{\tau_{fi}}\\
%                    &  &  - \displaystyle\frac{1}{\tau_{so1}+(\tau_{so2}-\tau_{so1})\displaystyle\frac{1}{1+e^{-2k_{so}(u-u_{so})}}}+w\frac{s}{\tau_{si}}\\
%\frac{dv}{dt}&=& \frac{- v}{\tau_{v2}^+}\\
%\frac{dw}{dt}&=&  \frac{-w}{\tau_w^+}\\
%\frac{ds}{dt}&=& \Big(\frac{1}{1+e^{-2k_{s}(u-u_{s})}}-s\Big)\frac{1}{\tau_{s2}}
%\end{eqnarray*}
%The jump condition $\jump_{3 \rightarrow 2}: u < \theta_v$.






%\subsection{EGF-NGF signaling pathway model}

%PC12 cells are a valuable model system in neuroscience. They proliferate in response to EGF stimulation but differentiate into sympathetic neurons in response to NGF. This interesting phenomenon has been intensively studied \citep{brown}. It has been reported that the signal specificity is correlated with different Erk dynamics. Specifically, a transient activation of Erk1/2 has been associated with cell proliferation, while a sustained activity has been linked to differentiation. How EGF and NGF affect the dynamics of active Erk through a network of intermediate signaling proteins is shown schematically in Figure 2. This model not only includes a common pathway to Erk through Ras shared by both the EGFR and NGFR, but also includes two important side branches through PI3K and C3G, which introduce multiple feedback loops thus complicating the dynamics. The ODE model of this pathway is available in the BioModels database \citep{}. It consists of 32 differential equations and 48 associated rate parameters (estimated from multiple sets of experimental data).


%\begin{figure}
%\centering
%\includegraphics[scale=0.3]{EGFNGF2.pdf}
%\caption{EGF-NGF pathway}
%\label{egf-ngf}
%\end{figure}


%This case study is for parameter estimation. Our advantage is that we can estimate interval values for parameters using noisy experimental data (i.e. data points with error bars). it implies that our method takes account of the stochasticity of bio-systems and can identify a robust region of the parameter solution space.

%We translate an ODE model into a hybrid model using the experimental data and then check if it can reach the final mode.
%the expected result is:
%egfngf: sat, with parameter intervals
%In Figure ?, we show the fitting graphs, i.e. trajectories generated using the estimated parameters fit the data.

% END




%The model has two modes which are defined as follows:

%\paragraph{Mode 1 (on-treatment)}
%The $\flow_1$ condition:
%\begin{eqnarray*}
%\frac{dx}{dt}&=& \Big(\alpha_x \Big(k_1+(1-k_1)\frac{z}{z+k_2}\Big)-\beta_x\Big(k_3+(1-k_3)\frac{z}{z+k_4}\Big)\\
%                    &  & - m_1\Big(1-\frac{z}{z_0}\Big)\Big)x\\
%\frac{dy}{dt}&=& m_1\Big(1-\frac{z}{z_0}\Big)+\alpha_y \Big(1- d\frac{z}{z_0}\Big) - \beta_y\\
%\frac{dz}{dt}&=& \frac{-z}{\tau}
%\end{eqnarray*}
%The jump condition $\jump_{1 \rightarrow 2}:$
%$$x + y \leq r_0 \wedge \frac{dx}{dt} + \frac{dy}{dt} < 0$$
%
%\paragraph{Mode 2 (off-treatment)}
%The $\flow_2$ condition:
%\begin{eqnarray*}
%\frac{dx}{dt}&=& \Big(\alpha_x \Big(k_1+(1-k_1)\frac{z}{z+k_2}\Big)-\beta_x\Big(k_3+(1-k_3)\frac{z}{z+k_4}\Big)\\
%                    &  & - m_1\Big(1-\frac{z}{z_0}\Big)\Big)x\\
%\frac{dy}{dt}&=& m_1\Big(1-\frac{z}{z_0}\Big)+\alpha_y \Big(1- d\frac{z}{z_0}\Big) - \beta_y\\
%\frac{dz}{dt} &=& \frac{z_0-z}{\tau}
%\end{eqnarray*}
%The jump condition $\jump_{2 \rightarrow 1}:$
%$$x + y \geq r_1 \wedge \frac{dx}{dt} + \frac{dy}{dt} > 0$$
%where $x(t)$, $y(t)$, and $z(t)$ represent the population of AD cells, the population of AI cells, and the serum androgen concentration, respectively. The growth dynamics of AD and AI cells are governed by their proliferation rate, apoptosis rate and mutation rate from AD to AI phenotype, depending on androgen concentration $z(t)$. The PSA level $v$ (ng ml$^{-1}$) is defined as $v(t)=x(t)+y(t)$. The treatment is suspended or restarted according to the value of $v$ and ${dv}/{dt}$. In mode $2$ (off-treatment), the androgen concentration is maintained at the normal level $z_0$ by homeostasis. In mode $1$ (on-treatment), the androgen is cleared at a rate $1/\tau$. Table \ref{prostate} lists the values of model parameters.




%The model is defined as follows:
%\paragraph{Mode 1 (on-treatment)}
%The $\flow_1$ condition:
%\begin{eqnarray*}
%\frac{dx}{dt} &=& (G_x(z) - M_{xy}(z))x\\
%\frac{dy}{dt} &=& M_{xy}(z)x+G_y(z)y\\
%\frac{dz}{dt} &=& \frac{-z}{\tau}
%\end{eqnarray*}
%The jump condition $\jump_{1 \rightarrow 2}:$
%$$x + y \leq r_0 \wedge \frac{dx}{dt} + \frac{dy}{dt} < 0$$
%
%\paragraph{Mode 2 (off-treatment)}
%The $\flow_2$ condition:
%\begin{eqnarray*}
%dx/dt &=& (G_x(z) - M_{xy}(z))x\\
%dy/dt &=& M_{xy}(z)x+G_y(z)y\\
%dz/dt &=& \frac{z_0-z}{\tau}
%\end{eqnarray*}
%The jump condition $\jump_{2 \rightarrow 1}:$
%$$x + y \geq r_1 \wedge \frac{dx}{dt} + \frac{dy}{dt} > 0$$
%where $x$, $y$, and $z$ represent the population of androgen dependent (AD) cells, the population of androgen independent (AI) cells, and androgen concentration, respectively. The PSA level $v$ (ng ml$^{-1}$)is defined as $v=x+y$. The treatment is suspended or restarted according to the value of $v$ and $\fact{dv}{dt}$. The growth of AD cells and AI cells are governed by the respective net growth rates $G_x{z}$ and $G_y(z)$ and the mutation rate $M_{xy}{z}$ from the AD to AI phenotype. Theses are represented as follows:
%\begin{eqnarray*}
%G_x(z) &=& \alpha_x
%(k_1+(1-k_1)\frac{z}{z+k_2})-\beta_x(k_3+(1-k_3)\frac{z}{z+k_4})\\
%G_y(z) &=& \alpha_y (1- d\frac{z}{z_0}) - \beta_y\\
%M_{xy}(z) &=& m_1(1-\frac{z}{z_0})
%\end{eqnarray*}






\section{Conclusion}

We have presented a framework using $\delta$-complete decision procedures for the parameter identification 
of hybrid biological systems. We have used $\delta$-satisfiable formulas to describe parameterized hybrid automata 
and to encode parameter synthesis problems. We have employed $\delta$-decision procedures to perform bounded model 
checking, and developed an algorithm to obtain the resulting parameters. 
Our verified numerical integration and constraint programming algorithms effectively compute an over-approximation of the system dynamics. An  $\mathsf{unsat}$ answer can always be trusted, while a $\delta$-$\mathsf{sat}$ answer might be due to the over-approximation (see Section 2 for more details). We chose this behavior as it better fits with the safety requirements expected by formal verification.
We have demonstrated the applicability of our method on a highly nonlinear hybrid model of a cardiac cell that are
difficult to analyze with other verification tools. We have successfully ruled out a model candidate which did not fit experimental observations, and we have identified critical parameter ranges that can induce cardiac disorders.

It is worth noting that our method can be applied to ODE based models with discrete events, which are special forms of hybrid automata. Such models are often specified using the Systems Biology Markeup Language (SBML) %\cite{sbml} 
and archived in the BioModels database \cite{biomodels}. Currently, we are currently developing an SBML-to-dReal translator to facilitate the $\delta$-decision based analysis of SBML models. 
Further, our method also has the potential to be applied to other model formalisms such as hybrid functional Petri nets \cite{hfpn} and the formalisms realized in Ptolemy \cite{ptolemy}. We plan to explore this in future work.
% Bing: cited ptolemy 
% [Reviewer#1] "However, the approach also inherits the problems of hybrid automata for specifying large hybrid systems, therefore I assume you mention Hybrid Petri Nets in the conclusion, have you thought about other possibilities to specify hybrid systems, like e.g. multi-formalisms as realized in Ptolemy, maybe this point could be discussed in more detail"
Another interesting direction is applying our method for parameter estimation from experimental data. By properly encoding the noisy wet-lab experimental data using logic formulas, bounded model checking can be utilized to find the unknown parameter values.
In this respect, the specification logic used in \cite{liu13} promises to offer helpful pointers.




%%% Local Variables:
%%% TeX-master: "main"
%%% End:




%ACKNOWLEDGMENTS are optional
%\section{Acknowledgments}

\bibliographystyle{abbrv}
\bibliography{sigproc}

\end{document}
