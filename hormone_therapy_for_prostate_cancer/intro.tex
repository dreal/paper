\section{Introduction}
Prostate cancer is the second leading cause of cancer-related deaths among men in United States \citep{cancerstat}. Hormone therapy in the form of androgen deprivation has been a cornerstone of the management of advanced prostate cancer for several decades. However, controversy remains regarding its optimum application \citep{nru}. Continuous androgen suppression (CAS) therapy has many side effects including anemia, osteoporosis, impotence, etc. Further, most patients experience a relapse after a median duration of 18-24 months of CAS treatment, due to the proliferation of androgen-independent (AI) cancer cells.

In order to reduce side effects of CAS and to delay the time to relapse, intermittent androgen suppression (IAS) was proposed aiming to limit the duration of androgen-poor conditions and avoid emergence of AI cells \citep{bruchovsky95}. In details, IAS therapy switches between on-treatment and off-treatment modes by monitoring the serum level of a tumor marker called prostate-specific antigen (PSA):  (i) when the PSA level decreases and reaches a lower threshold value $r_0$, androgen suppression is suspended; (ii) when the PSA level increases and reaches a upper threshold value $r_1$, androgen suppression is resumed by the administration of medical agents.

Recent clinical phase II and III trials confirm that IAS has significant advantages in terms of quality of life and cost. However, with respect to time to relapse and cancer-specific survival, the clinical trials suggest that to what extent IAS is superior to CAS depends on the individual patient and the on- and off-treatment scheme \citep{bruchovsky06,bruchovsky07,book13}. Thus, a crucial unsolved problem is how to design a personalized treatment scheme for each individual to achieve maximum therapeutic efficacy.

To answer this question, mathematical models have been developed to study the dynamics of prostate cancer under androgen suppression \citep{jackson04,ideta08, hirata10,pnas11}. Recently, attempts have been made to computationally classify patients and obtain the optimal treatment scheme \citep{chaos10,suzuki10}. However, these results relied on simplifying nonlinear hybrid dynamical systems to more manageable versions such as piecewise linear models \citep{chaos10} and piecewise affine systems \citep{suzuki10}. In this section, we show that our $\delta$-decision based parameter synthesis approach can help to design personalized treatment scheme based on nonlinear hybrid systems with arbitrary computable real functions. Here we focus on the hybrid model presented by \cite{ideta08}, which describes the growth of a prostate tumor as the dynamics of a mixed population of androgen-dependent (AD) and androgen-independent (AI) cells.

\paragraph{Contribution.}

\paragraph{Releated work.}

\paragraph{Organization.}
