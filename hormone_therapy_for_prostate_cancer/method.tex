\section{Delta-Reachability Analysis}
%\section{Preliminaries}

%In this section, we present a computational framework for adaptively predicting the treatment schedules for prostate cancer patients. Figure \ref{} shows the flowchart of our approach. 

%In this section, we present the $\delta$-reachability analysis based techniques we have developed for hybrid

Hybrid automata are difficult to analyze. It has been shown that even simple reachability questions for hybrid 
systems with linear differential dynamics are undecidable \citep{henzinger96}. Therefore, in order to analyze our hybrid model of prostate cancer progression, we employed a $\delta$-reachability based framework \cite{liu14} which can sidesteps undecidability and allows the parameter synthesis problem to be relaxed in a sound manner and solved algorithmically. 

\subsection{Delta-Decisions}
The framework of $\delta$-complete decision procedures~\cite{gao12a} aims to solve first-order logic formula with arbitrary computable real functions, such as elementary functions and solutions of Lipschitz-continuous ODEs \citep{gao12b}. The answers returned by such procedures are either $\mathsf{unsat}$ or $\delta$-$\mathsf{sat}$. Here, $\mathsf{unsat}$ means the corresponding formula is verifiably false, while $\delta$-$\mathsf{sat}$ means a $\delta$-weakening version of the formula is true. In other words, $\delta$-decision procedures overcome undecidability issues by returning answers with one-sided $\delta$-bounded errors. Note that $\delta$ is an arbitrarily small positive rational chosen by the user. The algorithms for solving $\delta$-decision problems were described in our previous work \cite{gao12b,gao13} and were implemented in the dReal toolset \cite{dreal}. 

\subsection{Parameter identification}
Further, we have also proposed an encoding scheme which aimed to answer bounded reachability problems of hybrid automata with nontrivial invariants \cite{liu14}. This encoding enabled us to tackle the parameter identification problem by answering a $k$-step reachability question: ``Is there a parameter
combination for which the model reaches the goal region in $k$ steps?'' Essentially, we describe the set of states of interest (goal region) as a first-order logic formula and perform bounded model checking \cite{BMC} to determine reachability of these states. We then adapt an interval constraint propagation based algorithm to explore the parameter space and identify the sets of resulting parameters. If none exist, then the model is 
{\em unfeasible}. Otherwise, a witness (\ie, a value for each parameter) is returned. We have developed the dReach tool \cite{dreach} (\verb#http://dreal.cs.cmu.edu/dreach.html#) that automatically builds reachability formulas from a hybrid model and a goal description. Such formulas are then solved by the $\delta$-complete solver dReal \citep{dreal}.

For the interested readers, we refer to %Appendix (\url{http://www.cs.cmu.edu/~liubing/hscc15/}) and 
\cite{liu14} for more details on %$\delta$-decisions and 
$\delta$-reachability analysis based parameter identification.

