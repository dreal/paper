\section{Conclusion}
\label{sec:Conclusion}

We have proposed a hybrid model to study the prostate cancer cell dynamics in response to hormone therapy. Using $\delta$-reachability analysis, we obtained interesting biological insights into the prostate cancer heterogeneity. We also developed a $\delta$-decisions based computational framework for predicting patient-specific treatment schedules. We have demonstrated the applicability of our method with the help of real clinical datasets. Our study explored the possibilities of using formal methods to tackle quantitative systems pharmacology problems. Our results also highlighted $\delta$-reachability analysis as a potent technique in this line of research. 

Experimental validation of our method might require years of clinical studies, which is beyond the scope of this case study. It is worth noting that our therapy design framework is generic and can be applied to other settings, for example, predicting the radiation dosing schedules for brain cancer \cite{leder14}. Furthermore, another interesting direction is to extend our model and framework to take into account the stochasticity of a cellular environment. In this respect, the probabilistic modeling and statistical analysis techniques in \cite{liu11,liu13,liu12bioinfo} might offer helpful pointers.
