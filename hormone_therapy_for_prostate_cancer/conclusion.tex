\section{Conclusion}
\label{sec:Conclusion}

In this paper we have proposed a hybrid model for the prostate cancer cell dynamics in response to the hormone therapy. The model was able to reproduce previous published clinical data. It also allowed us to obtain interesting insights about the population dynamics of hormone sensitive cells and castration resistant cells under androgen suppression treatment. Using $\delta$-complete decision procedures as the core, we proposed a computational framework for identifying personalized treatment schedules. After estimating model parameters from the data of individual patients, we have predicted a treatment schedule for each patient in order to avoid or delay the cancer relapse.

Experimental verification of our method might require years of clinical studies, which is beyond the scope of this case study. It is worth noting that our therapy schedule design framework is generic and can be applied to other settings, for example, predicting the radiation dosing schedules for brain cancer \cite{leder14}. Furthermore, another interesting direction is to extend our model and framework to take into account the stochasticity in cellular environment. In this respect, the probabilistic modeling and statistical analysis techniques in \cite{liu11,liu13,liu12bioinfo} might offer helpful pointers.
