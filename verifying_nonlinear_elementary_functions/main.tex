\documentclass{acm_proc_article-sp}

\begin{document}
\title{Verifying Nonlinear Elementary Functions in the Embedded GNU C Library}

\maketitle
\begin{abstract}
A wide range of static analysis is done on continuous programs, i.e., programs that implement continuous functions, usually over the real numbers. The interesting cases involve highly nonlinear functions. The
most basic cases are the ones in the standard C library that
implements elementary functions using floating point arithmetic. We
propose a method for verifying these programs that combines testing,
numerical analysis, and SMT solving, which aim to provide full
coverage in verification. The method can be understood as a type of
concolic testing, but the emphasis is on continuous functions and
numerical properties make it worthwhile to study specific techniques.
We generate point inputs to obtain preliminary coverage, and then use
SMT formulas to verify the behavior of the implementations around the
neighborhood of the tested points. In this way the algorithm can
completely cover any bounded interval of interest. We present a
comprehensive study of the floating point implementation of nonlinear
elementary functions in the Embedded GNU C library, and report serious
bugs in the library through the method, and as well as verified
intervals that guarantee correct implementations.

%%% Local Variables:
%%% mode: latex
%%% TeX-master: "main"
%%% End:

\end{abstract}

%\category{H.4}{Information Systems Applications}{Miscellaneous}
%\category{D.2.8}{Software Engineering}{Metrics}[complexity measures, performance measures]
%\terms{Theory}
%\keywords{ACM proceedings, \LaTeX, text tagging}

\section{Introduction}



\section{Preliminaries}

\subsection{Structure of the implementation of the function}

\subsection{SMT over the reals}


\section{The Test-and-Infer Loop}

\subsection{The main loop}

Give the over loop here and explain the general ideas. 


\subsection{Verifying Good Intervals}

For each good input $a\in \mathbb{R}$ and error bound $\varepsilon$, we find a neighborhood $I$ around $a$ such that 
$$\forall x\in I.\; |F_{\sin}(x)-\sin(x)|\leq \varepsilon$$
where $F_{\sin}$ denotes the function encoding the program.  

Clearly, we have
\begin{eqnarray}
|F_{\sin}(x)-\sin(x)|\leq |F_{\sin}(x)-\sin(a)|+|\sin(x)-\sin(a)|,
\end{eqnarray}
and we only need to show
\begin{eqnarray}
\forall x\in I.\;|F_{\sin}(x)-\sin(a)|+|\sin(x)-\sin(a)|\leq \varepsilon.
\end{eqnarray}
For this, we can check satisfiability of the negation of the formula, i.e.:
\begin{eqnarray}
\exists x\in I.\;|F_{\sin}(x)-\sin(a)|+|\sin(x)-\sin(a)|> \varepsilon.
\end{eqnarray}
With Taylor expansion, we know
\begin{eqnarray}
|\sin(x)-\sin(a)|>\cos(a)|x-a|-\frac{\sin(a)}{2}(x-a)^2 + ... %check it
\end{eqnarray}
Thus, we only need to show
\begin{eqnarray}
\exists x\in I.\;|F_{\sin}(x)-\sin(a)|+\cos(a)|x-a|-\frac{\sin(a)}{2}(x-a)^2> \varepsilon.
\end{eqnarray}

The program encoding can be sliced based on monitoring on the input. Basically we only need the slice that's used in producing the output. 

\subsection{Debugging Bad Intervals}

For a bad input $b$ and error bound $\varepsilon$, we aim to find a neighborhood $I$ around $b$ such that the error always exhibits:
$$\forall x\in I.\; |F_{\sin}(x)-\sin(x)|>\varepsilon$$
The bad cases usually come from some bugs in the program. For that we can have a formula encoding conditions on intermediate variables. Formalize this ...


\section{Results}

Generate a picture of the segmentation of the real line. 

A table for running time. Focus on coverage rate of each iteration of the loop. Compare with pure testing. 


\section{Conclusion}

The point is that the method is not limited to the functions, but for any implementation of a mathematically defined continuous process. The same methods can be used to test controllers for instance. 



\end{document}

%%% Local Variables:
%%% mode: latex
%%% TeX-master: t
%%% End:
