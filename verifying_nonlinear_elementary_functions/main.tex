\documentclass{acm_proc_article-sp}

\begin{document}
\title{Verifying Nonlinear Elementary Functions in the Embedded GNU C Library}

\maketitle
\begin{abstract}
We present a novel approach solve the probabilistic bounded reachability problem of 
hybrid systems with parameter uncertainty. Standard approaches to this problem require 
numerical solutions for large optimization problems, and become unfeasible for systems 
involving nonlinear dynamics over the reals. Our approach combines randomized 
sampling of probabilistic system parameters, SMT-based bounded reachability analysis, 
and statistical tests. We utilize $\delta-$complete decision procedures 
to solve reachability analysis in a sound way, i.e., we always decide correctly if, for a given
combination of parameters, the system actually reaches the unsafe region.
Compared to standard simulation-based analysis methods, our approach supports 
non-deterministic branching, increases the coverage of simulation, and avoids the
zero-crossing problem. We demonstrate that our method is feasible for general
hybrid systems with parametric uncertainty by applying the implemented tool {\bf SReach} to
a range of nonlinear hybrid systems with parametric uncertainty.

\hide{
We present a novel approach that combines Satisfiability Modulo Theories (SMT) and 
statistical testing to solve the probabilistic bounded reachability problem of 
hybrid systems with parameter uncertainty. That is, we want to find out whether 
a hybrid system with probabilistic system parameters reaches an unsafe region of the
state space within a finite number of steps with a probability greater (or less) than a 
fixed threshold. Standard approaches to this problem require numerical solutions for 
large optimization problems, and become unfeasible for systems involving nonlinear dynamics
over the reals. Our approach solves the reachability problem by combining randomized 
sampling of probabilistic system parameters, SMT-based bounded reachability analysis, 
and statistical tests. In particular, we utilize $\delta-$complete decision procedures 
to solve reachability analysis in a sound way, i.e., we always decide correctly if, for a given
combination of parameters, the system actually reaches the unsafe region (in the opposite case 
we may generate false positives, but this can be controlled by a precision parameter $\delta>0$).
Compared to other simulation-based analysis methods, our approach supports 
non-deterministic branching, increases the coverage of simulation, and avoids the
zero-crossing problem. We demonstrate that our method is feasible for general
hybrid systems with parametric uncertainty by applying the implemented tool {\bf SReach} to
a wide range of nonlinear hybrid systems.}
%to two representative examples - the prostate cancer treatment control and the cardiac system, 
%and further through applications to additional benchmarks.
\vspace{-.7cm}
\end{abstract}

%\category{H.4}{Information Systems Applications}{Miscellaneous}
%\category{D.2.8}{Software Engineering}{Metrics}[complexity measures, performance measures]
%\terms{Theory}
%\keywords{ACM proceedings, \LaTeX, text tagging}

\section{Introduction}



\section{Preliminaries}

\subsection{Structure of the implementation of the function}

\subsection{SMT over the reals}


\section{The Test-and-Infer Loop}

\subsection{The main loop}

Give the over loop here and explain the general ideas. 


\subsection{Verifying Good Intervals}

For each good input $a\in \mathbb{R}$ and error bound $\varepsilon$, we find a neighborhood $I$ around $a$ such that 
$$\forall x\in I.\; |F_{\sin}(x)-\sin(x)|\leq \varepsilon$$
where $F_{\sin}$ denotes the function encoding the program.  

Clearly, we have
\begin{eqnarray}
|F_{\sin}(x)-\sin(x)|\leq |F_{\sin}(x)-\sin(a)|+|\sin(x)-\sin(a)|,
\end{eqnarray}
and we only need to show
\begin{eqnarray}
\forall x\in I.\;|F_{\sin}(x)-\sin(a)|+|\sin(x)-\sin(a)|\leq \varepsilon.
\end{eqnarray}
For this, we can check satisfiability of the negation of the formula, i.e.:
\begin{eqnarray}
\exists x\in I.\;|F_{\sin}(x)-\sin(a)|+|\sin(x)-\sin(a)|> \varepsilon.
\end{eqnarray}
With Taylor expansion, we know
\begin{eqnarray}
|\sin(x)-\sin(a)|>\cos(a)|x-a|-\frac{\sin(a)}{2}(x-a)^2 + ... %check it
\end{eqnarray}
Thus, we only need to show
\begin{eqnarray}
\exists x\in I.\;|F_{\sin}(x)-\sin(a)|+\cos(a)|x-a|-\frac{\sin(a)}{2}(x-a)^2> \varepsilon.
\end{eqnarray}

The program encoding can be sliced based on monitoring on the input. Basically we only need the slice that's used in producing the output. 

\subsection{Debugging Bad Intervals}

For a bad input $b$ and error bound $\varepsilon$, we aim to find a neighborhood $I$ around $b$ such that the error always exhibits:
$$\forall x\in I.\; |F_{\sin}(x)-\sin(x)|>\varepsilon$$
The bad cases usually come from some bugs in the program. For that we can have a formula encoding conditions on intermediate variables. Formalize this ...


\section{Results}

Generate a picture of the segmentation of the real line. 

A table for running time. Focus on coverage rate of each iteration of the loop. Compare with pure testing. 


\section{Conclusion}

The point is that the method is not limited to the functions, but for any implementation of a mathematically defined continuous process. The same methods can be used to test controllers for instance. 



\end{document}

%%% Local Variables:
%%% mode: latex
%%% TeX-master: t
%%% End:
