\section{Experiments}\label{kepler}

We build the proof generation capacity into our open-source solver {\sf
  dReal}\footnote{\url{http://dreal.cs.cmu.edu} }. All the
experiments below are performed on a machine of with a 32-core 2.0GHz Intel
Xeon E5-2600 Processor and 64GB of RAM.
\paragraph{Nonlinear Lemmas in the Kepler Conjecture.} A main motivation for us to build the proof checker is to contribute
to the Flyspeck project~\cite{DBLP:conf/dagstuhl/Hales05}, for the
fully formalized proof of the Kepler conjecture. As lemmas for the
proof, hundreds of nonlinear real inequalities need to be verified.
Although the formulas usually contain only around ten variables, they
contain a huge number of nonlinear arithmetic operations and
trigonometric functions, and are mathematically challenging. In the original proof, Hales implemented procedures that combine
linear programming and interval arithmetic to establish all these
formulas, but the algorithms are formally verify. In
fact, the formal verification of these nonlinear inequalities is the last main piece
of work needed to complete the full project. The state of the art is
explained in a very recent thesis~\cite{keplerthesis}, reporting the
proofs of about 10 inequalities so far, using formal Taylor series and
interval arithmetic.

%update this section
Without any particular optimization on ICP, we have already observed
promising results. Out of 916 nonlinear formulas in the Flyspeck
project repository, we solved 111 of them returning {\sf unsat} with a
timeout of 180 seconds and precision $\delta=10^{-3}$. Out of these
formulas, we applied the proof checking algorithm, and formally proved
72 of them directly.The proof traces of these formulas can be very
large; for instance, we proved one with more than 2M lines in the
proof (54MB file). In Table~\ref{tbl:exp}, we list some of the
representative benchmarks to show scalability. A full table for all
the results is on the tool page.
\begin{multline*}
\forall\vec{x} \in [4.0, 6.3504]^5\; \Big(2\mathrm{arctan} (\frac{\Delta_2(\vec x)}{\sqrt{\Delta_1(\vec x) + \Delta_2(\vec x)^2} + \sqrt{\Delta_1(\vec x)}})\\
- 0.458(\sqrt{x_2} + \sqrt{x_3} +\sqrt{x_4} + \sqrt{x_5}) + 0.342\sqrt{x_1} + 3.319204\Big) < 0.0
\end{multline*}
where
\begin{eqnarray*}
%give this definition
  \Delta_1(\vec{x}) &=& 4x_1 (8x_1 (-x_1 + x_2 + x_3 + x_4 + x_5 - 8) + x_2 x_5 (x_1 - x_2 + x_3 + x_4 - x_5+8\\%sure there's a plus 8 here?
& &+ x_3x_4(x_1 + x_2 - x_3 - x_4 + x_5 + 8)+ 8 x_2 x_3 - x_1 x_3 x_5 - x_1  x_2  x_4 - 8 x_4 x_5))\\
\Delta_2(\vec{x}) &=& x_2 x_5 -x_2 x_3 + x_3x_4 - x_4 x_5 +x_1^2 -x_1x_2 - x_1x_3 - x_1x_4 -x_1 x_5
\end{eqnarray*}

\begin{table}[th!]
  \begin{center}
\begin{tabular}{|l||r|r|r||r|r|r|r|r|}
\hline
ID & \#Var & \#Arith & Nonlinear & Time$_{\text{S}}$ & Proof Size & \#Sub & \#Axiom & Time$_{\text{PC}}$ \\
\hline
\hline
461 & 6 & 36 & poly & 1.740 & 2145155 & 2 & 17442 & 203.886 \\
\hline
789 & 6 & 86 & atan2,sqrt & 1.640 & 350329 & 2 & 2464 & 128.077 \\
\hline
792 & 6 & 828 & atan2,sqrt & 0.400 & 19837 & 2 & 118 & 113.004 \\
\hline
745 & 6 & 36 & poly & 0.750 & 677580 & 2 & 5222 & 59.865 \\
\hline
785 & 6 & 80 & atan2,sqrt & 0.470 & 63388 & 2 & 526 & 26.450 \\
\hline
760 & 6 & 2767 & atan2,sqrt & 0.140 & 711 & 2 & 5 & 21.089 \\
\hline
820 & 6 & 95 & atan2,sqrt & 0.080 & 9134 & 2 & 54 & 14.703 \\
\hline
815 & 6 & 95 & atan2,sqrt & 0.330 & 41954 & 2 & 279 & 14.703 \\
\hline
814 & 6 & 95 & atan2,sqrt & 0.350 & 42102 & 2 & 278 & 14.703 \\
\hline
816 & 6 & 96 & atan2,sqrt & 0.110 & 12195 & 2 & 92 & 4.994 \\
\hline
817 & 6 & 96 & atan2,sqrt & 0.090 & 11792 & 2 & 93 & 4.993 \\
\hline
784 & 6 & 80 & atan2,sqrt & 0.060 & 7203 & 2 & 56 & 3.595 \\
\hline
781 & 6 & 86 & atan2,sqrt & 0.060 & 7481 & 2 & 45 & 2.657 \\
\hline
793 & 6 & 834 & atan2,sqrt & 0.020 & 18 & 1 & 1 & 1.855 \\
\hline
796 & 6 & 834 & atan2,sqrt & 0.010 & 18 & 1 & 1 & 1.710 \\
\hline
752 & 6 & 17 &  poly & 0.080 & 46360 & 2 & 277 & 1.709 \\
\hline
783 & 6 & 825 &atan2,sqrt & 0.020 & 93 & 1 & 1 & 1.549 \\
\hline
779 & 6 & 201 & atan2,sqrt & 0.010 & 10 & 1 & 1 & 0.705 \\
\hline
867 & 6 & 17 &  poly & 0.040 & 25820 & 2 & 147 & 0.683 \\
% 493 & 6 & 155 & atan2 & 0.001 & 12 & 1 & 1 & 0.526 \\
% 492 & 6 & 155 & atan2 & 0.001 & 12 & 1 & 1 & 0.512 \\
% 806 & 6 & 96 & atan2 & 0.001 & 10 & 1 & 1 & 0.484 \\
% 494 & 6 & 59 & atan2 & 0.001 & 10 & 1 & 1 & 0.475 \\
% 843 & 6 & 97 & atan2 & 0.001 & 10 & 1 & 1 & 0.471 \\
% 852 & 6 & 97 & atan2 & 0.001 & 10 & 1 & 1 & 0.412 \\
% 842 & 6 & 97 & atan2 & 0.001 & 10 & 1 & 1 & 0.412 \\
% 858 & 6 & 97 & atan2 & 0.001 & 10 & 1 & 1 & 0.411 \\
% 855 & 6 & 97 & atan2 & 0.001 & 10 & 1 & 1 & 0.411 \\
% 853 & 6 & 97 & atan2 & 0.001 & 10 & 1 & 1 & 0.411 \\
% 850 & 6 & 97 & atan2 & 0.001 & 10 & 1 & 1 & 0.411 \\
% 845 & 6 & 97 & atan2 & 0.001 & 10 & 1 & 1 & 0.411 \\
% 844 & 6 & 97 & atan2 & 0.001 & 10 & 1 & 1 & 0.411 \\
% 840 & 6 & 97 & atan2 & 0.001 & 10 & 1 & 1 & 0.411 \\
% 839 & 6 & 97 & atan2 & 0.001 & 108 & 1 & 1 & 0.411 \\
% 857 & 6 & 97 & atan2 & 0.001 & 10 & 1 & 1 & 0.410 \\
% 847 & 6 & 97 & atan2 & 0.001 & 10 & 1 & 1 & 0.410 \\
% 841 & 6 & 97 & atan2 & 0.001 & 10 & 1 & 1 & 0.410 \\
% 838 & 6 & 97 & atan2 & 0.001 & 10 & 1 & 1 & 0.410 \\
% 791 & 6 & 95 & atan2 & 0.001 & 10 & 1 & 1 & 0.406 \\
% 755 & 6 & 59 & atan2 & 0.001 & 10 & 1 & 1 & 0.392 \\
% 833 & 6 & 95 & atan2 & 0.001 & 10 & 1 & 1 & 0.385 \\
% 827 & 6 & 95 & atan2 & 0.001 & 10 & 1 & 1 & 0.385 \\
% 823 & 6 & 95 & atan2 & 0.010 & 78 & 1 & 1 & 0.385 \\
% 836 & 6 & 95 & atan2 & 0.001 & 10 & 1 & 1 & 0.384 \\
% 835 & 6 & 95 & atan2 & 0.001 & 10 & 1 & 1 & 0.384 \\
% 834 & 6 & 95 & atan2 & 0.001 & 10 & 1 & 1 & 0.384 \\
% 825 & 6 & 95 & atan2 & 0.001 & 10 & 1 & 1 & 0.384 \\
% 751 & 6 & 171 & atan2 & 0.001 & 5 & 1 & 1 & 0.302 \\
% 809 & 6 & 89 & atan2 & 0.001 & 10 & 1 & 1 & 0.299 \\
% 750 & 6 & 170 & atan2 & 0.001 & 5 & 1 & 1 & 0.299 \\
\hline
742 & 6 & 55 & acos,atan2,sqrt & 0.001 & 7 & 1 & 1 & 0.299 \\
% 506 & 6 & 49 &  & 0.001 & 7 & 1 & 1 & 0.293 \\
% 504 & 6 & 48 &  & 0.001 & 7 & 1 & 1 & 0.292 \\
% 505 & 6 & 48 &  & 0.001 & 7 & 1 & 1 & 0.291 \\
\hline
508 & 6 & 53 & acos,sqrt & 0.001 & 8 & 1 & 1 & 0.286 \\
\hline
507 & 6 & 29 & acos,sqrt & 0.001 & 8 & 1 & 1 & 0.278 \\
\hline
744 & 6 & 24 & asin,cos,sin & 0.001 & 8 & 1 & 1 & 0.275\\
% 740 & 6 & 9 &  & 0.001 & 5 & 1 & 1 & 0.272 \\
% 862 & 6 & 18 &  & 0.010 & 3752 & 2 & 18 & 0.233 \\
% 799 & 6 & 89 & atan2 & 0.001 & 10 & 1 & 1 & 0.216 \\
% 950 & 6 & 16 &  & 0.010 & 1798 & 2 & 16 & 0.206 \\
% 773 & 6 & 59 & atan2 & 0.001 & 10 & 1 & 1 & 0.206 \\
% 860 & 6 & 18 &  & 0.010 & 2197 & 2 & 13 & 0.203 \\
% 795 & 6 & 95 & atan2 & 0.001 & 10 & 1 & 1 & 0.202 \\
% 818 & 6 & 95 & atan2 & 0.001 & 63 & 1 & 1 & 0.182 \\
% 819 & 6 & 95 & atan2 & 0.001 & 78 & 1 & 1 & 0.180 \\
% 863 & 6 & 16 &  & 0.001 & 1798 & 2 & 16 & 0.174 \\
% 7 & 6 & 24 &  & 0.001 & 10 & 1 & 1 & 0.155 \\
% 865 & 6 & 16 &  & 0.001 & 330 & 2 & 2 & 0.141 \\
% 864 & 6 & 16 &  & 0.001 & 452 & 2 & 3 & 0.140 \\
% 861 & 6 & 18 &  & 0.001 & 123 & 1 & 1 & 0.131 \\
% 8 & 6 & 17 &  & 0.001 & 10 & 1 & 1 & 0.130 \\
\hline
\end{tabular}
  \end{center}
  \caption{
    Experimental results (Proved instances):
    ID = Problem ID,
    \#Var = Number of variables,
    \#Arith = Number of arithmetic operators,
    Nonlinear = Nonlinear operators occurred in problem,
    Proof Size = Number of lines of the proof,
    $\mathrm{TIME_S}$ = Solving time in seconds,
    \#Sub = Number of subproblems generated by proof checking,
    \#Axiom = Number of proved axioms,
    $\mathrm{TIME_{PC}}$ = Proof-checking time in seconds.
  }\label{tbl:exp}
  \vspace{-1cm}
\end{table}

%%% Local Variables:
%%% mode: latex
%%% TeX-master: "main"
%%% End:

\begin{table}
  \begin{center}
\begin{tabular}{|l||r|r|r||r|r|r|r|r|}
\hline
ID & \#Var & \#Arith & Nonlinear & Time$_{\text{S}}$ & Proof Size & \#Sub \\
\hline
\hline
260.smt2 & 6 & 90 &  & 5.030 & 6281203 & 1 \\
% 884.smt2 & 6 & 94 &  & 3.820 & 3829953 & 1 \\
% 491.smt2 & 6 & 92 &  & 1.630 & 1702874 & 1 \\
866.smt2 & 6 & 38 & sqrt & 0.390 & 543061 & 21476 \\
% 890.smt2 & 6 & 91 &  & 0.460 & 515537 & 18734 \\
% 868.smt2 & 6 & 52 &  & 0.360 & 350631 & 20120 \\
775.smt2 & 6 & 2765 & atan2,sqrt & 4.040 & 130253 & 2 \\
764.smt2 & 6 & 2767 & atan2,sqrt & 1.700 & 49657 & 2 \\
762.smt2 & 6 & 2767 & atan2,sqrt & 2.040 & 42238 & 2 \\
% 822.smt2 & 6 & 2743 & atan2,sqrt & 0.890 & 27959 & 2 \\
% 812.smt2 & 6 & 2737 & atan2,sqrt & 0.810 & 21836 & 2 \\
% 886.smt2 & 6 & 8112 & atan2,sqrt & 0.190 & 1493 & 2 \\
484.smt2 & 6 & 1835 & acos,atan2,sqrt & 0.060 & 16 & 1 \\
485.smt2 & 6 & 1961 & acos,atan2,sqrt & 0.070 & 16 & 1 \\
% 489.smt2 & 6 & 1865 & acos,atan2,sqrt & 0.060 & 16 & 1 \\
% 490.smt2 & 6 & 1991 & acos,atan2,sqrt & 0.070 & 16 & 1 \\
% 482.smt2 & 6 & 1158 & acos,atan2,sqrt & 0.030 & 15 & 1 \\
% 483.smt2 & 6 & 1221 & acos,atan2,sqrt & 0.040 & 15 & 1 \\
% 487.smt2 & 6 & 1883 & acos,atan2,sqrt & 0.060 & 15 & 1 \\
% 488.smt2 & 6 & 2009 & acos,atan2,sqrt & 0.070 & 15 & 1 \\
% 480.smt2 & 6 & 1823 & acos,atan2,sqrt & 0.020 & 13 & 1 \\
% 481.smt2 & 6 & 1949 & acos,atan2,sqrt & 0.030 & 13 & 1 \\
% 175.smt2 & 6 & 5236 & atan2,sqrt & 0.070 & 12 & 255 \\
% 459.smt2 & 6 & 3808 & acos,atan2,sqrt & 0.040 & 12 & 1 \\
% 460.smt2 & 6 & 3884 & acos,atan2,sqrt & 0.040 & 12 & 1 \\
% 476.smt2 & 6 & 4405 & acos,atan2,sqrt & 0.040 & 12 & 1 \\
% 477.smt2 & 6 & 3827 & acos,atan2,sqrt & 0.040 & 12 & 1 \\
% 478.smt2 & 6 & 4481 & acos,atan2,sqrt & 0.050 & 12 & 1 \\
% 479.smt2 & 6 & 3903 & acos,atan2,sqrt & 0.040 & 12 & 1 \\
498.smt2 & 6 & 573 & acos,matan,sqrt & 0.010 & 11 & 8191 \\
% 499.smt2 & 6 & 655 & acos,matan,sqrt & 0.020 & 11 & 8191 \\
% 500.smt2 & 6 & 492 & acos,matan,sqrt & 0.010 & 11 & 8191 \\
% 501.smt2 & 6 & 574 & acos,matan,sqrt & 0.010 & 11 & 8191 \\
% 502.smt2 & 6 & 574 & acos,matan,sqrt & 0.010 & 11 & 8191 \\
% 503.smt2 & 6 & 656 & acos,matan,sqrt & 0.020 & 11 & 8191 \\
% 801.smt2 & 6 & 208 & atan2,sqrt & 0.000 & 11 & 8171 \\
% 813.smt2 & 6 & 208 & atan2,sqrt & 0.000 & 11 & 8171 \\
% 497.smt2 & 6 & 496 & acos,matan,sqrt & 0.000 & 8 & 8191 \\
% 743.smt2 & 6 & 17 &  & 0.000 & 4 & 1 \\
% 10.smt2 & 6 & 6 &  & 0.000 & 0 & 1 \\
\hline
\end{tabular}
  \end{center}
  \caption{
    Experimental results (Unproved instances, Timeout = 300 sec):
    ID = Problem ID,
    \#Var = Number of variables,
    \#Arith = Number of arithmetic operators,
    Nonlinear = Nonlinear operators occurred in problem,
    Proof Size = Number of lines of the proof,
    $\mathrm{TIME_S}$ = Solving time in seconds,
    \#Sub = Number of subproblems generated by proof checking,
  }\label{tbl:exp}
\end{table}

%%% Local Variables:
%%% mode: latex
%%% TeX-master: "main"
%%% End:


%%% Local Variables:
%%% mode: latex
%%% TeX-master: "main"
%%% End:
