\section{Experiments}\label{kepler}
We implemented the proof generation capacity into our open-source solver {\sf
  dReal}\footnote{\url{http://dreal.cs.cmu.edu} }. All the experiments below are performed on a machine of with a 32-core 2.0GHz Intel
Xeon E5-2600 Processor and 64GB of RAM. The benchmarks and full tables of experiment statistics are also available on the tool page. 
\begin{table}
  \begin{center}
\begin{tabular}{|l||r|r|r||r|r|r|r|r|}
\hline
ID & \#Var & \#Arith & Nonlinear & Time$_{\text{S}}$ & Proof Size & \#Sub & \#Axiom & Time$_{\text{PC}}$ \\
\hline
\hline
461 & 6 & 36 &  & 1.740 & 2145155 & 2 & 17442 & 203.886 \\
789 & 6 & 86 & sqrt,atan2 & 1.640 & 350329 & 2 & 2464 & 128.077 \\
792 & 6 & 828 & sqrt,atan2 & 0.400 & 19837 & 2 & 118 & 113.004 \\
745 & 6 & 36 &  & 0.750 & 677580 & 2 & 5222 & 59.865 \\
785 & 6 & 80 & sqrt,atan2 & 0.470 & 63388 & 2 & 526 & 26.450 \\
760 & 6 & 2767 & sqrt,atan2 & 0.140 & 711 & 2 & 5 & 21.089 \\
820 & 6 & 95 & sqrt,atan2 & 0.080 & 9134 & 2 & 54 & 14.703 \\
815 & 6 & 95 & sqrt,atan2 & 0.330 & 41954 & 2 & 279 & 14.703 \\
814 & 6 & 95 & sqrt,atan2 & 0.350 & 42102 & 2 & 278 & 14.703 \\
816 & 6 & 96 & sqrt,atan2 & 0.110 & 12195 & 2 & 92 & 4.994 \\
817 & 6 & 96 & sqrt,atan2 & 0.090 & 11792 & 2 & 93 & 4.993 \\
784 & 6 & 80 & sqrt,atan2 & 0.060 & 7203 & 2 & 56 & 3.595 \\
781 & 6 & 86 & sqrt,atan2 & 0.060 & 7481 & 2 & 45 & 2.657 \\
793 & 6 & 834 & sqrt,atan2 & 0.020 & 18 & 1 & 1 & 1.855 \\
796 & 6 & 834 & sqrt,atan2 & 0.010 & 18 & 1 & 1 & 1.710 \\
752 & 6 & 17 &  & 0.080 & 46360 & 2 & 277 & 1.709 \\
783 & 6 & 825 &sqrt,atan2 & 0.020 & 93 & 1 & 1 & 1.549 \\
779 & 6 & 201 & sqrt,atan2 & 0.010 & 10 & 1 & 1 & 0.705 \\
867 & 6 & 17 &  & 0.040 & 25820 & 2 & 147 & 0.683 \\
% 493 & 6 & 155 & atan2 & 0.001 & 12 & 1 & 1 & 0.526 \\
% 492 & 6 & 155 & atan2 & 0.001 & 12 & 1 & 1 & 0.512 \\
% 806 & 6 & 96 & atan2 & 0.001 & 10 & 1 & 1 & 0.484 \\
% 494 & 6 & 59 & atan2 & 0.001 & 10 & 1 & 1 & 0.475 \\
% 843 & 6 & 97 & atan2 & 0.001 & 10 & 1 & 1 & 0.471 \\
% 852 & 6 & 97 & atan2 & 0.001 & 10 & 1 & 1 & 0.412 \\
% 842 & 6 & 97 & atan2 & 0.001 & 10 & 1 & 1 & 0.412 \\
% 858 & 6 & 97 & atan2 & 0.001 & 10 & 1 & 1 & 0.411 \\
% 855 & 6 & 97 & atan2 & 0.001 & 10 & 1 & 1 & 0.411 \\
% 853 & 6 & 97 & atan2 & 0.001 & 10 & 1 & 1 & 0.411 \\
% 850 & 6 & 97 & atan2 & 0.001 & 10 & 1 & 1 & 0.411 \\
% 845 & 6 & 97 & atan2 & 0.001 & 10 & 1 & 1 & 0.411 \\
% 844 & 6 & 97 & atan2 & 0.001 & 10 & 1 & 1 & 0.411 \\
% 840 & 6 & 97 & atan2 & 0.001 & 10 & 1 & 1 & 0.411 \\
% 839 & 6 & 97 & atan2 & 0.001 & 108 & 1 & 1 & 0.411 \\
% 857 & 6 & 97 & atan2 & 0.001 & 10 & 1 & 1 & 0.410 \\
% 847 & 6 & 97 & atan2 & 0.001 & 10 & 1 & 1 & 0.410 \\
% 841 & 6 & 97 & atan2 & 0.001 & 10 & 1 & 1 & 0.410 \\
% 838 & 6 & 97 & atan2 & 0.001 & 10 & 1 & 1 & 0.410 \\
% 791 & 6 & 95 & atan2 & 0.001 & 10 & 1 & 1 & 0.406 \\
% 755 & 6 & 59 & atan2 & 0.001 & 10 & 1 & 1 & 0.392 \\
% 833 & 6 & 95 & atan2 & 0.001 & 10 & 1 & 1 & 0.385 \\
% 827 & 6 & 95 & atan2 & 0.001 & 10 & 1 & 1 & 0.385 \\
% 823 & 6 & 95 & atan2 & 0.010 & 78 & 1 & 1 & 0.385 \\
% 836 & 6 & 95 & atan2 & 0.001 & 10 & 1 & 1 & 0.384 \\
% 835 & 6 & 95 & atan2 & 0.001 & 10 & 1 & 1 & 0.384 \\
% 834 & 6 & 95 & atan2 & 0.001 & 10 & 1 & 1 & 0.384 \\
% 825 & 6 & 95 & atan2 & 0.001 & 10 & 1 & 1 & 0.384 \\
% 751 & 6 & 171 & atan2 & 0.001 & 5 & 1 & 1 & 0.302 \\
% 809 & 6 & 89 & atan2 & 0.001 & 10 & 1 & 1 & 0.299 \\
% 750 & 6 & 170 & atan2 & 0.001 & 5 & 1 & 1 & 0.299 \\
742 & 6 & 55 & sqrt,acos,atan2 & 0.001 & 7 & 1 & 1 & 0.299 \\
% 506 & 6 & 49 &  & 0.001 & 7 & 1 & 1 & 0.293 \\
% 504 & 6 & 48 &  & 0.001 & 7 & 1 & 1 & 0.292 \\
% 505 & 6 & 48 &  & 0.001 & 7 & 1 & 1 & 0.291 \\
508 & 6 & 53 & sqrt,acos & 0.001 & 8 & 1 & 1 & 0.286 \\
507 & 6 & 29 & sqrt,acos & 0.001 & 8 & 1 & 1 & 0.278 \\
744 & 6 & 24 & asin,cos,sin & 0.001 & 8 & 1 & 1 & 0.275\\
% 740 & 6 & 9 &  & 0.001 & 5 & 1 & 1 & 0.272 \\
% 862 & 6 & 18 &  & 0.010 & 3752 & 2 & 18 & 0.233 \\
% 799 & 6 & 89 & atan2 & 0.001 & 10 & 1 & 1 & 0.216 \\
% 950 & 6 & 16 &  & 0.010 & 1798 & 2 & 16 & 0.206 \\
% 773 & 6 & 59 & atan2 & 0.001 & 10 & 1 & 1 & 0.206 \\
% 860 & 6 & 18 &  & 0.010 & 2197 & 2 & 13 & 0.203 \\
% 795 & 6 & 95 & atan2 & 0.001 & 10 & 1 & 1 & 0.202 \\
% 818 & 6 & 95 & atan2 & 0.001 & 63 & 1 & 1 & 0.182 \\
% 819 & 6 & 95 & atan2 & 0.001 & 78 & 1 & 1 & 0.180 \\
% 863 & 6 & 16 &  & 0.001 & 1798 & 2 & 16 & 0.174 \\
% 7 & 6 & 24 &  & 0.001 & 10 & 1 & 1 & 0.155 \\
% 865 & 6 & 16 &  & 0.001 & 330 & 2 & 2 & 0.141 \\
% 864 & 6 & 16 &  & 0.001 & 452 & 2 & 3 & 0.140 \\
% 861 & 6 & 18 &  & 0.001 & 123 & 1 & 1 & 0.131 \\
% 8 & 6 & 17 &  & 0.001 & 10 & 1 & 1 & 0.130 \\
\hline
\end{tabular}
  \end{center}
  \caption{
    Experimental results:
    ID = Problem ID,
    \#Var = Number of variables,
    \#Arith = Number of arithmetic operators,
    Nonlinear = Nonlinear operators occurred in problem,
    Time$_S$ = Solving time in seconds,
    Proof Size = Number of lines of the proof,
    $\mathrm{TIME_S}$ = Solving time in seconds,
    \#Axiom = Number of proved axioms,
    \#Sub = Number of subproblems generated by proof checking,
    $\mathrm{TIME_{PC}}$ = Proof-checking time in seconds.
  }\label{tbl:exp}
\end{table}

%%% Local Variables:
%%% mode: latex
%%% TeX-master: "main"
%%% End:

\begin{table}
  \begin{center}
\begin{tabular}{|l||r|r|r||r|r|r|r|r|}
\hline
ID & \#Var & \#Arith & Nonlinear & Time$_{\text{S}}$ & Proof Size & \#Sub \\
\hline
\hline
260.smt2 & 6 & 90 &  poly & 5.030 & 6281203 & 1 \\
\hline
% 884.smt2 & 6 & 94 &  & 3.820 & 3829953 & 1 \\
% 491.smt2 & 6 & 92 &  & 1.630 & 1702874 & 1 \\
866.smt2 & 6 & 38 & sqrt & 0.390 & 543061 & 21476 \\
\hline
% 890.smt2 & 6 & 91 &  & 0.460 & 515537 & 18734 \\
% 868.smt2 & 6 & 52 &  & 0.360 & 350631 & 20120 \\
775.smt2 & 6 & 2765 & atan2,sqrt & 4.040 & 130253 & 2 \\
\hline
764.smt2 & 6 & 2767 & atan2,sqrt & 1.700 & 49657 & 2 \\
\hline
762.smt2 & 6 & 2767 & atan2,sqrt & 2.040 & 42238 & 2 \\
% 822.smt2 & 6 & 2743 & atan2,sqrt & 0.890 & 27959 & 2 \\
% 812.smt2 & 6 & 2737 & atan2,sqrt & 0.810 & 21836 & 2 \\
% 886.smt2 & 6 & 8112 & atan2,sqrt & 0.190 & 1493 & 2 \\
\hline
484.smt2 & 6 & 1835 & acos,atan2,sqrt & 0.060 & 16 & 1 \\
\hline
485.smt2 & 6 & 1961 & acos,atan2,sqrt & 0.070 & 16 & 1 \\
% 489.smt2 & 6 & 1865 & acos,atan2,sqrt & 0.060 & 16 & 1 \\
% 490.smt2 & 6 & 1991 & acos,atan2,sqrt & 0.070 & 16 & 1 \\
% 482.smt2 & 6 & 1158 & acos,atan2,sqrt & 0.030 & 15 & 1 \\
% 483.smt2 & 6 & 1221 & acos,atan2,sqrt & 0.040 & 15 & 1 \\
% 487.smt2 & 6 & 1883 & acos,atan2,sqrt & 0.060 & 15 & 1 \\
% 488.smt2 & 6 & 2009 & acos,atan2,sqrt & 0.070 & 15 & 1 \\
% 480.smt2 & 6 & 1823 & acos,atan2,sqrt & 0.020 & 13 & 1 \\
% 481.smt2 & 6 & 1949 & acos,atan2,sqrt & 0.030 & 13 & 1 \\
% 175.smt2 & 6 & 5236 & atan2,sqrt & 0.070 & 12 & 255 \\
% 459.smt2 & 6 & 3808 & acos,atan2,sqrt & 0.040 & 12 & 1 \\
% 460.smt2 & 6 & 3884 & acos,atan2,sqrt & 0.040 & 12 & 1 \\
% 476.smt2 & 6 & 4405 & acos,atan2,sqrt & 0.040 & 12 & 1 \\
% 477.smt2 & 6 & 3827 & acos,atan2,sqrt & 0.040 & 12 & 1 \\
% 478.smt2 & 6 & 4481 & acos,atan2,sqrt & 0.050 & 12 & 1 \\
% 479.smt2 & 6 & 3903 & acos,atan2,sqrt & 0.040 & 12 & 1 \\
\hline
498.smt2 & 6 & 573 & acos,matan,sqrt & 0.010 & 11 & 8191 \\
% 499.smt2 & 6 & 655 & acos,matan,sqrt & 0.020 & 11 & 8191 \\
% 500.smt2 & 6 & 492 & acos,matan,sqrt & 0.010 & 11 & 8191 \\
% 501.smt2 & 6 & 574 & acos,matan,sqrt & 0.010 & 11 & 8191 \\
% 502.smt2 & 6 & 574 & acos,matan,sqrt & 0.010 & 11 & 8191 \\
% 503.smt2 & 6 & 656 & acos,matan,sqrt & 0.020 & 11 & 8191 \\
% 801.smt2 & 6 & 208 & atan2,sqrt & 0.000 & 11 & 8171 \\
% 813.smt2 & 6 & 208 & atan2,sqrt & 0.000 & 11 & 8171 \\
% 497.smt2 & 6 & 496 & acos,matan,sqrt & 0.000 & 8 & 8191 \\
% 743.smt2 & 6 & 17 &  & 0.000 & 4 & 1 \\
% 10.smt2 & 6 & 6 &  & 0.000 & 0 & 1 \\
\hline
\end{tabular}
  \end{center}
  \caption{
    Experimental results (Unproved instances, Timeout = 300 sec):
    ID = Problem ID,
    \#Var = Number of variables,
    \#Arith = Number of arithmetic operators,
    Nonlinear = Nonlinear operators occurred in problem,
    Proof Size = Number of lines of the proof,
    $\mathrm{TIME_S}$ = Solving time in seconds,
    \#Sub = Number of subproblems generated by proof checking,
  }\label{tbl:exp}
\end{table}

%%% Local Variables:
%%% mode: latex
%%% TeX-master: "main"
%%% End:


A main set of benchmarks that we studied is from the Flyspeck project~\cite{DBLP:conf/dagstuhl/Hales05,DBLP:conf/nfm/SolovyevH13}, which aims at a 
fully formalized proof of the Kepler conjecture. As lemmas for the
proof, hundreds of nonlinear real inequalities need to be verified.
Although the formulas usually contain only around ten variables, they
contain a huge number of nonlinear arithmetic operations and
trigonometric functions, and are mathematically challenging. In the original proof, Hales implemented procedures that combine
linear programming and interval arithmetic to establish all these
formulas, but the algorithms are formally verify. In
fact, the formal verification of these nonlinear inequalities is the last main piece
of work needed to complete the full project. 
%update this section
Without any particular optimization on ICP, we have observed
promising results. Out of 916 nonlinear formulas in the Flyspeck
project repository, the solver returns {\sf unsat} for 107 of them with a timeout of 5 minute each, and a precision $\delta=10^{-3}$. Out of these formulas, we automatically generated and validated the proofs for 72 instances. The proof traces of these formulas can be very
large; for instance, we proved one with more than 2M lines in the
proof (54MB file). In Table~\ref{tbl:exp}, we list some of the
representative benchmarks to show scalability. Many of these formulas are highly nonlinear, for instance the formula encoded in 760.smt2 is following one 
\begin{multline*}
\forall\vec{x} \in [4.0, 6.3504]^5\; \Big(2\mathrm{arctan} (\frac{\Delta_2(\vec x)}{\sqrt{\Delta_1(\vec x) + \Delta_2(\vec x)^2} + \sqrt{\Delta_1(\vec x)}})\\
- 0.458(\sqrt{x_2} + \sqrt{x_3} +\sqrt{x_4} + \sqrt{x_5}) + 0.342\sqrt{x_1} + 3.319204\Big) < 0.0
\end{multline*}
where
\begin{eqnarray*}
%give this definition
  \Delta_1(\vec{x}) &=& 4x_1 (8x_1 (-x_1 + x_2 + x_3 + x_4 + x_5 - 8)\\
   & &+ x_2 x_5 (x_1 - x_2 + x_3 + x_4 - x_5+8\\%sure there's a plus 8 here?
& &+ x_3x_4(x_1 + x_2 - x_3 - x_4 + x_5 + 8)+ 8 x_2 x_3 \\
& &- x_1 x_3 x_5 - x_1  x_2  x_4 - 8 x_4 x_5))\\
\Delta_2(\vec{x}) &=& x_2 x_5 -x_2 x_3 + x_3x_4 - x_4 x_5 +x_1^2 -x_1x_2\\
& & - x_1x_3 - x_1x_4 -x_1 x_5
\end{eqnarray*}

On the other hand, as mentioned above, we fail to establish about the proofs of unsatisfiability of about 30 instances. Table 2 shows some of these instances. They typically generate proofs that are large in size, or that the branch-and-prove loop has to generate too many sub-instances such that the proof checking can not terminate. 

%%% Local Variables:
%%% mode: latex
%%% TeX-master: "main"
%%% End: