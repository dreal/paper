\section{Probabilistic bounded reachability for hybrid systems with parametric uncertainty}

\subsection{Hybrid automata with parametric uncertainty}
The purpose of this subsection is to present a model for hybrid systems with probabilistic system parameters using the framework of hybrid automata in \cite{henzinger2000theory}. The difference between a hybrid automaton and one with parametric uncertainty lies in the existence of random variables as probabilistic system parameters (See Definition \ref{def:ha} for details).
\begin{definition}
\label{def:ha}
{\rm(Hybrid Automata with Parametric Uncertainty)} A hybrid automaton with probabilistic parameters $H_p$ consists of the following components.

{\bf Variables}. A finite set $X = \{ x_1, \cdots, x_n \}$ of real-numbered variables, where $n$ is the dimension of $H_p$. We write $\dot{X}$ for the set $\{\dot{x_1}, \cdots, \dot{x_n}\}$ to represent first derivatives of variables during the continuous change, and write $X'$ for the set $\{x_1', \cdots, x_n'\}$ to denote values of variables at the conclusion of the discrete change.

{\bf Random Variables}. A finite set $Y = \{ y_1, \cdots, y_m \}$ of discrete and continuous random variables, and the corresponding distribution set $Distr=\{distr_1, \cdots,\\ distr_m\}$, where, for $0 \le i \le m$, $y_i$ has the corresponding distribution $distr_i$. These random variables are used to describe the system uncertainty caused by probabilistic parameters.

{\bf Control graph}. A finite directed multigraph $(V,E)$. The vertices in $V$ are called control modes, and edges in $E$ are control switches.

{\bf Initial, invariant, and flow conditions}. As vertex labeling functions over each control mode $v \in V$, the initial condition $init(v)$ is predicate whose free variables are from $V$, the invariant condition $inv(v)$ is a predicate whose free variables are from $X$, and the flow condition $flow(v)$ is a predicate whose free variables are from $X \cup \dot{X}$.

{\bf Jump conditions}. An edge labeling function $jump$ that assigns to each control switch $e \in E$ a predicate whose free variables are from $X \cup X'$.

{\bf Events}. A finite set $\Sigma$ of events, and an edge labeling function $event: E \to \Sigma$ that assigns to each control switch an event. 
 
\end{definition}

\subsection{Probabilistic bounded reachability}
With respect to system analysis, we are interested in the probability of reaching the target states within a bounded number of transition steps. We now formally state the probabilistic reachability problem for hybrid automata with parametric uncertainty. 

Let $H$ be a hybrid automaton without parametric uncertainty, and $H_p$ be a system of hybrid automata with parametric uncertainty as in Definition \ref{def:ha}. As for the set of random variables $Y_{H_p}$, let $\Omega_{H_p}$ be the corresponding sample space. we write $S$ as a random sampler with respect to the joint distribution of all random variables - $S: Y_{H_p} \to \omega_{H_p}$. For each sample $s_i$ generated by $S$, we write $H_{p}[s_i/Y_{H_p}]$ as the corresponding hybrid automaton without parametric uncertainty, where the set of random variables $Y_{H_p}$ has been replaced by their corresponding sampled values in $s_i$. Let $k \in N$ be a step bound, and $R_{H}^k$ denotes the set of all initialized trajectories of length $k$ of $H$. Also, we write $Goal$ be the set of target states.
\begin{remark}
\label{def:br}
{\rm (Bounded Reachability for Hybrid Automata)}
The bounded reachability problem for $H$ asks if there is a trace $\sigma \in R_{H}^k$ that visits the $Goal$.
\end{remark}
\begin{definition}
\label{def:pbr}
{\rm (Probabilistic Bounded Reachability for Hybrid Automata with Parametric Uncertainty)}\\
The probabilistic bounded reachability for $H_p$ estimates the maximum probability, over $\Omega_{H_p}$, of the bounded reachability for $H_{p}[s_i/Y_{H_p}]$, where $s_i \in S(Y_{H_p})$.
\end{definition}

\subsection{The main algorithm}.
To solve the probabilistic bounded reachability problem for a given hybrid model with probabilistic parameters $MP$, our method first samples the involved set of random variables, denoted as $RV$, according to their distributions. Then, according to each sample $S_i$, we obtain a model of the hybrid system without any probabilistic parameters $M_i$. {\bf SReach} then calls {\bf dReach} \cite{gaodelta} with the precision $\delta$ and the unfolding steps $k$. {\bf dReach} is a bounded reachability analyzer based on {\bf dReal} \cite{gao2013dreal}, and returns either unsat or $\delta$-sat for $M_i$. With a sufficient number of sampled models, and a specified statistical testing method $ST$, {\bf SReach} terminates the entire procedure. It either returns the maximal probability of the system satisfying the given reachability property, or accepts or rejects according to the comparison whether the returned probability is larger (or smaller) than the specified threshold. Algorithm \ref{fig:sreach} illustrates the main algorithm of our approach.

\begin{algorithm}
  \centering
  \caption{SReach}
  \label{fig:sreach}
  \begin{algorithmic}[1]
    \Procedure{SReach}{$MP$, $ST$, $\delta$, $k$}
        \State $Succ \gets 0$
        \State $Fail \gets 0$
        \State $RV \gets \mathrm{ExtractRV}(MP)$
        \Repeat
        	   \State $S_i \gets \mathrm{Sim}(RV)$
            \State $M_i \gets \mathrm{Gen}(MP, S_i)$
            \State $Res \gets \mathrm{dReach}(M_i, \delta, k)$
            \If{$Res = sat$}
            	%Good
		\State $Succ \gets Succ + 1$
	  \Else
	   	%Bad
		\State $Fail \gets Fail + 1$
	  \EndIf
        \Until{$ST.done$}
        \State $Est\_prob \gets Succ / (Succ + Fail)$
   \EndProcedure
  \end{algorithmic}
\end{algorithm}

\subsection{Statistical tests}.
Consider a hybrid system with parametric uncertainty $H$ and a reachability property $\phi$. Algorithm \ref{fig:sreach} can be used to answer two types of questions depending on which statistical testing technique is chosen: (1) Qualitative: Is the probability for $H$ to satisfy $\phi$ greater or less than a certain threshold? and
(2) Quantitative: What is the probability for $H$ to satisfy $\phi$? Similar to SMC, the answer is given up to some correctness precision because of the use of statistical tests.  To deal with qualitative questions, we consider the following hypothesis testing methods.

\textit{Lai's test} \cite{lai1988nearly}.
As a simple class of sequential tests, it tests the one-sided composite hypotheses $H_0: \; \theta \leq \theta_0$ versus $H_1:\; \theta \geq \theta_1$ for the natural parameter $\theta$ of an exponential family of distributions under the $0-1$ loss and cost $c$ per observation. \cite{lai1988nearly} shows that these tests have nearly optimal frequentist properties and also provide approximate Bayes solutions with respect to a large class of priors. 

\textit{Bayes factor test} \cite{kass1995bayes}.
The use of Bayes factors is a Bayesian alternative to classical hypothesis testing. It is based on the Bayes' theorem. Hypothesis testing with Bayes factors is more robust than frequentist hypothesis testing, as the Bayesian form avoids model selection bias, evaluates evidence in favor the null hypothesis, includes model uncertainty, and allows non-nested models to be compared. Also, frequentist significance tests become biased in favor of rejecting the null hypothesis with sufficiently large sample size. 

\textit{Bayes factor test with indifference region}. 
A hypothesis test has ideal performance if the probability of the Type-I error (respectively, Type-II error) is exactly $\alpha$ (respectively, $\beta$). However, these requirements make it impossible to ensure a low probability for both types of errors simultaneously (see \cite{younes2005verification} for details). A solution is to use an indifference region. The indifference region indicates the distance between two hypotheses, which is set to separate two hypotheses.

\textit{Sequential probability ratio test (SPRT)} \cite{wald1945sequential}. 
As for he SPRT, we consider a simple hypothesis $H_0:\;\theta = \theta_0$ against a simple alternative $H_1:\;\theta = \theta_1$. With the critical region $\Lambda_n$ and two thresholds $A$, and $B$, SPRT decides that $H_0$ is true and stops when $\Lambda_n < A$. It decides that $H_1$ is true and terminates if $\Lambda_n > B$. If $A\; < \Lambda_n < B$, it will collect another observation to obtain a new critical region $\Lambda_{n+1}$. The SPRT is optimal, among all sequential tests, in the sense that it minimizes the average sample size. In {\bf SReach}, we have implemented SPRT with indifference region.

To offer quantitative answers, {\bf SReach} also supports estimation procedures as below.

\textit{Chernoff-Hoeffding bound} \cite{hoeffding1963probability}. To estimate the probability $p$ for $H$ to satisfy $\phi$, given a precision $\delta'$, the Chernoff-Hoeffding bound can given a value $p'$ such that $|p' \; - \; p| \le \delta'$ with confidence $1\;-\; \alpha$.

\textit{Bayesian Interval Estimation with Beta prior} \cite{zuliani2010bayesian}. This method estimates $p$, the unknown probability that a random sampled model satisfies a specified reachability property. The estimate will be in the form of a confidence interval, congaing $p$ with an arbitrary high probability.  \cite{zuliani2010bayesian} assumes that the unknown $p$ is given by a random variable, whose density is called the prior density, and focuses on Beta priors. It has been showed that, with this Bayesian interval estimation method, the probability of giving a wrong answer is arbitrarily small, and speed of obtaining an answer is higher than the sequential hypothesis testing.