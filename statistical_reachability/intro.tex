\section{Introduction} 
\vspace{-.2cm} 
Hybrid systems, as models exhibiting both continuous and discrete dynamic behavior, have become 
a widely used modeling formalism for real-world safety-critical systems,
including for example cyber-physical systems, chemical-physical process control, and biomedical systems. 
The core of hybrid system analysis studies is to construct accurate computational models for real-world systems, 
and to verify that they meet their design requirements. Simulation-based testing 
is the most used approach regardless of its incompleteness. Reachability analysis computes instead set over-approximations that
cover all system behaviors, which lead to a better input coverage but are naturally harder to scale.
It is important to note that reachability analysis can handle models with nondeterminism, and in many cases aim for an infinite time horizon.

We consider the {\em probabilistic bounded reachability problem}, which is to decide whether
a hybrid system with probabilistic system parameters reaches an unsafe region of the
state space within a finite number of steps with a probability greater than a given threshold.
Although uncertainty raises naturally and the cause for the parametric uncertainty is multifaceted, two factors 
are critical when building hybrid models for real-world systems. 
First, probabilistic parameters are needed when the physics controlling the system is known, but some 
parameters are either not known precisely, or are expected to vary from individual to individual, 
or may change by the end of the system operational lifetime. 
Second, system uncertainty may occur when the model is constructed or learned directly from experimental data. 
Due to imprecise experimental measurements, the values of system parameters may have ranges of variation 
with some associated likelihood of occurrence. 
In both cases we assume that, for probabilistic system parameters, their probability distributions
are known, and it is desired to design models which achieve specified performance for these variations.   

We describe our tool {\bf SReach} that integrates the existing $\delta$-complete bounded 
reachability analysis technique \cite{gaodelta} with statistical testing, in order
to support bounded probabilistic reachability analysis for hybrid systems with parametric uncertainty. 
%It helps developers, and researchers in estimating appropriate values, ranges, or distributions for system parameters when constructing models for probabilistic hybrid systems, and in validating hybrid models with probabilistic parameters according to given reachability properties. 
We show experiment results of {\bf SReach} for the probabilistic reachability analysis of many realistic hybrid systems that are highly nonlinear and nondeterministic, including a prostate cancer treatment model and cardiac atrial fibrillation model. 

\paragraph{Related Work.}To verify the correctness of stochastic hybrid models, there exist several techniques and tools, such as simulation-based verification \cite{zuliani2010bayesian, abate2007probabilistic}, logic-based verification \cite{platzer2011stochastic}, constraint solving \cite{franzle2008stochastic}, and the probabilistic model checker PRISM \cite{website:prism}. However, it is still difficult to formally analyze stochastic hybrid systems with nonlinear dynamics and complex discrete controls \cite{alur2011formal, henzinger2000theory}. Theoretically, it is well known that the safety verification problem for hybrid systems with simple dynamics is highly undecidable. Consequently, a unified framework for solving the reachability problem seems impossible, especially for nonlinear hybrid systems. In details, a major difficulty stems from the need of solving logic formulae with nonlinear functions over the reals. Recently, \cite{gao2013dreal, gao2013satisfiability} 
have defined a sound relaxation of this problem and associated decision procedures that always decide correctly 
if, for a given combination of parameters, the system actually reaches the unsafe region (in the opposite case
false positives may be generated, but this can be controlled by a precision parameter $\delta>0$).
The $\delta${\em -decision problem} is decidable for bounded first-order formulae over the reals with arbitrary 
Type 2 computable functions, which include almost all functions used in realistic hybrid systems.   
