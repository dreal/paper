\section{Introduction}  
Hybrid systems, as models exhibiting both continuous and discrete dynamic behavior, have become a widely used modeling formalism. Their real-world safety-critical applications involve areas from cyber-physical systems, to chemical-physical process control, and to biomedical systems. The core of hybrid systems analysis studies is to construct accurate computerized models for real-world systems, and to validate that they are guaranteed to meet the design goals. To validate hybrid models, alternative to the simulation-based testing which is an inherently incomplete technique, the reachability analysis overwhelms it in the following aspects. For one thing, the reachability analysis offers an over-approximative set-valued cover of all system behaviors, which leads to a better coverage rate. For another, the reachability analysis has a high scalability, as it can handle models with finite time or infinite time, and models with nondeterminism as well. In this paper, we address the bounded reachability problem for hybrid systems with probabilistic uncertainty in system parameters as our first step to handle the reachability problem for the general stochastic hybrid systems. Although uncertainty raises naturally, and the cause for the parametric uncertainty is multifaceted, two factors are critical when building hybrid models for real-world systems. First of all, the probabilistic parameters are needed when the physics controlling the system is known, while some system parameters are either not known precisely, or are expected to vary from individual to individual, or may change by the end of the operational lifetime. The system uncertainty also happens when the model is constructed or learned directly from experimental data. Due to experimental measurements, the values of system parameters may have a range of variations with chances of occurrence. In both cases, it is assumed that, for probabilistic system parameters, the range of variation and probability density functions are known, and it is desired to design models which achieve specified performance for these variations.   

To verify the correctness of stochastic hybrid models, researchers have proposed several techniques, such as simulation-based verification \cite{zuliani2010bayesian, abate2007probabilistic}, logic-based verification \cite{platzer2011stochastic}, and constraint solving \cite{franzle2008stochastic}. However, it is still difficult to formally analyze stochastic hybrid systems with nonlinear dynamics and complex discrete controls \cite{alur2011formal, henzinger2000theory}. Theoretically, it is well known that the safety verification problem for hybrid systems with simple dynamics is highly undecidable. Consequently, a unified framework for solving the reachability problem seems impossible, especially for nonlinear hybrid systems. In details, a major difficulty stems from the need of solving logic formulae with nonlinear functions over the reals. Recently, \cite{gao2013dreal, gao2013satisfiability} have defined the $\delta-decision \;  problem$ which easies the solving to a great extend. The $\delta$-decision problem is decidable for bounded first-order formulae over the reals with arbitrary Type 2 computable functions, which covers almost all functions used in realistic hybrid systems.   

This paper presents a tool - {\bf SReach}, which leverages on the existing verification techniques for hybrid systems, and extends them to support the bounded probabilistic reachability analysis for general hybrid systems with parametric uncertainty. The tool integrates the existing $\delta$-decision based bounded reachability analysis technique \cite{gaodelta} with distinct statistical testing methods. It helps developers, and researchers in estimating appropriate values, ranges, or distributions for system parameters when constructing models for probabilistic hybrid systems, and in validating hybrid models with probabilistic parameters according to given reachability properties. We used this tool for reachability analysis of two representative examples - the prostate cancer treatment control and the cardiac system, and of additional 19 benchmarks.