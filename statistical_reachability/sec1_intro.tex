\section{Introduction} 
% motivation for considering shs
% motivation for considering probabilistic reachability
Stochastic hybrid systems (SHSs) are dynamical systems, exhibiting discrete, continuous, and stochastic dynamics. Due to the generality, SHSs have been widely used in various areas, including cyber-physical systems, financial decision problems, chemical-physical process control, and biological systems \cite{blom2006stochastic, clarke2011statistical}. The popularity of SHSs in real-world applications motivates researchers to put a significant effort into the analysis methods for this class of systems. One of the elementary questions for the quantitative analysis of SHSs is the probabilistic reachability problem. It is to compute the probability of reaching a certain set of states. The set may represent certain unsafe states which should be avoided or visited only with some small probability, or dually, good states which should be visited frequently. There are two reasons why this kind of problems catches researchers' attention. One is that most temporal properties can be reduced to reachability problems, considering the very expressive hybrid modeling framework. The other is that, probabilistic state reachability is a hard and challenging problem which is undecidable in general. 

% types of shs
To describe stochastic dynamics, uncertainties have been added to general hybrid systems in a number of different ways. One simplest way replaces some constant system parameters with random variables, resulting in general hybrid automata (GHAs) with parametric uncertainty. Another approach integrates deterministic flows with probabilistic jumps. When state changes forced by continuous dynamics involve discrete random events, we refer to them as probabilistic hybrid automata (PHAs) \cite{sproston2000decidable}. When state changes also involve continuous probabilistic events, we call this kind of model stochastic hybrid automata (SHAs) \cite{franzle2011measurability}. Other models describe randomness by substituting deterministic flows with stochastic ones, such as stochastic differential equations (SDEs) \cite{ludwiga1974sde} and stochastic hybrid programs (SHPs) \cite{platzer2011stochastic}, where the random perturbation affects the dynamics continuously. When all such modifications have been applied, the resulting models are called the general stochastic hybrid systems (GSHSs) \cite{hu2000towards}. Among these different models, of particular interest for this paper are GHAs with parametric uncertainty and PHAs with additional randomness of both transition probabilities and resets. 


% motivation for considering GHAs with parametric uncertainty
% motivation for considering PHAs 
When modeling real-world systems using hybrid models, parametric uncertainty arises naturally. Although its cause is multifaceted, two factors are critical. On the one hand, probabilistic parameters are needed when the physics controlling the system is known, but some parameters are either not known precisely, or are expected to vary because of individual differences, or may change by the end of the system's operational lifetime. On the other hand, system uncertainty may occur when the model is constructed or learned directly from experimental data. Due to imprecise experimental measurements, the values of system parameters may have ranges of variation with some associated likelihood of occurrence. Clearly, the GHAs with parametric uncertainty are suitable models considering these major causes. Note that, in both cases, we assume that the probability distributions of probabilistic system parameters are known, and it is desired to design models which achieve specified performance for these variations. Another interesting and more expressive class of models is PHAs, which extends the model of hybrid automata \cite{henzinger2000theory} with discrete probability distributions. More precisely, for discrete transitions in a model, instead of making a purely nondeterministic choice over the set of currently enabled jumps, a PHA nondeterministically chooses among the set of recently enabled discrete probability distributions, each of which is defined over a set of transitions. Although randomness is defined to only influence discrete dynamics of the model, PHAs are still very useful and have interesting practical applications \cite{spr2001thesis}. In this paper, we consider a variation of PHAs, where additional randomness of both transition probabilities and resets of some system variables are allowed. In other words, in terms of the randomness on jump probabilities, we mean that the probabilities attached to probabilistic jumps from one mode, instead of obeying to a discrete distribution with predefined constant probabilities, can be expressed by equations involving random variables whose distributions can be either discrete or continuous. This extension is motivated by the fact that some transition probabilities can vary due to factors such as individual differences in the real-world systems. When it comes to the randomness of transition resets, we allow a system variable can be reset to a value obtained according to a known discrete or continuous distribution, instead of being assigned with a fixed value. For example, with this extension, on update, variable $t$ can become any value between $1$ and $2$ with equal probability. 

{\it {\bf Related Work.}} Analysis approaches for GSHSs are often based on Monte-Carlo simulation \cite{blom2004particle}. Considering the hardness dealing with this general case, efforts have been mainly placed on different subclasses. For PHAs, Zhang et al. \cite{zhang2012safety} abstracted the original PHA to a probabilistic automaton (PA), and then used the established Model Checking methods (e.g. PRISM \cite{website:prism}) for the abstracting model. Hahn et al. also discussed an abstraction-based method where the given PHA was translated into a $n$-player stochastic game using two different abstraction techniques \cite{hahn2011game}. Another method proposed is a SMT-based bounded Model Checking procedure \cite{franzle2008stochastic}. In \cite{amin2006reachability, abate2007probabilistic, abate2011two, abate2011quantitative}, a similar class of models called discrete-time stochastic hybrid systems (DTSHSs) is widely used in the control theory was considered. With regard to the system analysis, the control problem is to find an optimal control policy that minimizes the probability of reaching unsafe states. Zuliani et al. also mentioned a simulation-based method for model checking DTSHSs against bounded temporal properties \cite{zuliani2010bayesian}. We refer to this method as Statistical Model Checking (StatMC).  Although StatMC does not belong to the class of exhaustive state-space exploration methods, it usually returns results faster than the exhaustive search with a predefined arbitrarily small error bound on the estimated probability. StatMC was integrated into UPPAAL \cite{larsen1997uppaal} in order to handle very general networks of SHAs recently \cite{david2012statistical}. To analyze reachability problems of stochastic hybrid automata (SHAs), in \cite{franzle2011measurability}, a given SHA was first over-approximated by a PHA, and then the verification procedure introduced in \cite{zhang2012safety} was exploited to model check the over-approximating PHA. Plazter introduced another interesting modeling formalism - stochastic hybrid programs (SHPs) in \cite{platzer2011stochastic}. This formalism is quite expressive on randomness: it takes stochastic differential equations, discrete probabilistic branching, and random assignments to real-valued variables into account. To specify system properties, Platzer proposed a logic called stochastic differential dynamic logic, and then suggested a proof calculus to verify logical properties of SHPs. 


% briefly introduce SReach algorithms
In this paper, we describe our tool {\it SReach} which supports bounded probabilistic reachability analysis for the above two interesting model classes. It harvests and combines the recently proposed $\delta$-complete bounded reachability analysis technique \cite{gaodelta} with the statistical testing. Our technique saves the virtues of the Satisfiability Modulo Theories (SMT) based Bounded Model Checking for GHAs \cite{cordeiro2012smt, tinelli2012smt}, namely the fully symbolic treatment of hybrid state spaces, while advancing the reasoning power to probabilistic models and requirements. Moreover, by adopting statistical tests, {\it SReach} can place a predefined arbitrarily small error bound on the estimated probability compared to the real value. Comparing to the existing tools introduced in \cite{zhang2012safety, franzle2008stochastic, david2012statistical, website:prism}, besides offering a sound way to analyze nonlinear dynamics within the SHSs, {\it SReach} also supports probabilistic bounded reachability analysis for hybrid systems with parametric uncertainty. Furthermore, for PHAs, {\it SReach} considers a more general and useful formalism where general randomness of transition probabilities and variable resets are allowed. We discuss three biological models - a prostate cancer treatment model, a cardiac atrial fibrillation model, and our synthesized Killerred biological model - to show how {\it SReach} can be used to answer several types of questions including model validation, parameter estimation, and sensitivity analysis. To further demonstrate the feasibility of {\it SReach}, we also present experimental results for additional real-world hybrid systems that are highly nonlinear and nondeterministic.

%Over the last decade, research efforts concerning SHSs are rapidly increasing. At the same time, Model Checking methods and tools for probabilistic systems, such as PRISM \cite{website:prism}, have been proposed. However, results related to the analysis and verification of SHSs are still limited. For instance, analysis approaches for GSHSs are often based on Monte-Carlo simulation \cite{blom2004particle}. Considering the hardness dealing with the general class, efforts have been mainly placed on different subclasses. For PHAs, Zhang et al. abstracted the original PHA to a probabilistic automaton (PA), and then used the established Model Checking methods for the abstracting model \cite{zhang2012safety}. Hahn et al. also discussed an abstraction-based method where the given PHA was translated into a $n$-player stochastic game using two different abstraction techniques \cite{hahn2011game}. Another method proposed is a SMT-based bounded Model Checking procedure \cite{franzle2008stochastic}. \cite{amin2006reachability, abate2007probabilistic, abate2011two, abate2011quantitative} considered a similar class of models that is widely used in the control theory, which is called discrete-time stochastic hybrid systems (DTSHSs). With regard to the system analysis, the control problem concerned can be understood as to find an optimal control policy that minimizes the probability of reaching unsafe states. Zuliani et al. also mentioned a simulation-based method for model checking DTSHSs against bounded temporal properties \cite{zuliani2010bayesian}. We refer to this method as Statistical Model Checking (StatMC).  Although this statistical model checking procedure does not belong to the class of exhaustive state-space exploration methods, it usually returns results faster than the exhaustive search with a predefined arbitrarily small error bound on the estimated probability. StatMC is currently integrated into UPPAAL in order to handle very general networks of SHAs \cite{david2012statistical}. To analyze reachability problems of stochastic hybrid automata (SHAs), in \cite{franzle2011measurability}, a given SHA is firstly over-approximated by a PHA, and then the verification procedure introduced in \cite{zhang2012safety} is exploited to model check the over-approximating PHA. Another interesting work is about stochastic hybrid programs (SHPs) introduced in \cite{platzer2011stochastic}. This formalism is quite expressive on randomness. To specify system properties, Platzer proposed a logic called stochastic differential dynamic logic, and then suggested a proof calculus to verify logical properties of SHPs. 

%Akin to PHAs, DTSHSs comprise nondeterministic as well as discrete probabilistic choices of state transitions. Unlike PHAs, DTSHSs are sampled at discrete time points, use control inputs to model nondeterminism, do not have an explicit notion of symbolic transition guards, and support a more general concept of randomness which can describe discretized stochastic differential equations. With regard to the system analysis, the control problem concerned can be understood as to find an optimal control policy that minimizes the probability of reaching unsafe states. A backward recursive procedure which is also called dynamic programming scheme was then proposed to solve the problem \cite{amin2006reachability, abate2007probabilistic}. Another approach to a very similar problem as above, where a DTSHS model is without nondeterministic control inputs, was presented in \cite{abate2011two}. Comparing to former method where the grid is used to generate and numerically solve a discretized dynamic programming scheme, the latter approach exploits the grid to construct a discrete-time Markov chain (DTMC), and then employs standard model checking procedures for the DTMC. This approach then had been used in \cite{abate2011quantitative} as an analysis procedure for the probabilistic reachability problems in the product of a DTSHS and a B\"{u}chi automaton representing a linear temporal property. Zuliani et al. also mentioned a simulation-based method for model checking DTSHSs against bounded temporal properties \cite{zuliani2010bayesian}. We refer to this method as Statistical Model Checking (StatMC). The main idea of StatMC is to generate enough simulations of the system, record the checking result returned from a trace checker from each simulation, and then use statistical testing and estimation methods to determine, with a predefined degree of confidence, whether the system satisfies the property. Although this statistical model checking procedure does not belong to the class of exhaustive state-space exploration methods, it usually returns results faster than the exhaustive search with a predefined arbitrarily small error bound on the estimated probability. StatMC is currently integrated into UPPAAL in order to handle very general networks of SHAs \cite{david2012statistical}. In \cite{franzle2011measurability}, as an extension of PHAs, stochastic hybrid automata (SHAs) allow continuous probability distributions in the discrete state transitions. With respect to the verification procedure, a given SHA is firstly over-approximated by a PHA via discretizing continuous distributions into discrete ones with the help of additional uncountable nondeterminism. As mentioned, this over-approximation preserves safety properties. For the second step, the verification procedure introduced in \cite{zhang2012safety} is exploited to model check the over-approximating PHA. Another interesting work is about stochastic hybrid programs (SHPs) introduced in \cite{platzer2011stochastic}. This formalism is quite expressive on randomness: it takes stochastic differential equations, discrete probabilistic branching, and random assignments to real-valued variables into account. However, nondeterminism and parallel composition are not considered. To specify system properties, Platzer proposed a logic called stochastic differential dynamic logic, and then suggested a proof calculus to verify logical properties of SHPs. 

The paper proceeds by first introducing two modeling formalisms of SHSs under consideration - GHAs with parametric uncertainty, and PHAs with additional randomness in Section 2. Section 3 explains how {\it SReach} encodes stochastic dynamics and combines SMT-based BMC with statistical tests. Case studies and additional experiments are discussed in Section 4. Section 5 concludes the paper.
