\section{Introduction} 
% motivation for considering shs
% motivation for considering probabilistic reachability
Stochastic hybrid systems (SHSs) are dynamical systems exhibiting discrete, continuous, and stochastic dynamics. Due to their generality, SHSs have been widely used in various areas, including cyber-physical systems, financial decision problems, and biological systems \cite{blom2006stochastic, clarke2011statistical}. The popularity of SHSs in real-world applications motivates researchers to put a significant effort into analysis methods for this class of systems. One of the elementary questions for the quantitative analysis of SHSs is the probabilistic reachability problem, i.e., computing the probability of reaching a certain set of states. The set may represent unsafe states which should be avoided or visited only with some small probability, or dually, good states which should be visited frequently. There are two reasons why this kind of problem catches the researchers' attention. One is that most temporal properties can be reduced to reachability problems, considering the very expressive hybrid modeling framework. The other is that probabilistic state reachability is a hard and challenging problem which is undecidable in general. 

\vspace{-.2cm}
% types of shs
To describe stochastic dynamics, uncertainties have been added to hybrid systems in a number of different ways. One of the simplest ways replaces some of the system parameters with random variables, resulting in general hybrid automata (GHAs) with parametric uncertainty. Another approach integrates deterministic flows with probabilistic jumps. When state changes forced by continuous dynamics involve discrete random events, we refer to such systems as probabilistic hybrid automata (PHAs) \cite{sproston2000decidable}. When state changes also involve continuous probabilistic events, we call this kind of models stochastic hybrid automata (SHAs) \cite{franzle2011measurability}. Other models describe randomness by substituting deterministic flows with stochastic ones, such as stochastic differential equations (SDEs) \cite{ludwiga1974sde}, where the random perturbation affects the dynamics continuously. When all such modifications have been applied, the resulting models are called general stochastic hybrid systems (GSHSs) \cite{hu2000towards}. Among these different models, of particular interest for this paper are GHAs with parametric uncertainty and PHAs with additional randomness for both transition probabilities and variable resets. 

\vspace{-.2cm}
% motivation for considering GHAs with parametric uncertainty
% motivation for considering PHAs 
When modeling real-world systems using hybrid models, parametric uncertainty arises naturally. Although its cause is multifaceted, two factors are critical. First, probabilistic parameters are needed when the physics controlling the system is known, but some parameters are either not known precisely, are expected to vary because of individual differences, or may change by the end of the system's operational lifetime. Second, system uncertainty may occur when the model is constructed directly from experimental data. Due to imprecise experimental measurements, the values of system parameters may have ranges of variation with some associated likelihood of occurrence. Clearly, the GHAs with parametric uncertainty are suitable models considering these major causes. Note that, in both cases, we assume that the probability distributions of probabilistic system parameters are known. Another interesting and more expressive class of models is PHAs, which extends hybrid automata \cite{henzinger2000theory} with discrete probability distributions. More precisely, for discrete transitions in a model, instead of making a purely nondeterministic choice over the set of currently enabled jumps, a PHA nondeterministically chooses among the set of recently enabled discrete probability distributions, each of which is defined over a set of transitions. Although randomness is defined to only influence the discrete dynamics of the model, PHAs are still very useful and have interesting practical applications \cite{spr2001thesis}. In this paper, we consider a variation of PHAs, where additional randomness for both transition probabilities and resets of some system variables are allowed. In other words, in terms of the randomness for jump probabilities, we mean that the probabilities attached to probabilistic jumps from one mode, instead of having a discrete distribution with predefined constant probabilities, can be expressed by equations involving random variables whose distributions can be either discrete or continuous. This extension is motivated by the fact that some transition probabilities can vary due to factors such as individual and environmental differences in real-world systems. When it comes to the randomness of variable resets, we allow that a system variable can be reset to a value obtained according to a known discrete or continuous distribution, instead of being assigned with a fixed value. For example, with this extension, on a discrete update, variable $t$ can be assigned to any value between $1$ and $2$ with equal probability. 

\vspace{-.2cm}
% briefly introduce SReach algorithms
In this paper, we describe our tool {\it SReach} which supports probabilistic bounded reachability analysis for these two interesting model classes: GHAs with parametric uncertainty and PHAs with additional randomness. It combines the recently proposed $\delta$-complete bounded reachability analysis technique \cite{gaodelta} with statistical testing techniques. Our technique saves the virtues of the Satisfiability Modulo Theories (SMT) based Bounded Model Checking (BMC) for GHAs \cite{cordeiro2012smt, tinelli2012smt}, namely the fully symbolic treatment of hybrid state spaces, while advancing the reasoning power to probabilistic models and requirements. By utilizing the $\delta$-complete analysis method, the full nondeterminism of models can be considered. By adapting statistical tests, {\it SReach} can place arbitrarily small error bounds on the estimated probabilities. Compared to standard simulation-based approaches, our approach supports nondeterministic branching, increases the coverage of simulation, and avoids the zero-crossing problem which is critical for simulation-based methods. Comparing to the existing tools introduced in \cite{zhang2012safety, franzle2008stochastic, david2012statistical, website:prism}, besides offering a sound way to analyze nonlinear dynamics within the SHSs, {\it SReach} also supports probabilistic bounded reachability analysis for hybrid systems with parametric uncertainty. With this modeling formalism, important elements such as probabilistic initial conditions and random variable coefficients can all be expressed by multiple random variables. Furthermore, for PHAs, {\it SReach} considers a more general and useful formalism where general randomness for transition probabilities and variable resets are allowed. We discuss three biological models - a cardiac atrial fibrillation model, a prostate cancer treatment model, and our synthesized Killerred biological model - to show how {\it SReach} can be used to answer several types of questions including model validation, parameter estimation, and sensitivity analysis. To further demonstrate the feasibility of {\it SReach}, we also apply it to additional real-world hybrid systems with parametric uncertainty, e.g. the quadcopter stabilization control.


\vspace{-.2cm}
{\it {\bf Related Work.}} Analysis approaches for GSHSs are often based on Monte-Carlo simulation \cite{blom2004particle}. Considering the difficulty in dealing with this general case, efforts have been mainly placed on different subclasses. For PHAs, Zhang et al. \cite{zhang2012safety} abstracted the original PHA to a probabilistic automaton (PA), and then used established Model Checking methods (e.g. PRISM \cite{website:prism}) for the abstracted model. Hahn et al. also discussed an abstraction-based method where the given PHA was translated into a $n$-player stochastic game using two different abstraction techniques \cite{hahn2011game}. Another method proposed is an SMT-based BMC procedure \cite{franzle2008stochastic}. In \cite{amin2006reachability, abate2007probabilistic, abate2011two, abate2011quantitative}, a similar class of models called discrete-time stochastic hybrid systems (DTSHSs), which is widely used in control theory, was considered. With regard to system analysis, the control problem is to find an optimal control strategy that minimizes the probability of reaching unsafe states. Zuliani et al. also mentioned a simulation-based method for model checking DTSHSs against bounded temporal properties \cite{zuliani2010bayesian}. We refer to this method as Statistical Model Checking (StatMC).  Although StatMC does not belong to the class of exhaustive state-space exploration methods, it usually returns results faster than the exhaustive search with a predefined arbitrarily small error bound on the estimated probability. StatMC was recently integrated into UPPAAL \cite{larsen1997uppaal} in order to handle very general networks of SHAs \cite{david2012statistical}. To analyze reachability problems of SHAs, Fr\"{a}nzle et al. \cite{franzle2011measurability} first over-approximated a given SHA by a PHA, and then exploited the verification procedure introduced in \cite{zhang2012safety} to model check the over-approximating PHA. Plazter introduced another interesting modeling formalism - stochastic hybrid programs (SHPs) in \cite{platzer2011stochastic}. To specify system properties, Platzer proposed a logic called stochastic differential dynamic logic, and then suggested a proof calculus to verify logical properties of SHPs. 
% This formalism is quite expressive on randomness: it takes stochastic differential equations, discrete probabilistic branching, and random assignments to real-valued variables into account. 


\vspace{-.2cm}
The paper proceeds by first, in Section 2, introducing two modeling formalisms of SHSs under consideration: GHAs with parametric uncertainty, and PHAs with additional randomness. Section 3 explains how {\it SReach} solves the probabilistic bounded reachability problem by encoding stochastic dynamics and combining SMT-based BMC with statistical tests. Case studies and additional experiments are discussed in Section 4. Section 5 concludes the paper.
