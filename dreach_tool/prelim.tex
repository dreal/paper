\section{Preliminaries}
\subsection{Hybrid systems}
Let $H = \langle X, Q, \mathsf{flow}, \mathsf{jump},
\mathsf{inv},\mathsf{init}\rangle$ be a hybrid system, where
$\mathsf{flow}$, $\mathsf{jump}$, $\mathsf{inv}$, $\mathsf{init}$ are
first-order formulas over the reals that allow polynomials,
trigonometric functions, exponential functions, logarithmetic
functions, and Lipschitz-continuous ODEs.

Now specify a numerical error bound $\delta$, and recall that for any
formula $\varphi$ we have defined a notion of $\delta$-perturbation of
$\varphi$, written as $\varphi^{\delta}$. We can then define the
$\delta$-perturbation of $H$ as:
\[
H^{\delta} = \langle X, Q, {\mathsf{flow}}^{\delta},
{\mathsf{jump}}^{\delta}, {\mathsf{inv}}^{\delta},
{\mathsf{init}}^{\delta}\rangle,
\]
by simply relaxing the logic formulas in the representation of $H$.
Choose $n\in\mathbb{N}$ to be a bound on the number of discrete mode
changes and $T\in \mathbb{R}^+$ an upper bound on the time duration.
Let $\mathsf{unsafe}$ encode a subset of $X\times Q$, the state space
of $H$. The bounded $\delta$-reachability problem asks for one of the
following answers:

\begin{itemize}
\item  safe: $H$ cannot reach $\mathsf{unsafe}$ in $n$ steps within
  time $T$.
\item $\delta$-unsafe: $H^{\delta}$ can reach ${\mathsf{unsafe}}^{\delta}$ in $n$ steps within time $T$.
\end{itemize}

%%% Local Variables:
%%% mode: latex
%%% TeX-master: "main"
%%% End:
