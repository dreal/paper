\section{Solving Techniques}

We describe the underlying techniques of \dReach{}.

\subsection{Bounded $\delta$-reachability}\label{sec:delta-reachability}
Let $H = \langle X, Q, \flow, \jump, \inv, \init\rangle$ be a hybrid
system as standardly defined. We use first-order formulae over the
real numbers to represent $H$, by writing $$H = \langle X, Q,
\varphi_{\flow}, \varphi_{\jump}, \varphi_{\inv},
\varphi_{\init}\rangle$$ where $\varphi_{\flow}, \varphi_{\jump},
\varphi_{\inv}$ and $\varphi_{\init}$ are logic formulae that define
the corresponding predicates in the standard definition. Now, let
$\delta\in \mathbb{Q}^+$ be a chosen error bound, we define the
$\delta$-perturbation of $H$ to be $$H^{\delta} = \langle X, Q,
\varphi_{\flow}^{\delta}, \varphi_{\jump}^{\delta},
\varphi_{\inv}^{\delta}, \varphi_{\init}^{\delta}\rangle.$$ Here,
$\varphi^{\delta}$ is a syntactic variant of $\varphi$ which relaxes
the numerical terms in $\varphi$ up to an error bound $\delta$. The
notion is formally defined in our recent
work~\cite{DBLP:conf/lics/GaoAC12,DBLP:conf/cade/GaoAC12,DBLP:journals/corr/GaoKCC14}.
We now define the bounded $\delta$-reachability problem that \dReach{}
solves.

%\paragraph{Bounded $\delta$-Reachability.}
Let $n\in \mathbb{N}$ be a bound and $T\in \mathbb{R}^+$ be an upper
bound of time duration. We write $\unsafe$ to denote a subset of the
state space of $H$ defined by a first-order formula. The bounded
$\delta$-reachability problem asks for one of the following answers
\begin{itemize}
 \item $H$ cannot reach $\unsafe$ in $n$ steps within time $T$.
 \item $H^{\delta}$ can reach $\unsafe^{\delta}$ in $n$ steps within time $T$.
\end{itemize}
Note that these answers are not weaker than the precise ones. When
{\sf safe} is the answer, we know for certain that $H$ does not reach
the unsafe region; when {\sf $\delta$-unsafe} is the answer, there
exists some $\delta$-bounded perturbation in the system that {\em
  would} render it unsafe. Note that the error-bound $\delta$ can be
chosen to be arbitrarily small, so that the {\sf$\delta$-unsafe}
answer discovers robustness problem in the system, which should be
regarded as unsafe indeed.

\subsection{Encoding Reachability}

In our work~\cite{DBLP:journals/corr/GaoKCC14}, we already studied
logic encodings of bounded reachability problems of hybrid
systems. The encoding scheme is based on the standard bounded model
checking while non-trivial mode invariants and systems with
nondeterministic flows make the problem interesting.

In this section, we explain the extensions and variable naming
convention that we make to the standard SMT-LIB~\cite{BarST-SMT-10} to
represent flows and mode invariants of hybrid systems.

\paragraph{Variable naming convention}
In our encoding, a system variable $\texttt{x\_i\_p}$ has two
subscripts $i \in \mathbb{N}$ and $p \in \{0, t\}$. The first
subscript $i$ indicates that it represents the value of a system
variable $x$ at the $i$-th step. The second subscript $p \in \{0, t\}$
denotes the value at the begining of mode (end of mode,
respectively). For instance, $x\_0\_t$ denotes the value of variable
$x$ at the end of first mode (step 0).

\paragraph{define-ode and integral}
A flow consists of a group of ordinary differential equations. In
\drh{}, we provide a command \texttt{define-ode} to associate a flow
with a name. For instance, we use $\texttt{define-ode}$ as follows to
assign a name $\mathrm{flow_1}$ to a group of ODE,
$\frac{\mathrm{d}x}{\mathrm{d}t} = v$ and
$\frac{\mathrm{d}v}{\mathrm{d}t} = -9.8$.
\[
\texttt{(define-ode flow\_1 ((= d/dt[x] v) (= d/dt[v] -9.8)))}
\]
To encode reachability properties of hybrid systems, we need a way to
formulate constraints between the initial variables $\vec{x}^0$ and the
final state variables $\vec{x}^t$ at time $t$. Let $\vec{g_i}$ be the
right-hand sides of the ODEs of $flow_i$. Then Using the
Picard-Lindel$\ddot{o}$f representation, we can formulate the
constraint as
\begin{align*}
[x_1^t, x_2^t, \dots, x_n^t] &= \int_0^t \vec{g}_i([x_1(s), x_2(s), \cdots,
x_n(s)]) \mathrm{d}s.
\end{align*}
In \drh{}, we introduce a new keyword \texttt{integral} and encode the
above constraint as
\[
\texttt{(integral 0 t [x\_0\_0 x\_0\_1 ... x\_0\_n] flow\_i)}.
\]
Note that we do not include a variable in $\vec{x}^t$ such as
\texttt{x\_t\_n} because it can be inferred from the given time
\texttt{t} and $\vec{x}^0$ variables (such as \texttt{x\_0\_n}).
\paragraph{forall\_t} To encode mode invariants in hybrid systems, we
need $\exists\forall^t$-formulas~\cite{DBLP:conf/fmcad/GaoKC13} which
is a restricted form of $\exists\forall$ formula where the universal
quantifications are limited to the time variables. In \drh{}, we
introduce a new keyword $\texttt{forall\_t}$ to encode
$\exists\forall^t$ formulas. Given a time bound $[0, time_i]$, mode
invariant f at mode $n$ is encoded into \texttt{(forall\_t n [0
  time\_i] f)}.

% \subsection{Solving the SMT Formulas}

% TODO: ?

%%% Local Variables:
%%% mode: latex
%%% TeX-master: "main"
%%% End:
