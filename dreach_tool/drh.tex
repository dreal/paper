\section{Input Format}

We now describe the input language for describing hybrid systems and
specifying reachability properties. It consists of five sections
- macro definitions, variable declarations, mode definitions, and
initial condition, and goals.
\begin{align*}
  \textit{drh} := \ & \textit{macro-definition}^*\\
                    & \textit{variable-declaration}^+\\
                    & \textit{mode-definition}^+\\
                    & \textit{initial-condition}\\
                    & \textit{goal}^+
\end{align*}
In macro definitions, it allows users to define macros in C
preprocessor (\texttt{cpp}) style which can be used in the following
sections. Note that macro expansions occur before the other parts are processed.

A variable declaration has a form:
\[
\textit{variable-declaration} \ := \ \texttt{[}
                                     \textit{l}
                                     \texttt{,}
                                     \ \textit{u}
                                     \texttt{]}
                                     \ \textit{var}
                                     \texttt{;}
\]
and it declares a real variable, $var$ and its domain $[l, u]$ which
is in Real interval $\mathbb{IR}$. It requires a special variable
declaration for \textit{time}, to specify the upperbound of time
duration in the analysis of bounded $\delta$-reachability.

A mode definition consists of mode id, mode invariant, flow, and jump.
\begin{align*}
  \textit{mode-definition} \ := & \ \texttt{\{}
                                    \texttt{mode} \ \textit{id}\texttt{;}\\
                           & \ \ \  \texttt{invt}:(\textit{formula} \texttt{;})^+\\
                           & \ \ \  \texttt{flow}:\textit{ode}^+\\
                           & \ \ \ \texttt{jump}:\textit{jump}^+ \texttt{\}}
\end{align*}
\textit{id} is a unique positive interger assigned to a mode. An
invariant is a conjuction of logic formulae which must always hold in
a mode. A flow describes a continuous dynamics of a mode by providing
a set of ordinary differential equations (\textit{ode}s) which is a
form of
``\texttt{d/dt[}\textit{x}\texttt{]=}\textit{exp}''. \textit{jump} is
a form of ``\textit{guard} \texttt{==>} \texttt{@}\textit{n}
\textit{reset}'' where \textit{guard} is a logic formula specifying a
condition to make a transition, $n$ denotes the target mode-id, and
\textit{reset} is a logic formula connecting the old and new values
for the transition.

\texttt{initial-condition} is of a form
``\texttt{@}\textit{mode-id} \textit{formula}\texttt{;}''
where \textit{mode-id} is an initial mode of a hybrid system and
\textit{formula} specifies the initial configuration of it.

\texttt{goal} shares the same syntactic structure,
``\texttt{@}\textit{mode-id} \textit{formula}\texttt{;}'' of
\textit{initial-condition} with a different interpretation. It poses a
reachability question: ``Is there a trajectory of a hybrid system
reaching \textit{mode-id} while satisfying the goal condition \textit{formula}?''.

\begin{figure}
  \centering
  \begin{Verbatim}[fontfamily=courier, frame=single, framesep=1mm,
  numbers=left, fontsize=\scriptsize]
#define D 0.45
#define K 0.9
[0, 15] x;
[9.8] g;
[-18, 18] v;
[0, 3] time;
{   mode 1;
    invt:
        (v <= 0);
        (x >= 0);
    flow:
        d/dt[x] = v;
        d/dt[v] = -g - (D * v ^ 2);
    jump:
        (x = 0) ==> @2 (and (x' = x) (v' = - K * v)); }
 {  mode 2;
    invt:
        (v >= 0);
        (x >= 0);
    flow:
        d/dt[x] = v;
        d/dt[v] = -g + (D * v ^ 2);
    jump:
        (v = 0) ==> @1 (and (x' = x) (v' = v)); }
init:
@1    (and (x >= 5) (v = 0));

goal:
@1    (and (x >= 0.45));
\end{Verbatim}
\caption{An example of \drh{} format: Inelastic bouncing ball with air
  resistance. At lines 1 and 2, we define a drag coefficient $D = 0.45$
  and an elastic coefficient $K = 0.9$ using \texttt{\#define} macros.
  At lines 3 - 6, we declare variables $x, g, v,$ and $time$. At lines
  7 - 15 and 16 - 24, we define two modes -- the falling and the
  bouncing-back modes respectively. At lines 25 and 26, we specify
  that this hybrid system starts at mode 1 (\texttt{@1}) with initial
  condition satisfying $x \ge 5 \land v = 0$. At lines 28 and 29, it
  is asking whether we can have a trajectory ending at mode 1
  (\texttt{@1}) while the height of the ball is higher than $0.45$.}
\label{fig:bouncing-ball-drh}
\end{figure}

Figure~\ref{fig:bouncing-ball-drh} shows a canonical example of hybrid
systems, an inelastic bouncing ball with air resistance, in \drh{}
format.

%%% Local Variables:
%%% mode: latex
%%% TeX-master: "main"
%%% End:
