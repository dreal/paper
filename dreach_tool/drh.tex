\section{drh: Language for Modeling and Specifying Hybrid Systems}

We define \texttt{drh}, a language for describing hybrid systems and
specifying their reachability properties. It consists of five sections
- macro definitions, variable declarations, mode definitions, and
initial condition, and goals.
\begin{align*}
  \textit{drh} := \ & \textit{macro-definition}^*\\
                    & \textit{variable-declaration}^+\\
                    & \textit{mode-definition}^+\\
                    & \textit{initial-condition}\\
                    & \textit{goal}^+
\end{align*}
In macro definitions, it allows users to define macros in C
preprocessor (\texttt{cpp}) style which can be used in the following
sections. Note that macro expansions occur before the other parts are processed.

A variable declaration has a form:
\[
\textit{variable-declaration} \ := \ \texttt{[}
                                     \textit{l}
                                     \texttt{,}
                                     \ \textit{u}
                                     \texttt{]}
                                     \ \textit{var}
                                     \texttt{;}
\]
and it declares a real variable, $var$ and its domain $[l, u]$ which
is in Real interval $\mathbb{IR}$. It requires a special variable
declaration for \textit{time}, to specify the upperbound of time
duration in the analysis of bounded $\delta$-reachability.

A mode definition consists of mode id, mode invariant, flow, and jump.
\begin{align*}
  \textit{mode-definition} \ := & \ \texttt{\{}
                                    \texttt{mode} \ \textit{id}\texttt{;}\\
                           & \ \ \  \texttt{invt}:(\textit{formula} \texttt{;})^+\\
                           & \ \ \  \texttt{flow}:\textit{ode}^+\\
                           & \ \ \ \texttt{jump}:\textit{jump}^+ \texttt{\}}
\end{align*}
\textit{id} is a unique positive interger assigned to a mode. An
invariant is a conjuction of logic formulae which must always hold in
a mode. A flow describes a continuous dynamics of a mode by providing
a set of ordinary differential equations (\textit{ode}s) which is a
form of
``\texttt{d/dt[}\textit{x}\texttt{]=}\textit{exp}''. \textit{jump} is
a form of ``\textit{guard} \texttt{==>} \texttt{@}\textit{n}
\textit{reset}'' where \textit{guard} is a logic formula specifying a
condition to make a transition, $n$ denotes the target mode-id, and
\textit{reset} is a logic formula connecting the old and new values
for the transition.

\texttt{initial-condition} is of a form
``\texttt{@}\textit{mode-id} \textit{formula}\texttt{;}''
where \textit{mode-id} is an initial mode of a hybrid system and
\textit{formula} specifies the initial configuration of it.

\texttt{goal} shares the same syntactic structure,
``\texttt{@}\textit{mode-id} \textit{formula}\texttt{;}'' of
\textit{initial-condition} with a different interpretation. It poses a
reachability question: ``Is there a trajectory of a hybrid system
reaching \textit{mode-id} while satisfying the goal condition \textit{formula}?''.

%%% Local Variables:
%%% mode: latex
%%% TeX-master: "main"
%%% End:
