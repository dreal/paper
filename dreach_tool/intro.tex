\section{Introduction}

% Need a paragrapgh or two to explain why the tool is interesting and
% significant should be provided.
\dReach{} is a bounded model checker for hybrid systems. It encodes bounded reachability problems of hybrid systems as first-order formulas over the real numbers, and solves them using $\delta$-decision procedures in the SMT solver \dReal{}. \dReach{} is able to handle a wide range of highly nonlinear hybrid systems. It has scaled well on various realistic nonlinear models from biomedical and robotics applications~\cite{}. 

It is well-known that the standard bounded reachability problems for simple hybrid systems are already highly undecidable~\cite{DBLP:conf/rex/AlurD91,DBLP:conf/hybrid/AlurCHH92}. In previous work~\cite{}, we have defined the notion of $\delta$-reachability problem of hybrid systems. In this new framework, we have shown that bounded $\delta$-reachability is decidable for a wide range of hybrid systems, with reasonable complexity bounds~\cite{}. We give a brief review of the framework in Section~\ref{}. 

\paragraph{Related Work} 
%reachable set computation tools: flow star, SpaceX, Phaver, 
%theorem provers:
%similar tools: iSAT, RSolver -- emphasize on the nonlinearity that we can handle. 

The paper is structured as follows. 

%put the following two paragraphs in a \section{Bounded delta-reachability} 
Let $H = \langle X, Q, \flow, \jump, \inv, \init\rangle$ be a hybrid
system as standardly defined. We use first-order formulae over the
real numbers to represent $H$, by writing $$H = \langle X, Q,
\varphi_{\flow}, \varphi_{\jump}, \varphi_{\inv},
\varphi_{\init}\rangle$$ where $\varphi_{\flow}, \varphi_{\jump},
\varphi_{\inv}$ and $\varphi_{\init}$ are logic formulae that define
the corresponding predicates in the standard definition. Now, let
$\delta\in \mathbb{Q}^+$ be a chosen error bound, we define the
$\delta$-perturbation of $H$ to be $$H^{\delta} = \langle X, Q,
\varphi_{\flow}^{\delta}, \varphi_{\jump}^{\delta},
\varphi_{\inv}^{\delta}, \varphi_{\init}^{\delta}\rangle.$$ Here,
$\varphi^{\delta}$ is a syntactic variant of $\varphi$ which relaxes
the numerical terms in $\varphi$ up to an error bound $\delta$. The
notion is formally defined in our recent
work~\cite{DBLP:conf/lics/GaoAC12,DBLP:conf/cade/GaoAC12,DBLP:journals/corr/GaoKCC14}.
We now define the bounded $\delta$-reachability problem that
\dReach{} solves.

\paragraph{Bounded $\delta$-Reachability.} Let $n\in \mathbb{N}$ be a bound and
$T\in \mathbb{R}^+$ be an upper bound of time duration. We write $\unsafe$ to denote a subset of
the state space of $H$ defined by a first-order formula. The bounded
$\delta$-reachability problem asks for one of the following answers
\begin{itemize}
 \item $H$ cannot reach $\unsafe$ in $n$ steps within time $T$.
 \item $H^{\delta}$ can reach $\unsafe^{\delta}$ in $n$ steps within time $T$.
\end{itemize}
Note that these answers are not weaker than the precise ones. When {\sf safe} is the answer, we know for certain
that $H$ does not reach the unsafe region; when {\sf $\delta$-unsafe} is the
answer, there exists some $\delta$-bounded perturbation in the
system that {\em would} render it unsafe. Note that the error-bound $\delta$ can
be chosen to be arbitrarily small, so that the {\sf$\delta$-unsafe} answer
discovers robustness problem in the system, which should be regarded as unsafe
indeed.


%%% Local Variables:
%%% mode: latex
%%% TeX-master: "main"
%%% End:
