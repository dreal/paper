\section{Introduction}\label{sec:intro}

% Need a paragrapgh or two to explain why the tool is interesting and
% significant should be provided.

\dReach{} is a bounded model checker for hybrid systems. It encodes
bounded reachability problems of hybrid systems as first-order
formulas over the real numbers, and solves them using
$\delta$-decision procedures in the SMT solver \dReal{}. \dReach{} is
able to handle a wide range of highly nonlinear hybrid systems. It has
scaled well on various realistic nonlinear models from biomedical and
robotics applications~\cite{}. For instance, Figure~\ref{fig:prostate-example} 
shows the nonlinear dynamics of a prostate cancer treatment model (the full hybrid system is 
in the Appendix) and a visualization of a concrete counterexample generated by \dReach{}. 
\begin{figure}[!h]
  \subfloat[An example of nonlinear hybrid system model: off-treatment
  mode of the prostate cancer treatement model~\cite{}\label{subfig-1:prostate}]{
    \includegraphics[width=0.48\textwidth]{images/prostatebw-mode2.pdf}
  }
  \hfill
  \subfloat[Visualization of a concrete counterexample generated from
  dReach for the prostate cancer treatment model.]{%
    \includegraphics[width=0.48\textwidth]{images/prostate}
  }
  \caption{An example of the nonlinear dynamics that can be handled by ~\dReach{}, and its counterexample-generation capacity.}
  \label{fig:prostate-example}
\end{figure}

It is well-known that the standard bounded reachability problems for
simple hybrid systems are already highly
undecidable~\cite{DBLP:conf/rex/AlurD91,DBLP:conf/hybrid/AlurCHH92}. However, in
previous work~\cite{}, we have defined the notion of
$\delta$-reachability problem of hybrid systems. In this new
framework, we have shown that bounded $\delta$-reachability is
decidable for a wide range of hybrid systems, with reasonable
complexity bounds~\cite{}. We give a brief review of the framework in
Section~\ref{sec:delta-reachability}.

The following key features of ~\dReach{} separate it from other existing tools. 
%insert explanations for each item. 
\begin{itemize}
\item Any nonlinear dynamics. Any logic formulas in the signature for describing jumps, conditions, etc. 
\item Faithful encoding of invariants as $\exists\forall$ problems. 
\item No explicit storage of reachable states but perform property guided search for concrete trajectories.
It avoids the state explosion in representing the full reachable set. This is analogous to the difference between SAT-based 
model checking and symbolic model checking.  
\item It provides a general framework that can make full use of exisiting reachable set computation algorithms and logic solving.  
\end{itemize}
%Realistic hybrid systems involves nonlinear ODEs with transcendental
%functions. \dReach{} allows users to specify a hybrid system in a
%nonlinear signature as it is without linearizing or overapproximating
%it. Users can provide the tool with a numerical error bound $\delta$,
%a bounded time horizon $[0, T]$, and a maximum number of mode switches
%$k$ for the analysis. As a result of analysis, \dReach{} will return
%either \textbf{$\delta$-sat} with a concrete counterexample, or
%\textbf{unsat} which does not involve numerical errors. We also
%provide a visualization for the $\delta$-sat case to help
%understand the analysis result.

\paragraph{Related Work}
%reachable set computation tools: flow star, SpaceX, Phaver,
%theorem provers:
%similar tools: iSAT, RSolver -- emphasize on the nonlinearity that we can handle.

The paper is structured as follows.


%%% Local Variables:
%%% mode: latex
%%% TeX-master: "main"
%%% End:
