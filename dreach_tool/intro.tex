\section{Introduction}

% Need a paragrapgh or two to explain why the tool is interesting and
% significant should be provided.

Hybrid system provides a way to model complex dynamical systems which
exhibit both of discrete and continuous behaviors. This expresive
power allows many areas to adopt it to formulate models. The examples
include flight controllers~\cite{?}, medical equipments~\cite{?}, and
biological systems~\cite{?}.

However, this expresive power limits the possibility of analysis. In
the seminal work of
\cite{DBLP:conf/rex/AlurD91,DBLP:conf/hybrid/AlurCHH92}, it is already
shown that even for simple classes of hybrid automata the problem of
bounded reachability is undecidable. This decidability result has led
verification tools either to accept a restricted form of
input~\cite{?} or to over-approximate the set of reachable states of
input models~\cite{?}.

We present our tool \dReach{} for solving hybrid system
reachability problems. Our tool performs bounded model checking for a
wide range of hybrid systems whose descriptions can involve various
nonlinear elementrary functions (such as $\exp$ and $\sin$) and
ODEs. We avoid the undecidability of the problem in the standard
definition by switching to what we call the ``bounded
$\delta$-reachability'' which we now briefly explain.

Let $H = \langle X, Q, \flow, \jump, \inv, \init\rangle$ be a hybrid
system as standardly defined. We use first-order formulae over the
real numbers to represent $H$, by writing $$H = \langle X, Q,
\varphi_{\flow}, \varphi_{\jump}, \varphi_{\inv},
\varphi_{\init}\rangle$$ where $\varphi_{\flow}, \varphi_{\jump},
\varphi_{\inv}$ and $\varphi_{\init}$ are logic formulae that define
the corresponding predicates in the standard definition. Now, let
$\delta\in \mathbb{Q}^+$ be a chosen error bound, we define the
$\delta$-perturbation of $H$ to be $$H^{\delta} = \langle X, Q,
\varphi_{\flow}^{\delta}, \varphi_{\jump}^{\delta},
\varphi_{\inv}^{\delta}, \varphi_{\init}^{\delta}\rangle.$$ Here,
$\varphi^{\delta}$ is a syntactic variant of $\varphi$ which relaxes
the numerical terms in $\varphi$ up to an error bound $\delta$. The
notion is formally defined in our recent
work~\cite{DBLP:conf/lics/GaoAC12,DBLP:conf/cade/GaoAC12,DBLP:journals/corr/GaoKCC14}.
We now define the bounded $\delta$-reachability problem that
\dReach{} solves.

\paragraph{Bounded $\delta$-Reachability.} Let $n\in \mathbb{N}$ be a bound and
$T\in \mathbb{R}^+$ be an upper bound of time duration. We write $\unsafe$ to denote a subset of
the state space of $H$ defined by a first-order formula. The bounded
$\delta$-reachability problem asks for one of the following answers
\begin{itemize}
 \item $H$ cannot reach $\unsafe$ in $n$ steps within time $T$.
 \item $H^{\delta}$ can reach $\unsafe^{\delta}$ in $n$ steps within time $T$.
\end{itemize}
Note that these answers are not weaker than the precise ones. When {\sf safe} is the answer, we know for certain
that $H$ does not reach the unsafe region; when {\sf $\delta$-unsafe} is the
answer, there exists some $\delta$-bounded perturbation in the
system that {\em would} render it unsafe. Note that the error-bound $\delta$ can
be chosen to be arbitrarily small, so that the {\sf$\delta$-unsafe} answer
discovers robustness problem in the system, which should be regarded as unsafe
indeed.


%%% Local Variables:
%%% mode: latex
%%% TeX-master: "main"
%%% End:
