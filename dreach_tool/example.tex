\section{Example: Inelastic Bouncing Ball with Drag}
\begin{wrapfigure}{l}{0.5\textwidth}
  \centering
  \includegraphics[width=0.5 \textwidth]{images/bouncing_ball.pdf}
  \caption{Bouncing ball example}
  \label{fig:bouncing-ball}
\end{wrapfigure}


\begin{figure}[!h]
  \centering
  \begin{Verbatim}[fontfamily=courier, frame=single, framesep=1mm,
  numbers=left, fontsize=\scriptsize]
#define D 0.45
#define K 0.9
[0, 15] x;
[9.8] g;
[-18, 18] v;
[0, 3] time;
{
  mode 1;
  invt:
        (v <= 0);
        (x >= 0);
  flow:
        d/dt[x] = v;
        d/dt[v] = -g + (- D * v ^ 1);
  jump:
        (x = 0) ==> @2 (and (x' = x) (v' = - K * v));
}
{
  mode 2;
  invt:
        (v >= 0);
        (x >= 0);
  flow:
        d/dt[x] = v;
        d/dt[v] = -g + (- D * v ^ 1);
  jump:
        (v = 0) ==> @1 (and (x' = x) (v' = v));
}
init:
@1    (and (x >= 5) (v = 0));

goal:
@1    (and (x >= 0.45));
\end{Verbatim}
  \caption{drh format of bouncing ball example}
  \label{fig:bouncing-ball-drh}
\end{figure}

\begin{figure}[!h]
  \centering
  \begin{Verbatim}[fontfamily=courier, frame=single, framesep=1mm,
  numbers=left, fontsize=\scriptsize]
(set-logic QF_NRA_ODE)
(declare-fun x () Real)
(declare-fun v () Real)
(declare-fun x_0_0 () Real)
(declare-fun x_0_t () Real)
...
(declare-fun x_10_0 () Real)
(declare-fun x_10_t () Real)
(declare-fun v_0_0 () Real)
(declare-fun v_0_t () Real)
...
(declare-fun v_10_0 () Real)
(declare-fun v_10_t () Real)
(declare-fun time_0 () Real)
...
(declare-fun time_10 () Real)
(declare-fun mode_0 () Real)
...
(declare-fun mode_10 () Real)
(define-ode flow_1 ((= d/dt[x] v)
                    (= d/dt[v] (+ (- 0.0 9.8) (* -0.45 (^ v 1.0))))))
(define-ode flow_2 ((= d/dt[x] v)
                    (= d/dt[v] (+ (- 0.0 9.8) (* -0.45 (^ v 1.0))))))
(assert (<= 0.0 x_0_0))
(assert (<= x_0_0 15.0))
...
(assert (<= -18.0 v_10_t))
(assert (<= v_10_t 18.0))
(assert (<= 0.0 time_0))
(assert (<= time_0 3.0))
...
(assert (<= 0.0 time_10))
(assert (<= time_10 3.0))
...

(assert (and (and (= v_0_0 0.0) (>= x_0_0 5.0)) (= mode_0 1.0) (=
[x_0_t v_0_t] (integral 0. time_0 [x_0_0 v_0_0] flow_1)) (= mode_0
1.0) (forall_t 1.0 [0.0 time_0] (<= v_0_t 0.0)) (<= v_0_t 0.0) (<=
...
x_9_t) (= [x_10_t v_10_t] (integral 0. time_10 [x_10_0 v_10_0]
flow_1)) (= mode_10 1.0) (forall_t 1.0 [0.0 time_10] (<= v_10_t 0.0))
(<= v_10_t 0.0) (<= v_10_0 0.0) (forall_t 1.0 [0.0 time_10] (>= x_10_t
0.0)) (>= x_10_t 0.0) (>= x_10_0 0.0) (= mode_10 1.0) (>= x_10_t
0.45))) (check-sat) (exit)
\end{Verbatim}
  \caption{SMT2 encoding of the bounded reachability problem of
    bouncing ball ($k = 3$) }
  \label{fig:bouncing-ball-smt2}
\end{figure}

%%% Local Variables:
%%% mode: latex
%%% TeX-master: "main"
%%% End:
