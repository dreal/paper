\section*{A2. Experimental Results}\label{sec:exp}

All benchmarks and data shown here are also available on the tool
website.All experiments were conducted on a machine with a 3.4GHz
octa-core Intel Core i7-2600 processor and 16GB RAM, running 64-bit
Ubuntu 12.04LTS. Table~\ref{tbl:exp} is a summary of the running time
of the tool on various hybrid system models.

\paragraph{Atrial Fibrillation.} We studied the Atrial Fibrillation
model as developed in~\cite{DBLP:conf/cav/GrosuBFGGSB11}.
The model
has four discrete control locations, four state variables, and
nonlinear ODEs. A typical set of ODEs in the model is:
\begin{eqnarray*}
\frac{du}{dt} &=& e + (u-\theta_v)(u_u-u ) v g_{fi} + wsg_{si}-g_{so}(u)\\
\frac{ds}{dt} &=& \displaystyle\frac{g_{s2}}{(1+\exp(-2k(u-us)))} -  g_{s2}s\\
\frac{dv}{dt} &=& -g_v^+\cdot v \hspace{1cm} \frac{dw}{dt} = -g_w^+\cdot w
\end{eqnarray*}
The exponential term on the right-hand side of the ODE is the sigmoid function, which often appears in modelling biological switches.
\paragraph{Prostate Cancer Treatment.} The Prostate Cancer Treatment
model~\cite{CMSB14} exhibits more nonlinear ODEs. The reachability
questions are
\begin{eqnarray*}
\frac{dx}{dt} &=& (\alpha_x
(k_1+(1-k_1)\frac{z}{z+k_2}-\beta_x( (1-k_3)\frac{z}{z+k_4}+k_3)) - m_1(1-\frac{z}{z_0}))x + c_1 x\\
\frac{dy}{dt} &=& m_1(1-\frac{z}{z_0})x+(\alpha_y (1- d\frac{z}{z_0}) - \beta_y)y+c_2y\\
\frac{dz}{dt} &=& \frac{-z}{\tau} + c_3z\\
\frac{dv}{dt} &=& (\alpha_x
(k_1+(1-k_1)\frac{z}{z+k_2}-\beta_x(k_3+(1-k_3)\frac{z}{z+k_4}))\\
& &- m_1(1-\frac{z}{z_0}))x + c_1 x + m_1(1-\frac{z}{z_0})x+(\alpha_y (1- d\frac{z}{z_0}) - \beta_y)y+c_2y
\end{eqnarray*}
\paragraph{Electronic Oscillator.} The EO model represents an electronic oscillator model that contains nonlinear ODEs such as the following:
\begin{eqnarray*}
\frac{dx}{dt} &=& - ax \cdot sin(\omega_1 \cdot \tau)\\
\frac{dy}{dt} &=& - ay \cdot sin( (\omega_1 + c_1) \cdot \tau) \cdot sin(\omega_2)\cdot 2\\
\frac{dz}{dt} &=& - az \cdot sin( (\omega_2 + c_2) \cdot \tau) \cdot cos(\omega_1)\cdot 2\\
\frac{\omega_1}{dt} &=& - c_3\cdot \omega_1\ \ \ \frac{\omega_2}{dt} = -c_4\cdot\omega_2\ \ \ \frac{d\tau}{dt} = 1
\end{eqnarray*}
\paragraph{Quadcopter Control.} We developed a model that contains the full dynamics of a quadcopter. We use the model to solve control problems by answering reachability questions. A typical set of the differential equations are the following:
\begin{eqnarray*}
\frac{\mathrm{d}\omega_x}{\mathrm{d}t} &=& L\cdot k\cdot (\omega_1^2 - \omega_3^2)(1/I_{xx})-(I_{yy} - I_{zz})\omega_y\omega_z/I_{xx}\\
\frac{\mathrm{d}\omega_y}{\mathrm{d}t} &=& L\cdot k\cdot(\omega_2^2 - \omega_4^2)(1/I_{yy})-(I_{zz} - I_{xx})\omega_x\omega_z/I_{yy}\\
\frac{\mathrm{d}\omega_z}{\mathrm{d}t} &=& b\cdot(\omega_1^2 - \omega_2^2 + \omega_3^2 - \omega_4^2)(1/I_{zz})-(I_{xx} - I_{yy})\omega_x\omega_y/I_{zz}\\
\frac{\mathrm{d}\phi}{\mathrm{d}t} &=& \omega_x + \displaystyle{\frac{\sin\left(\phi\right) \sin\left(\theta\right)}{{\left(\frac{\sin\left(\phi\right)^{2} \cos\left(\theta\right)}{\cos\left(\phi\right)} + \cos\left(\phi\right) \cos\left(\theta\right)\right)} \cos\left(\phi\right)}}\omega_y + \displaystyle\frac{\sin\left(\theta\right)}{\frac{\sin\left(\phi\right)^{2} \cos\left(\theta\right)}{\cos\left(\phi\right)} + \cos\left(\phi\right) \cos\left(\theta\right)}\omega_z\\
\frac{\mathrm{d}\theta}{\mathrm{d}t} &=& -(\displaystyle\frac{\sin\left(\phi\right)^{2} \cos\left(\theta\right)}{{\left(\frac{\sin\left(\phi\right)^{2} \cos\left(\theta\right)}{\cos\left(\phi\right)}\omega_y + \cos\left(\phi\right) \cos\left(\theta\right)\right)} \cos\left(\phi\right)^{2}} + \frac{1}{\cos\left(\phi\right)})\omega_y\\
& &\hspace{5cm}-\displaystyle\frac{\sin\left(\phi\right) \cos\left(\theta\right)}{{\left(\frac{\sin\left(\phi\right)^{2} \cos\left(\theta\right)}{\cos\left(\phi\right)} + \cos\left(\phi\right) \cos\left(\theta\right)\right)} \cos\left(\phi\right)}\omega_z \\
\frac{\mathrm{d}\psi}{\mathrm{d}t} &=& \displaystyle\frac{\sin\left(\phi\right)}{{\left(\frac{\sin\left(\phi\right)^{2} \cos\left(\theta\right)}{\cos\left(\phi\right)} + \cos\left(\phi\right) \cos\left(\theta\right)\right)} \cos\left(\phi\right)}\omega_y + \displaystyle\frac{1}{\frac{\sin\left(\phi\right)^{2} \cos\left(\theta\right)}{\cos\left(\phi\right)} + \cos\left(\phi\right) \cos\left(\theta\right)}\omega_z\\
\frac{\mathrm{d}{xp}}{\mathrm{d}t} &=& (1/m)(\sin(\theta)\sin(\psi)k(\omega_1^2 + \omega_2^2 +\omega_3^2+\omega_4^2) - k\cdot d\cdot{xp})\\
\frac{\mathrm{d}{yp}}{\mathrm{d}t} &=& (1/m)(-\cos(\psi)\sin(\theta)k(\omega_1^2 + \omega_2^2 +\omega_3^2+\omega_4^2) - k\cdot d\cdot{yp})\\
\frac{\mathrm{d}{zp}}{\mathrm{d}t} &=& (1/m)(-g-\cos(\theta)k(\omega_1^2 + \omega_2^2 +\omega_3^2+\omega_4^2) - k\cdot d\cdot{zp}\\
\frac{\mathrm{d}x}{\mathrm{d}t} &=& {xp}, \frac{\mathrm{d}y}{\mathrm{d}t} = {yp}, \frac{\mathrm{d}z}{\mathrm{d}t} = {zp}
\end{eqnarray*}

\newcommand{\hmodel}[2]{\href{http://dreal.cs.cmu.edu/#1}{#2}}
{\small
\begin{table}[!th]
  \centering
  \small
  \begin{tabular}{l|r|r|r|r|r|r|r|r}
    \hline
    \hline
    Benchmark    & \#Mode& \#Depth & \#ODEs & \#Vars  & Delta  & Result       & Time(s) & Trace \\
    \hline
    \hline
      AF-GOOD & 4     & 3        & 20     & 53      & 0.001     & SAT &  0.425    & 793K     \\
       AF-BAD & 4     & 3        & 20     & 53      & 0.001     & UNSAT &  0.074    & ---      \\
  AF-TO1-GOOD & 4     & 3        & 24     & 62      & 0.001     & SAT &  2.750    & 224K     \\
   AF-TO1-BAD & 4     & 3        & 24     & 62      & 0.001     & UNSAT &  5.189    & ---     \\
  AF-TO2-GOOD & 4     & 3        & 24     & 62      & 0.005     & SAT &  3.876    & 553K     \\
   AF-TO2-BAD & 4     & 3        & 24     & 62      & 0.001     & UNSAT &  8.857    & ---     \\
 AF-TSO1-TSO2 & 4     & 3        & 24     & 62      & 0.001     & UNSAT &  0.027    & ---     \\
       AF8-K7 & 8     & 7        & 40     & 101     & 0.001     & SAT & 10.478   & 3.8M      \\
      AF8-K23 & 8     & 23       & 40     & 293     & 0.001     & SAT & 135.29   & 11M      \\
    \hline
    \hline
    EO-K2  & 3     & 2        & 18     & 48      & 0.01    & SAT & 3.144    & 1.9M      \\
    EO-K11 & 3     & 11       & 99     & 174     & 0.01    & UNSAT & 0.969    & ---       \\
    \hline
    \hline
    QUAD-K1  & 2   & 1          & 34     & 89      & 0.01      & SAT & 2.386 &  10M \\
    QUAD-K2  & 2   & 2          & 34     & 125     & 0.01      & SAT & 4.971 &  13M \\
    QUAD-K3  & 4   & 3          & 68     & 161     & 0.01      & SAT & 13.755 & 42M \\
    QUAD-K3U & 4   & 3          & 68     & 161     & 0.01      & UNSAT & 2.846 & --- \\
    \hline
    \hline
    CT       & 2   & 2         & 10      & 41      & 0.005     & SAT & 345.84 & 3.1M\\
    CT       & 2   & 2         & 10      & 41      & 0.002     & SAT & 362.84 & 3.1M\\
    \hline
    \hline
    BB-K10 & 2     & 10       & 22     & 66      & 0.01        & SAT & 8.057     & 123K  \\
    BB-K20 & 2     & 20       & 42     & 126     & 0.01        & SAT & 39.196    & 171K  \\
    \hline
    \hline
  \end{tabular}
  \caption{\small
    \#Mode = Number of modes in the hybrid system,
    \#Depth = Unrolling depth,
    \#ODEs = Number of ODEs in the unrolled formula,
    \#Vars = Number of variables in the unrolled formula,
    Result = Bounded Model Checking Result (delta-SAT/UNSAT)
    Time = CPU time (s),
    Trace = Size of the ODE trajectory,
    AF = Atrial Filbrillation,
    EO = Electronic Oscillator,
    QUAD = Quadcopter Control,
    CT = Cancer Treatment,
    BB = Bouncing Ball with Drag.
    % TIMES = Solving time in seconds, TO = Timeout (30min), PC = Proof
    % Checked, #PA = Number of proved axioms, #SP = Number of subproblems
    % generated by proof checking, TIMEPC = Proof-checking time in seconds, #D =
    % Number of iteration depth required in proof checking
}\label{tbl:exp}
\end{table}

% \paragraph{Example encoding} The bounded reachability problem of a
% bouncing ball example (when $k = 3$) is encoded into the following
% shortened SMT2 formula.
% \begin{Verbatim}[fontfamily=courier, frame=single, framesep=1mm,  numbers=left, fontsize=\scriptsize]
% (set-logic QF_NRA_ODE)
% (declare-fun x_0_0 () Real) ...
% (declare-fun v_0_t () Real) ...
% (declare-fun time_0 () Real) ...
% (define-ode flow_1 ((= d/dt[x] v)
%                     (= d/dt[v] (+ (- 0.0 9.8) (* -0.45 (^ v 2.0))))))
% (define-ode flow_2 ((= d/dt[x] v)
%                     (= d/dt[v] (+ (- 0.0 9.8) (* +0.45 (^ v 2.0))))))
% (assert (<= 0.0 x_0_0)) ...
% (assert (<= v_10_t 18.0))
% (assert (<= 0.0 time_0))
% (assert (and (and (= v_0_0 0.0) (>= x_0_0 5.0)) (= mode_0 1.0) (=
% [x_0_t v_0_t] (integral 0. time_0 [x_0_0 v_0_0] flow_1)) (= mode_0
% 1.0) (forall_t 1.0 [0.0 time_0] (<= v_0_t 0.0)) (<= v_0_t 0.0) (<=
% ...
% x_9_t) (= [x_10_t v_10_t] (integral 0. time_10 [x_10_0 v_10_0]
% flow_1)) (= mode_10 1.0) (forall_t 1.0 [0.0 time_10] (<= v_10_t 0.0))
% (<= v_10_t 0.0) (<= v_10_0 0.0) (forall_t 1.0 [0.0 time_10] (>= x_10_t
% 0.0)) (>= x_10_t 0.0) (>= x_10_0 0.0) (= mode_10 1.0) (>= x_10_t
% 0.45))) (check-sat) (exit)
% \end{Verbatim}

%%% Local Variables:
%%% mode: latex
%%% TeX-master: "main"
%%% End:
