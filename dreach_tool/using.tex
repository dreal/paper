\section{Using \dReach{}}\label{sec:using-dreach}
% We now describe the input format and command line options of
% \dReach{}.
\subsection{Input Format}\label{sec:input-format}
The input format for describing hybrid systems and reachability properties consists of five
sections: macro definitions, variable declarations, mode definitions,
and initial condition, and goals. We focus on intuitive explanations here.
%, and the formal grammar is given in the Appendix.
Figure~\ref{fig:bouncing-ball-drh} shows how to describe a small
example hybrid system, an inelastic bouncing ball with air resistance.

\begin{itemize}
\item In macro definitions, we allows users to define macros in C
preprocessor style which can be used in the following
sections. Macro expansions occur before the other parts are processed.

\item A variable declaration specifies a real variable and its domain
  in a real interval. \dReach{} requires special declaration for
  \textit{time} variable, to specify the upperbound of time duration.

\item A mode definition consists of mode id, mode invariant, flow, and jump.
\textit{id} is a unique positive interger assigned to a mode. An
invariant is a conjuction of logic formulae which must always hold in
a mode. A flow describes the continuous dynamics of a mode by providing
a set of ODEs. The first
formula of \textit{jump} is interpreted as a guard, a logic formula
specifying a condition to make a transition. Note that this allows a
transition but does not force it. The second argument of
\textit{jump}, $n$ denotes the target mode-id. The last one is
\textit{reset}, a logic formula connecting the old and new values for
the transition.

\item \textit{initial-condition} specifies the initial mode of a hybrid
system and its initial configuration. \textit{goal} shares the same
syntactic structure of \textit{initial-condition}.
\end{itemize}
\begin{figure}
  \centering
  \begin{Verbatim}[fontfamily=courier, frame=single, framesep=1mm, fontsize=\footnotesize]
 1   #define D 0.45
 2   #define K 0.9
 3   [0, 15] x; [9.8] g; [-18, 18] v; [0, 3] time;
 4   {   mode 1;
 5       invt: (v <= 0);  (x >= 0);
 6       flow: d/dt[x] = v; d/dt[v] = -g - (D * v ^ 2);
 7       jump: (x = 0) ==> @2 (and (x' = x) (v' = - K * v)); }
 8   {   mode 2;
 9       invt: (v >= 0); (x >= 0);
10       flow: d/dt[x] = v; d/dt[v] = -g + (D * v ^ 2);
11       jump: (v = 0) ==> @1 (and (x' = x) (v' = v)); }
12   init: @1 (and (x >= 5) (v = 0));
13   goal: @1 (and (x >= 0.45));
\end{Verbatim}
\caption{An example of \drh{} format: Inelastic bouncing ball with air
  resistance. Lines 1 and 2 define a drag coefficient $D = 0.45$ and
  an elastic coefficient $K = 0.9$. Line 3 declares variables
  $x, g, v,$ and $time$. At lines 4 - 7 and 8 - 11, we define two
  modes -- the falling and the bouncing-back modes respectively. At
  line 12, we specify the hybrid system to start at mode 1
  (\texttt{@1}) with initial condition satisfying
  $x \ge 5 \land v = 0$. At line 13, it asks whether we can have
  a trajectory ending at mode 1 (\texttt{@1}) while the height of the
  ball is higher than $0.45$.}
\label{fig:bouncing-ball-drh}
\end{figure}
\vspace{-1.7em}
\subsection{Command Line Options}
\dReach{} follows the standard unix command-line usage:
\begin{Verbatim}[fontfamily=courier, framesep=1mm, fontsize=\small]
dReach <options> <drh file>
\end{Verbatim}
It has the following options:
\begin{itemize}
\item If \texttt{-k <N>} is used, set the unrolling bound $k$ as $N$
  (Default: 3). It also provides \texttt{-u <N>} and \texttt{-l <N>}
  options to specify upper- and lower-bounds of unrolling bound.
\item If \texttt{--precision <p>} is used, use precision $p$ (Default: $0.001$).
\item If \texttt{--visualize} is set, \dReach{} generates extra visualization data.
\end{itemize}
We have a web-based visualization toolkit\footnote{The detailed
  instructions are available at
  \url{https://github.com/dreal/dreal/blob/master/doc/ode-visualization.md}.}
which processes the generated visualization data and shows the
counterexample trajectory. It provides a way to navigate and
zoom-in/out trajectories which helps understand and debug the target
hybrid system better.

%%% Local Variables:
%%% mode: latex
%%% TeX-master: "main"
%%% End:
