
We now formally define hybrid systems, which we use for modelling cyber-physical systems. The following
definition is standard.

\begin{definition}
A hybrid system consists of the following components:
\begin{itemize}
\item $Q = \{q_0, ..., q_m\}$ a set of modes (discrete components of the system),
\item $X = [u_1, v_1] \times ... \times [u_n, v_n] \times [0, T] \subset \mathbb{R}^{n+1}$ a domain 
	of continuous variables,
\item $S = Q \times X$ is the hybrid state space of the system,
\item $U \subseteq S$ an {\em unsafe} region of the state space,
\end{itemize}
and predicates (or relations)
\begin{itemize}
	\item $\text{flow}_{q}(\textbf{x}^{0}, \textbf{x}^{t})$ mapping the continuous state 
	$\textbf{x}^{0}$ at time 0 to state $\textbf{x}^{t}$ at time point $t \in [0, T]$ in mode $q$
	\item $\text{init}_q(\textbf{x})$ indicating that $s = (q, \textbf{x})$ belongs to the set 
	of initial states,
	\item $\text{jump}_{q \rightarrow q'}(\textbf{x}^{t}, \textbf{x}^{0})$ indicating that the 
	system can make a transition from mode $q$, upon reaching the jump condition in continuous 
	state $\textbf{x}^{t}$ at time point $t \in [0, T]$, to mode $q^\prime$ and setting the
	continuous state to $\textbf{x}^{0}$,
	\item $\text{unsafe}_{q}(\textbf{x})$ indicating that $s = (q, \textbf{x}) \in U$
\end{itemize}
\end{definition}
\begin{remark}
The continuous dynamics of the system is defined in each flow, and it can either be presented 
as a system of ODEs or explicitly.
\end{remark}

In this paper we are interested in {\em deterministic} hybrid systems, where in each mode only
one jump is allowed. Informally, the semantics of a hybrid system can be thought as piece-wise
continuous: in each mode the system dynamics follows a continuous flow, while a mode change
(as result of a jump) might also involve a discontinuous change in the variables $\textbf{x}$.
The evolution of a hybrid system is then obtained by iteratively composing the relations flow and
jump, starting with init --- see Definition \ref{def:reachability} below for an example. More details 
can be found in \cite{DBLP:conf/hybrid/AlurCHH92}.

In order to overcome the undecidability of reasoning about hybrid systems, Gao {\em et al.} recently
defined the concept of $\delta$-satisfiability over the reals \cite{DBLP:conf/lics/GaoAC12}, 
and presented a corresponding $\delta$-complete decision procedure \cite{DBLP:conf/cade/GaoAC12}. 
The main idea is to decide correctly whether slightly {\em relaxed} sentences over the reals 
are satisfiable or not. The following definitions are from \cite{DBLP:conf/lics/GaoAC12}.
\begin{definition}
A bounded quantifier is one of the following:
\[ \begin{array}{ll}
	\exists^{[a, b]} x  = \exists x : (a \le x \wedge x \le b) \\[1ex]
	\forall^{[a, b]} x  = \forall x : (a \le x \wedge x \le b)
\end{array} \]
\end{definition}

\begin{definition}
A bounded $\Sigma_1$ sentence is an expression of the form:
\begin{equation*} 
	\exists^{I_{1}} x_{1}, ..., \exists^{I_{1}} x_{n}: \psi(x_{1}, ..., x_{n})
\end{equation*}
where $I_{i} = [a_{i}, b_{i}]$ are intervals, $\psi(x_{1}, ..., x_{n})$ is a Boolean combination 
of atomic formulas of the form $g(x_{1}, ..., x_{n})\, \texttt{op} \, 0$, where $g$ is a composition 
of Type 2-computable functions and $\texttt{op} \in \{<, \le, >, \ge, =, \ne\}$. 
\end{definition}
\begin{remark}
Any bounded $\Sigma_1$ sentence is equivalent to a $\Sigma_1$ sentence in which all the atoms are
of the form $f(x_{1}, ..., x_{n}) = 0$ (\ie, the only \texttt{op} needed is `=') 
\cite{DBLP:conf/cade/GaoAC12}.
\end{remark}
Essentially, Type 2-computable functions can be approximated arbitrarily well by finite
computations of a special kind of Turing machines (Type 2 machines); most `useful' functions over 
the reals (\eg, continuous functions) are Type 2-computable \cite{kobook}.

The notion of $\delta$-weakening \cite{DBLP:conf/lics/GaoAC12} of a bounded sentence is central
to $\delta$-satisfiability.
\begin{definition}
Let $\delta \in \mathbb{Q^+} \cup \{0\}$ be a constant and $\phi$ a bounded $\Sigma_1$-sentence 
in the standard form 
\begin{equation} \label{def:sigma1}
\phi = \exists^{I_{1}} x_{1}, ..., \exists^{I_{n}} x_{n}: \bigwedge^{m}_{i = 1}(\bigvee^{k_i}_{j = 1} f_{ij}(x_{1}, ..., x_{n}) = 0)
\end{equation}
where $f_{ij}(x_{1}, ..., x_{n}) = 0$ are atomic formulas. The $\delta$-{\em weakening} of $\phi$ 
is the formula:
\begin{equation}
\phi^\delta = \exists^{I_{1}} x_{1}, ..., \exists^{I_{n}} x_{n}: \bigwedge^{m}_{i = 1}(\bigvee^{k_i}_{j = 1} |f_{ij}(x_1, ..., x_n)| \le \delta)
\end{equation}
\end{definition}
Note that $\phi$ implies $\phi^\delta$, while the converse is obviously not true.
The bounded $\delta$-satisfiability problem asks for the following: given a sentence of 
the form (\ref{def:sigma1}) and $\delta \in \mathbb{Q^+}$, correctly decide whether
\begin{itemize}
	\item \textbf{unsat}: $\phi$ is false,
	\item \textbf{$\delta$-true}: $\phi^\delta$ is true.
\end{itemize}
If the two cases overlap either decision can be returned: such a scenario reveals that the 
formula is {\em fragile} --- a small perturbation (\ie, a small $\delta$) can change the formula's
truth value. The dReal tool \cite{DBLP:conf/cade/GaoKC13} implements an algorithm for solving 
the $\delta$-satisfiability problem. Basically, the algorithm combines a DPLL procedure \cite{DPLL} 
(for handling the Boolean parts of the formula) with interval constraint propagation \cite{handbookICP}
(for handling the real arithmetic atoms). More details on the implementation can be found 
in \cite{DBLP:conf/cade/GaoKC13}.

A qualitative property of hybrid systems that can be checked is bounded $\delta$-reachability. 
It asks whether the system reaches the unsafe region after $k \in \mathbb{N}$ discrete transitions.


\begin{definition} \cite{DBLP:journals/corr/GaoKCC14}\label{def:reachability}
Bounded $k$ step $\delta$-reachability in hybrid systems can be encoded as a bounded $\Sigma_1$-sentence
\begin{equation} \label{eq:reachability}
\begin{split}
\exists \textbf{x}^{0}_{0,q_0}, \exists \textbf{x}^{t}_{0,q_0}, ... , \exists \textbf{x}^{0}_{0,q_m}, \exists \textbf{x}^{t}_{0,q_m} , ... , \exists \textbf{x}^{0}_{k,q_m}, \exists \textbf{x}^{t}_{k,q_m}:\\
(\bigvee_{q \in Q} (\text{init}_{q}(\textbf{x}^{0}_{0,q}) \wedge \text{flow}_{q}(\textbf{x}^{0}_{0,q},\textbf{x}^{t}_{0,q}))) \\
\wedge (\bigwedge_{i=0}^{k-1} (\bigvee_{q,q' \in Q}(\text{jump}_{q \rightarrow q'}(\textbf{x}^{t}_{i,q}, \textbf{x}^{0}_{i+1,q'}) \\
\wedge (\text{flow}_{q'}(\textbf{x}^{0}_{i+1,q'}, \textbf{x}^{t}_{i+1,q'}))) \wedge (\bigvee_{q \in Q} \text{unsafe}_{q}(\textbf{x}^{t}_{k,q}))))
\end{split}
\end{equation}
where $\textbf{x}^{0}_{i,q}$ and $\textbf{x}_{i,q}$ represent the continuous state in the mode $q$ at the depth $i$, and $q'$ is a successor mode.
\end{definition}

Intuitively, the formula above can be understood as follows: the first conjunction is asking for 
a set of continuous variables which satisfy the initial condition in one of the modes and the 
flow in that mode; the second conjunction is looking for a set of vectors which satisfy any $k$ 
discrete jumps and flows in each successor mode defined by the jumps; the third conjunction is 
verifying whether the state of the system (the mode and the set of continuous variables in the 
mode after $k$ jumps) belongs to the unsafe region. Note that the previous definition asks for
reachability in {\em exactly} $k$ steps. One can build a disjunction of formula (\ref{eq:reachability})
for all values from 1 to $k$, thereby obtaining reachability {\em within} $k$ steps.

The $\delta$-reachability problem can be solved using the described $\delta$-complete decision 
procedure, which will correctly return one of the following answers: 
\begin{itemize}
	\item \textbf{unsat}: the system never reaches the bad region $U$,
	\item \textbf{$\delta$-true}: the $\delta$-perturbation of (\ref{eq:reachability}) is true,
	and a witness, \ie, an assignment for all the variables, is returned.
\end{itemize}


