
Cyber-physical systems are characterised by the tight integration of digital computing
(the {\em cyber} part) with a {\em physical} environment or device. Such systems exploit the
flexibility of digital computing and communication to enhance or enable new capabilities 
of physical systems. Hybrid systems are mathematical models that combine continuous dynamics 
and discrete control, and enjoy widespread use for modelling cyber-physical systems.
For example, Stateflow/Simulink\footnote{\url{www.mathworks.com/simulink}} is the {\em de facto} 
standard tool for model-based design of embedded systems; the semantics of Stateflow/Simulink 
models can be given in terms of hybrid systems \cite{Tiwari02,SimulinkSemantics}. The use of 
cyber-physical systems 
permeates our society, including many safety-critical applications where a malfunctioning can 
result in threats to, or even loss of, human life. For example, modern aircraft are efficiently
flown most of the time by a computer program, while anti-lock brakes, stability control, and 
traction control contribute to safer cars. Again, electronic biomedical devices (\eg, digital 
infusion pumps that deliver drugs or nutrients to patients) offer superior flexibility and 
accuracy than traditional devices. It is therefore extremely important to verify formally that 
such cyber-physical systems are safe.

The state space of a hybrid system consists of a discrete component and of a continuous component. 
The fundamental {\em reachability} problem is to decide whether a hybrid system reaches an {\em unsafe} 
region of its state space (a subset of states indicating incorrect behaviour of the system).
Unfortunately, the reachability problem is undecidable even for simple hybrid systems with constant 
differential dynamics \cite{DBLP:conf/hybrid/AlurCHH92}. 
With further restrictions, \eg, same constant differential dynamics across all the variables --- 
timed automata --- the reachability problem becomes PSPACE-complete \cite{DBLP:conf/icalp/AlurD90}.
However, hybrid systems arising from practical applications feature much richer dynamics, including
non-linear functions over the reals, \eg, exponentials and trigonometric functions,
for which reachability questions are in general undecidable \cite{Richardson68}.
This means that the reachability problem must be modified, if we want to solve it algorithmically.
It has been recently noted that the reachability problem for general hybrid systems can be relaxed 
in a sound manner and encoded as a first-order satisfiability modulo theory (SMT) 
formula \cite{DBLP:conf/lics/GaoAC12} which can be correctly solved by a $\delta$-complete decision 
procedure \cite{DBLP:conf/cade/GaoAC12}. 
Such procedures allow a `tuneable' precision in the answer provided. Basically, SMT solvers and 
rigorous numerical methods (\eg, interval constraint propagation) are used 
to verify a conservative approximation of the system behaviour, so that bugs will always be detected. 
Although the over approximation can be tight (tuneable by an arbitrarily small rational parameter, 
$\delta$), the decision procedure may produce false alarms. It may report a bug when in reality 
the system is safe. However, such a scenario indicates that the system is safe but fragile, thereby 
providing valuable information to the system designer. The dReal tool \cite{DBLP:conf/cade/GaoKC13}
implements $\delta$-complete procedures for nonlinear arithmetic first-order formulae.

For many practical applications it is necessary to augment the definition of hybrid system with 
stochastic behaviour. Stochastic systems arise naturally when modelling phenomena which are 
intrinsically probabilistic, \eg, soft errors in computing hardware. Also, stochastic systems 
can arise due to uncertainty in (deterministic) system components, its behaviours, and its 
environment. The reachability problem for stochastic hybrid systems is no longer a decision
problem. Rather, it generalises that by asking what is the {\em probability} that the system
reaches the unsafe region. Note that for hybrid systems with both stochastic and non-deterministic
behaviour, the problem results in general in a range of probabilities, thereby becoming an
optimisation problem. In this paper we give a formal definition of the {\em probabilistic} bounded 
$\delta$-reachability problem for hybrid systems with random continuous initial parameters. 
We then show how $\delta$-complete decision procedures can be employed for solving this problem.
One of the constraints imposed by the $\delta$-complete procedure on the formula is that all 
the variables should be bounded. Nevertheless, we show how to solve a special case of the
reachability problem that include random variables with unbounded support.
Currently, dReal allows solving the reachability problems only for the systems with one random 
initial parameter and with dynamics given explicitly. However, many hybrid systems have 
multiple random parameters and dynamics defined by a system of ordinary differential equations 
(ODEs). This motivated us for implementing a validated procedure for integration.

Summarising, in this paper we make the following contributions:
\begin{itemize}
	\item we give a SMT formulation of the probabilistic $\delta$-reachability problem for 
		hybrid systems with random initial parameters;
	\item we show how to use $\delta$-complete procedures to solve probabilistic 
		$\delta$-reachability, also with random variables with unbounded support;
	\item we give a validated integration procedure that enables solving probabilistic 
		$\delta$-reachability of general hybrid systems with random initial parameters 
		and ODEs dynamics.
\end{itemize}
At the time of writing we are not aware of any other technique that addresses probabilistic bounded 
reachability from a rigorous point of view, \ie, guaranteed to be accurate numerically. 
