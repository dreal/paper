\section{Probabilistic Reachability}
%\begin{equation}
%\int_{\Omega} I_{B}(r)dP(r)
%\end{equation}
%where $dP(r)$ is a probability measure of the random variable and $Id_{B}$ is the indicator function defined as:
%\begin{equation}
%	I_{B}(r) = 
%	\begin{cases}
%		1, \text{if the system reaches $U$ on $k$-th step} \\
% 		0, \text{otherwise}
%	\end{cases}
%\end{equation}
We now present a more general procedure for calculating the probability of reaching 
the unsafe region in $k$ steps. The procedure works for general hybrid systems
whose dynamics is given implicitly, \eg, as a solution of ODEs. In particular, we do not
require the dynamics to be invertible with respect to the random parameter. The main idea 
is to compute the probability by integrating an indicator function over the probability measure 
of the random variable as:
\begin{equation*}
\int_{\Omega} I_{U}(r)dP(r)
\end{equation*}
where $P(r)$ is a probability measure of the random variable, $\Omega$ is the range of the 
random variable, and $I_{U}$ is the indicator function defined as:
\begin{equation*}
        I_{U}(r) = 
        \begin{cases}
                1, \text{system with parameter $r$ reaches $U$ in $k$-steps} \\
                0, \text{otherwise}
        \end{cases}
\end{equation*}
The procedure for solving probabilistic reachability thus consists of a validated integration 
procedure and a $\delta$-complete decision procedure used for verifying the indicator function
(and thus building the set $B$).

\noindent{\em Notation.}
For an interval $[r]=[\underline{r}, \overline{r}] \subset \mathbb{R}$ we denote the size of
the interval by $width([r]) = \overline{r} - \underline{r}$ and by                         
$mid([r]) = \frac{\overline{r} + \underline{r}}{2}$ the central point of the interval.

\subsection{Validated Integration Procedure}
In the implementation of a validated integration procedure we employ the (1/3) Simpson rule:
\begin{equation}\label{eq:Simpson}
\begin{split}
K= \int_{a}^{b} \, f(x)\, dx\, = \frac{width([I])}{6}(f(a) + \\
4 f(mid([I])) + f(b)) - \frac{width([I])^5}{2880}f^{(4)}(\xi)
\end{split}
\end{equation}
where $I=[a,b]$, $\xi \in [I]$ and $f^{(4)}$ is the fourth derivative of $f$.
Applying interval arithmetics an interval extension of function $f$ and its fourth derivative 
can be obtained.
\begin{definition}
An interval extension of a function $f:X \rightarrow Y$ is an operator $[\cdot]$ such that:
\begin{equation*}
	\forall x \in [r] \subseteq X: f(x) \in [f]([r]) \subseteq Y
\end{equation*}
\end{definition} 

The interval version of Simpson's rule can be obtained simply by replacing in (\ref{eq:Simpson})
occurences of $f$ and $f^{(4)}$ with their interval extension $[f]$, and by replacing $\xi$ with
the entire $[a,b]$ interval \cite{ValidatedIntegration}:
\[
\begin{split}
K \in [K]([a, b]) = \frac{width([I])}{6}([f](a) + 4 [f](mid([I])) + \\
[f](b)) - \frac{width([I])^5}{2880}[f]^{(4)}([I]).
\end{split}
\]
Furthermore, by definition of integral:
\[
\begin{split}
K \in \Sigma_{i = 1}^{n} [K]([x]_{i})
\end{split}
\]
where $n$ is a number of disjoint intervals $[x]_{i}$ in $[a, b]$. Interval extensions can
be readily computed using interval arithmetics libraries such as FILIB++ \cite{filib}.

In order to guarantee $\delta$-completeness of the integration it is sufficient to divide 
$[a, b]$ into $n$ disjoint intervals $[x]_{i}$ such that for each $[x]_{i}$ we have
$width([I]([x]_{i})) < \delta \frac{width([x]_{i})}{b - a}$. Then the exact value of the 
integral will be within an interval of width smaller than $\delta$. Pseudo-code for the 
procedure computing the integral up to an arbitrary $\delta \in \mathbb{Q}^{+}$ is given 
in Algorithm \ref{alg:integration}.

\begin{algorithm}
\label{alg:integration}
input: $f(x)$, $[a, b]$, $\delta$\;
output: $[I]$\;
$B$.push($[a, b]$)\;
$[I] = [0.0, 0.0]$\;
\While{$size(B) > 0$}{
	$[x]$ = $B$.pop()\;
	$[S]([x]) = \frac{width([x])}{6}([f](\underline{x}) + 4 [f](mid([x])) + [f](\overline{x})) - \frac{(width([x]))^5}{2880}[f^{(4)}]([x])$\;
	\If{$width([S]([x])) > \delta \frac{width([x])}{b - a}$}{
		$B$.push($[\underline{x}, mid([x])]$)\;
		$B$.push($[mid([x]), \overline{x}]$)\;
	}\Else{
		$[I] = [I] + [S]([x])$\;
	}
}
\Return $[I]$\;
\caption{Validated Integration Procedure}
\end{algorithm}

\subsection{Borel set}
Now we need to correctly identify the Borel set, or equivalently, verifying the indicator function
above. Let $\phi$ be a formula of the form:
\begin{equation} \label{eq:phi}
\begin{split}
\phi([r]) = \exists \rho \in [r], \exists \textbf{x}^{0}_{0,q_0}, \exists \textbf{x}^{t}_{0,q_0}, \exists t_{0, q_0}, ... , \exists \textbf{x}^{0}_{0,q_m}, \\ \exists \textbf{x}^{t}_{0,q_m}, \exists t_{0, q_m}, ... , \exists \textbf{x}^{0}_{k,q_m}, \exists \textbf{x}^t_{k,q_m}, \exists t_{k, q_m}:\\
(\bigvee_{q \in Q} (\text{init}_{q}(r, \textbf{x}^{0}_{0,q}) \wedge \text{flow}_{q}(r, \textbf{x}^{0}_{0,q}, \textbf{x}^{t}_{0,q}))) \wedge \\
(\bigwedge_{i=0}^{k-1} (\bigvee_{q,q' \in Q}(\text{jump}_{q \rightarrow q'}(r, \textbf{x}^{t}_{i,q}, \textbf{x}^{0}_{i+1,q'}) \wedge \\
(\text{flow}_{q'}(r, \textbf{x}^{0}_{i+1,q'}, \textbf{x}^{t}_{i+1,q'}))) \wedge (\bigvee_{q \in Q} \text{unsafe}_{q}(r, \textbf{x}^{t}_{k,q}))))
\end{split}
\end{equation}
If the formula is true then $[r]$ contains a value such that the system initial random parameter 
reaches the unsafe region.

Taking a complement of the unsafe region $U^{C} = S / U$ (where $S$ is the state space of the 
system) and defining a predicate $\text{unsafe}^{C}_{q}(r,\textbf{x}^{t})$ evaluating to true iff
$s = (q, r, \textbf{x}^{t}_{q}) \in U^{C}$ we want to ensure that the system never reaches the 
unsafe region {\em within} the $k$-th step with an initial random parameter from $[r]$. 
In order to conclude that it is sufficient to evaluate the formula:
\begin{equation} \label{eq:phi_C}
\begin{split}
\phi^{C}([r]) = 
	\exists \rho \in [r], \exists \textbf{x}^{0}_{0,q_0}, \exists \textbf{x}^{t}_{0,q_0}, \exists t_{0, q_0} , ... , \exists \textbf{x}^{0}_{0,q_m}, \\
\exists \textbf{x}^{t}_{0,q_m}, \exists t_{0, q_m}, ... , \exists \textbf{x}^{0}_{k,q_m}, \exists \textbf{x}^t_{k,q_m}, \exists t_{k, q_m}, \forall t'_{k, q_m} \in [0, t_{k, q_m}]:\\
(\bigvee_{q \in Q} (\text{init}_{q}(r, \textbf{x}^{0}_{0,q}) \wedge \text{flow}_{q}(r, \textbf{x}^{0}_{0,q}, \textbf{x}^{t}_{0,q}))) \wedge \\
(\bigwedge_{i=0}^{k-1} (\bigvee_{q,q' \in Q}(\text{jump}_{q \rightarrow q'}(r, \textbf{x}^{t}_{i,q}, \textbf{x}^{0}_{i+1,q'}) 
\wedge \\
(\text{flow}_{q'}(r, \textbf{x}^{0}_{i+1,q'}, \textbf{x}^{t}_{i+1,q'}))) \wedge \\
(\bigvee_{q \in Q} (\text{unsafe}^{C}_{q}(r, \textbf{x}^{t}_{k,q}) \wedge (\text{jump}_{q \rightarrow q'}(r, \textbf{x}^{t}_{k,q}, \textbf{x}^{0}_{k+1,q'})))))
\end{split}
\end{equation}
If the formula evaluates to true then the system does not reach the unsafe region on the $k$-th step.
Then set $B$ can be defined as the collection
$\{[r]_{i} : \phi([r]_{i}) \wedge (\neg \phi^{C}([r]_{i})\}$.

\subsection{Probabilistic reachability}
Integrating a probability density function of a random variable we obtain intervals 
such that size of the enclosure containing the exact value of a partial sum for each 
of them is smaller than a local error.

Then a decision about the relation between each of the obtained intervals and the (unknown) 
Borel set $B$ should be taken. In order to achieve this we use a $\delta$-complete decision 
procedure, \ie, dReal, to verify formulas (\ref{eq:phi}) and (\ref{eq:phi_C}). The following 
three outcomes are considered:
\begin{itemize}
	\item{$\phi([r])$ is $\delta$-\textbf{sat} and $\phi^C([r])$ is \textbf{unsat}}
	\item{$\phi([r])$ is \textbf{unsat}}
	\item{$\phi([r])$ is $\delta$-\textbf{sat} and $\phi^C([r])$ is $\delta$-\textbf{sat}}
\end{itemize}

In the first case, we can conclude that $[r] \cap B = [r]$. Therefore, $[r] \subseteq B$ and 
the value of the partial sum on this interval should be added to the resulting value of the 
integral. The second case allows concluding that $[r] \cap B = \emptyset$. Hence, this interval 
can be completely removed from the integration domain. Finally, in the last case, 
$[r] \cap B \ne \emptyset$ and $[r] \cap B \ne [r]$. In other words, the interval $[r]$ is not 
fully contained in the set $B$. Hence, this interval should be split into two subintervals and 
each of them should be checked in the same way.

After each run through the set of intervals $[r]_{i}$, the over-approximated $[I_{over}]$ and 
under-approximated $[I_{under}]$ interval values of the integral are calculated. The 
under-approximated value is calculated only over the intervals fully contained in set $B$ 
while the over-approximation of the integral involves values of partial sum on the intervals 
which are either fully or partially contained in set $B$. The exact value of the integral will 
always be within the interval $[\underline{I_{under}}, \overline{I_{over}}]$. The algorithm 
terminates when the length of the interval $[\underline{I_{under}}, \overline{I_{over}}]$ is 
smaller than $\delta$. Pseudo-code for the procedure is presented in Algorithm \ref{alg:prob-reach}.
\begin{algorithm}
\label{alg:prob-reach}
input: $f(x)$, $[a, b]$, $\delta_{1}$, $\delta_{2}$, $\phi$, $\phi^C$\;
output: $[I]$\;
\tcp{stack for all the intervals}
$B$.push($[a, b]$)\;
\tcp{stack only for the intervals and partial sum satisfying a local error}
$T = \emptyset$\;
\While{$size(B) > 0$}
{
	$[x]$ = $B$.pop()\;
	$[S]([x]) = \frac{width([x])}{6}([f](\underline{x}) + 4 [f](mid([x])) + [f](\overline{x})) - \frac{(width([x]))^5}{2880}[f^{(4)}]([x])$\;
	\If{$width([S]([x])) > \delta_{1} \frac{width([x])}{b - a}$}
	{
		$B$.push($[\underline{x}, mid([x])]$)\;
		$B$.push($[mid([x]), \overline{x}]$)\;
	}\Else
	{
		$T$.push($\{[x], [S]([x])\}$)\;
	}
}

$[I_{under}] = [0.0, 0.0]$\;
$[I_{over}] = [0.0, 0.0]$\;
$[I] = [0.0, 1.0]$\;
\tcp{stack containing extra divisions of the intervals}
$D = \emptyset$\;
\While{$width([I]) > \delta_{2}$}
{
	\While{$size(T) > 0$}
	{
	$\{[x], [S]([x])\} = T$.pop()\;
	\If{$\phi([x])$}
	{
		$[I_{over}] = [I_{over}] + [S]([x])$\;
		\If{$\phi^{C}([x])$}
		{
			$D$.push($\{[\underline{x}, mid([x])], [S([\underline{x}, mid([x])])]\}$)\;
			$D$.push($\{[mid([x])], \overline{x}, [S([mid([x])], \overline{x})]\}$)\;
		}\Else
		{
			$[I_{under}] = [I_{under}] + [S]([x])$\;
		}
	}
	$[I] = [\underline{I_{under}}, \overline{I_{over}}]$\;
	$[I_{over}] = [I_{under}]$\;
	$T = D$\;
	$D = \emptyset$\;	
	}
}

\Return $[I]$\;

\caption{Probabilistic $\delta$-reachability}
\end{algorithm}
