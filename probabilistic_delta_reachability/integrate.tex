In the previous section we argued that solving probabilistic reachability 
questions amounts to solving $\Sigma_1$ sentences of the following type:
\begin{equation} \label{eq:integration-problem}
\exists x \in [a, b] : \int_{a}^{x} f(t)\, dt \ge C
\end{equation}
for some constant $C\in [0,1]$.

\begin{proposition} \label{prop:ivp}
Let $f\mathord{:}[a,b] \rightarrow \mathbb{R}$ be a Lipschitz-continuous function. Then there 
exists one and only one continuous and differentiable function $F$ on $[a, b]$ 
such that $F(a) = 0$ and $\forall x \in [a, b]: \int_a^x f(t)\, dt = F(x)$
\end{proposition}

\begin{IEEEproof}
Let us formulate an initial value problem:
\begin{equation} \label{eq:IVP}
(F'(x) = f(x)) \wedge (F(a) = 0)
\end{equation}
where $f(x)$ is Lipschitz-continuous in $F(x)$ and continuous on $[a,b]$. 
By the Picard-Lindel\"{o}f theorem and the assumption $F(a)=0$, the formulated IVP 
has a unique solution on $[a, b]$ derived as following:
\begin{equation*} 
\forall x \in [a,b]: \int_a^x f(t)\, dt = F(x) - F(a) = F(x)
\end{equation*}
\end{IEEEproof}

Proposition \ref{prop:ivp} allows assessing the value of the integral by the value of the 
function which is obtained from the initial value problem (\ref{eq:IVP}). Therefore, 
the integration problem (\ref{eq:integration-problem}) is equivalent to the sentence
\begin{equation*} %\label{eq:encoding}
\exists x \in [a, b] : (F'(x) = f(x)) \wedge (F(x) \ge C) \wedge (F(a) = 0)
\end{equation*}
which can be directly solved by dReal as a $\delta$-satisfiability problem.

