We now introduce $\delta$-reachability in a probabilistic context.

\begin{definition}\label{def:pHS}
A hybrid system with {\em random initial parameters} consists of the following components:
\begin{itemize}
\item $Q = \{q_0, ..., q_m\}$ a set of modes (discrete components of the system),
\item $X = [u_1, v_1] \times ... \times [u_n, v_n] \times [0, T] \subset \mathbb{R}^{n+1}$ a domain 
	of continuous variables,
\item $R = [a_1, b_1] \times ... \times [a_l, b_l] \subset \mathbb{R}^{l}$ a domain of {\em random}
	parameters,
\item $S = Q \times R \times X$ is the state space of the system,
\item $U \subseteq S$ an {\em unsafe} region of the state space,
\end{itemize}
predicates (relations)
\begin{itemize}
	\item $\text{flow}_{q}(\textbf{r}, \textbf{x}^{0}, \textbf{x}^{t})$ mapping the continuous state 
	$\textbf{x}^{0}$ at time 0 to state $\textbf{x}^{t}$ at time point $t \in [0, T]$ in mode $q$
	\item $\text{init}_q(\textbf{r}, \textbf{x})$ indicating that $ (q, \textbf{r}, \textbf{x})$ 
	belongs to the set of initial states,
	\item $\text{jump}_{q \rightarrow q'}(\textbf{r}, \textbf{x}^{t}, \textbf{x}^{0})$ indicating that the 
	system can make a transition from mode $q$, upon reaching the jump condition in continuous 
	state $\textbf{x}^{t}$ at time point $t \in [0, T]$, to mode $q^\prime$ and setting the
	continuous state to $\textbf{x}^{0}$,
	\item $\text{unsafe}_{q}(\textbf{r}, \textbf{x})$ indicating that 
	$(q, \textbf{r}, \textbf{x}) \in U$,
\end{itemize}
and we require that for all $q\in Q$ the sets defined by $\text{flow}_q, \text{init}_q, \text{jump}_q$,
and $\text{unsafe}_q$ are Borel.
\end{definition}
\begin{remark}
The newly introduced random parameters are assigned in the initial mode and remain unchanged throughout
the system's evolution: note that flow and jump do not specify any change on $\textbf{r}$. Also, the
Borel assumption for the sets defined by the predicates is a theoretical requirement for well-definedness
of probabilities, and in practice it is easily satisfied. (We recall that Borel sets are obtained
through countable union and complement of open sets.)
\end{remark}

Then bounded $k$ step reachability problem for the defined system will be modified.
\begin{definition}
The bounded $k$ step $\delta$-reachability for hybrid systems with random initial parameters 
is the $\Sigma_1$ sentence:
\begin{equation*} %\label{eq:reachability-2}
\begin{split}
\exists \textbf{r}, \exists \textbf{x}^{0}_{0,q_0}, \exists \textbf{x}^t_{0,q_0}, ... , \exists \textbf{x}^{0}_{0,q_m}, \exists \textbf{x}^t_{0,q_m} , ... , \exists \textbf{x}^{0}_{k,q_m}, \exists \textbf{x}^t_{k,q_m}:\\
(\bigvee_{q \in Q} (\text{init}_{q}(\textbf{r}, \textbf{x}^{0}_{0,q}) \wedge \text{flow}_{q}(\textbf{r}, \textbf{x}^{0}_{0,q},\textbf{x}^t_{0,q}))) \\
\wedge (\bigwedge_{i=0}^{k-1} (\bigvee_{q,q' \in Q}(\text{jump}_{q \rightarrow q'}(\textbf{r}, \textbf{x}^t_{i,q}, \textbf{x}^{0}_{i+1,q'}) \\
\wedge (\text{flow}_{q'}(\textbf{r}, \textbf{x}^{0}_{i+1,q'}, \textbf{x}^t_{i+1,q'}))) \wedge (\bigvee_{q \in Q} \text{unsafe}_{q}(\textbf{r}, \textbf{x}^t_{k,q}))))
\end{split}
\end{equation*}
\end{definition}

If we associate a probability measure to the random parameters, then we can assess {\em quantitative} 
system properties (such as the probability that the system reaches the unsafe region).
In particular, we consider the following problem: 
\textit{what is the probability that a hybrid system with random initial parameters reaches the 
unsafe region in $k$ steps?}

To calculate the probability of reaching the unsafe region, we first need to find the set 
which contains all the values of the random variables satisfying the reachability property 
of a hybrid system. Such a set is constructed as follows:
\begin{equation} \label{def:Borel-set}
\begin{split}
B = \{\textbf{r} : \exists \textbf{x}^{0}_{0,q_0}, \exists \textbf{x}^t_{0,q_0}, ... , \exists \textbf{x}^{0}_{0,q_m}, \exists \textbf{x}^t_{0,q_m} , ... , \exists \textbf{x}^{0}_{k,q_m}, \exists \textbf{x}^t_{k,q_m}\mathord{:}\\
(\bigvee_{q \in Q} (\text{init}_{q}(\textbf{r}, \textbf{x}^{0}_{0,q}) \wedge \text{flow}_{q}(\textbf{r}, \textbf{x}^{0}_{0,q},\textbf{x}^t_{0,q}))) \\
\wedge (\bigwedge_{i=0}^{k-1} (\bigvee_{q,q' \in Q}(\text{jump}_{q \rightarrow q'}(\textbf{r}, \textbf{x}^t_{i,q}, \textbf{x}^{0}_{i+1,q'}) \\
\wedge (\text{flow}_{q'}(\textbf{r}, \textbf{x}^{0}_{i+1,q'}, \textbf{x}^t_{i+1,q'}))) \wedge (\bigvee_{q \in Q} \text{unsafe}_{q}(\textbf{r}, \textbf{x}^t_{k,q}))))\}.
\end{split}
\end{equation}
\begin{proposition}
The set $B$ defined by (\ref{def:Borel-set}) is Borel.
\end{proposition}
\begin{IEEEproof}
Immediate from the fact that (Definition \ref{def:pHS}) the sets defined 
by $\text{flow}_q, \text{init}_q, \text{jump}_q$, and $\text{unsafe}_q$ are Borel, and conjunction and 
disjunctions correspond to set intersection and union, respectively.
\end{IEEEproof}

The proposition entails that the probability that the system reaches the unsafe region is well-defined. 
Such a probability is computed by integrating the probability measure of  the random variables over the 
set (\ref{def:Borel-set}). In particular, we need to compute the following integral
\begin{equation} \label{eq:probabilistic-reachability}
\int_{B} \,dP(\textbf{r})
\end{equation}
where $B$ is the set (\ref{def:Borel-set}) and $P(\textbf{r})$ is a probability measure over 
the random parameters $\textbf{r}$. In case that the random parameters are independent then 
$P(\textbf{r}) = \Pi_{i=1}^{l} P_i(r_{i})$, where $P_i$ is the probability measure of 
parameter $r_i$. For most practical applications, the infinitesimal probability measure 
$dP(\textbf{r})$ is simply given by a (probability) density function.

