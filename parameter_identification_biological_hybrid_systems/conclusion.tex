\section{Conclusion}

We have presented a framework using $\delta$-complete decision procedures for the parameter identification 
of hybrid biological systems. We have used $\delta$-satisfiable formulas to describe parameterized hybrid automata 
and to encode parameter synthesis problems. We have employed $\delta$-decision procedures to perform bounded model 
checking, and developed an algorithm to obtain the resulting parameters.
We have demonstrated the applicability of our method on a highly nonlinear hybrid model of a cardiac cell that
% Bing: Reviewer#1 "The statement that this cannot be handled by state of the art tools requires further justification"
% cannot
are difficult to
be handled by other verification tools. We have successfully ruled out a model candidate which did not fit 
experimental observations, and we have identified critical parameter ranges that can induce cardiac disorders.

% Bing: Reviewer#3 "Indeed the SBML standard allows for specifying such models (jump conditions are called "events") and there are existing software packages capable of implementing and simulating them (eg COPASI), which would also be appropriate to mention in their introduction or discussion."
It is worth noting that our method can be applied to ODE based models with discrete events, which are special forms of hybrid automata. Such models are often specified using the Systems Biology Markeup Language (SBML) %\cite{sbml} 
and archived in the BioModels database \cite{biomodels}. Currently, we are currently developing an SBML-to-dReal translator to facilitate the $delta$-decision based analysis of dozens of such models. 
Further, our method also has the potential to be applied to model classes such as hybrid functional Petri nets models \citep{hfpn}. We plan to explore this in future work. Another interesting direction is applying our method for parameter estimation from experimental data. By properly encoding the noisy wet-lab experimental data using logic formulas, bounded model checking can be utilized to find the unknown parameter values.
% Bing: Reviewer#1 "You suggested to apply due to noisy wet-lab data statistical model checking. The number of runs (so to speak) of many wet-lab experiments are rather limited often only 3 maybe 7 - so I am not sure whether statistical model checking will work here. Maybe I understood this wrong, so please clarify" 
In this respect, the specification logic used in \cite{liu13} promises to offer helpful pointers.



% Bing: [todo] Reviewer#1 "However, the approach also inherits the problems of hybrid automata for specifying large hybrid systems, therefore I assume you mention Hybrid Petri Nets in the conclusion, have you thought about other possibilities to specify hybrid systems, like e.g. multi-formalisms as realized in Ptolemy, maybe this point could be discussed in more detail"



%%% Local Variables:
%%% TeX-master: "main"
%%% End:
