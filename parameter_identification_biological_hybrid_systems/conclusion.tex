\section{Conclusion}

We have presented a framework using $\delta$-complete decision procedures for the parameter identification 
of hybrid biological systems. We have used $\delta$-satisfiable formulas to describe parameterized hybrid automata 
and to encode parameter synthesis problems. We have employed $\delta$-decision procedures to perform bounded model 
checking, and developed an algorithm to obtain the resulting parameters. 
Our verified numerical integration and constraint programming algorithms effectively compute an over-approximation of the system dynamics. An  $\mathsf{unsat}$ answer can always be trusted, while a $\delta$-$\mathsf{sat}$ answer might be due to the over-approximation (see Section 2 for more details). We chose this behavior as it better fits with the safety requirements expected by formal verification.
We have demonstrated the applicability of our method on a highly nonlinear hybrid model of a cardiac cell that are
difficult to analyze with other verification tools. We have successfully ruled out a model candidate which did not fit experimental observations, and we have identified critical parameter ranges that can induce cardiac disorders.

It is worth noting that our method can be applied to ODE based models with discrete events, which are special forms of hybrid automata. Such models are often specified using the Systems Biology Markeup Language (SBML) %\cite{sbml} 
and archived in the BioModels database \cite{biomodels}. Currently, we are currently developing an SBML-to-dReal translator to facilitate the $\delta$-decision based analysis of SBML models. 
Further, our method also has the potential to be applied to other model formalisms such as hybrid functional Petri nets \cite{hfpn} and the formalisms realized in Ptolemy \cite{ptolemy}. We plan to explore this in future work.
Another interesting direction is applying our method for parameter estimation from experimental data. By properly encoding the noisy wet-lab experimental data using logic formulas, bounded model checking can be utilized to find the unknown parameter values.
In this respect, the specification logic used in \cite{liu13} promises to offer helpful pointers.




%%% Local Variables:
%%% TeX-master: "main"
%%% End:
