\section{Conclusion} %[TODO]
Hybrid automata are well-studied formalisms for modeling the behavior of biological systems. In this article, we have presented a framework using $\delta$-complete decision procedures for the parameter identification of hybrid biological systems. We have used the $\lrf$-formulas to describe parameterized hybrid automata and encode parameter synthesis problems. Determining the satisfiability of $\lrf$-formulas with nonlinear real functions is undecidable. To overcome this, we have employed the $\delta$-decision procedures to perform bounded model checking, and developed an interval constrains propagation based algorithm to obtain the resulting parameters.

We have demonstrated the applicability of our method with the help of two hybrid biological models. In the prostate cancer case study, our method successfully ruled out model candidates which hopelessly fit the experimental observation. We also designed personalized treatment schemes for individual patients. By investigating a highly nonlinear model of the cardiac cell, we have identified critical parameters that can induce cardiac disorders.

Our $\delta$-decisions based parameter synthesis method has the potential to be applied to model classes such as hybrid functional Petri nets models \citep{hfpn}. We plan to explore this in our future work. Another interesting direction will be applying our method to tackling the parameter estimation problem, which is currently one of the most important challenges in systems biology. By properly encoding the noisy web-lab experimental data using logic formulas, bounded model checking can be utilized to estimate the unknown parameter values. In this connection, a model checking based parameter estimation framework presented in \cite{liu13} promises to offer helpful pointers.

%Finally, we will develop a GPU-based implementation of our method to exploit the inherent massive parallelism in generating trajectories through numerical integration.





\paragraph{Funding:} This work has been partially supported by the National Science Foundation (NSF), NSF Expeditions in Computing.

\paragraph{Conflict of Interest:} none declared.
