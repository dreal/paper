\section{Methods}

We encode reachability problems of hybrid automata using a first-order language $\lrf$ over the reals. We then use delta-complete decision procedures to solve these formulas. We have studied the logic and delta-decision problems in previous work~\cite{}, and a brief review is given in the Appendix. 
\subsection{Hybrid Automata and {\large$\lrf$}-Encoding of Reachability}\label{language}
\begin{definition}[$\lrf$-Representations of Hybrid Automata]\label{lrf-definition}
A hybrid automaton in $\lrf$-representation is a tuple
\begin{multline*}
H = \langle X, Q, \{{\flow}_q(\vec x, \vec y, t): q\in Q\},\{\inv_q(\vec x): q\in Q\},\\
\{\jump_{q\rightarrow q'}(\vec x, \vec y): q,q'\in Q\},\{\init_q(\vec x): q\in Q\}\rangle
\end{multline*}
where $X\subseteq \mathbb{R}^n$ for some $n\in \mathbb{N}$, $Q=\{q_1,...,q_m\}$ is a finite set of modes, and the other components are finite sets of quantifier-free $\lrf$-formulas.
\end{definition}
\begin{example}[Nonlinear Bouncing Ball]
The bouncing ball is a standard hybrid system model. It can be $\lrf$-represented in the following way:
\begin{itemize}
\item $X = \mathbb{R}^2$ and $Q = \{q_u, q_d\}$. We use $q_u$ to represent bounce-back mode and $q_d$ the falling mode.
\item $\flow = \{\flow_{q_u}(x_0, v_0, x_t, v_t, t), \flow_{q_d}(x_0, v_0, x_t, v_t, t)\}$. We use $x$ to denote the height of the ball and $v$ its velocity. Instead of using time derivatives, we can directly write the flows as integrals over time, using $\lrf$-formulas:
\begin{itemize}
\item $\flow_{q_u}(x_0, v_0, x_t, v_t, t)$ defines the dynamics in the bounce-back phase:
$$(x_t = x_0 + \int_0^{t} v(s) ds) \wedge (v_t = v_0 + \int_0^t g(1-\beta v(s)^2) ds)$$
\item $\flow_{q_d}(x_0, v_0, x_t, v_t, t)$ defines the dynamics in the falling phase:
$$(x_t = x_0 + \int_0^{t} v(s) ds) \wedge (v_t = v_0 + \int_0^t g(1+\beta v(s)^2) ds)$$
\end{itemize}where
$\beta$ is a constant. Again, note that the integration terms define Type 2 computable functions.
\item $\jump = \{\jump_{q_u \rightarrow q_d} (x, v, x', v'), \jump_{q_d \rightarrow q_u} (x, v, x', v')\}$ where
\begin{itemize}
\item $\jump_{q_u \rightarrow q_d} (x, v, x', v')$ is $(v= 0 \wedge x' = x \wedge v' = v)$.
\item $\jump_{q_d \rightarrow q_u} (x, v, x', v')$ is $(x= 0 \wedge v' = \alpha v\wedge x'=x)$,  for some constant $\alpha$.
\end{itemize}
\item $\init_{q_d}$ is $(x=10 \wedge v=0)$ and $\init_{q_u}$ is $\bot$.
\item $\inv_{q_d}$ is $(x>=0 \wedge v>=0)$ and $\inv_{q_u}$ is $(x>=0 \wedge v<=0)$.
\end{itemize}
\end{example}
We now define the $\lrf$-encoding of bounded reachability. The encodings are standard bounded model checking, and have been studied in existing work but without the generality of a full $\lrf$-language. As a result, some issues have not been discovered. For example, the full encoding of non-deterministic flows with invariant conditions require second-order quantification, and the first-order encoding requires additional assumptions. We will give the full $\lrf$-encodings and discuss such details.
\begin{notation}
Let $H$ be a hybrid automaton. We use $\unsafe = \{\unsafe_q:q\in Q\}$ as the $\lrf$-representation of an unsafe region in the state space of $H$. We can write $\llbracket \unsafe\rrbracket = \bigcup_{q\in Q} \llbracket \unsafe_q \rrbracket\times \{q\}$.
\end{notation}
First, we need to define a set of auxiliary formulas that will be important for ensuring that a particular mode is picked at a certain step.
\begin{definition}
Let $Q = \{q_1,...,q_m\}$ be a set of modes. For any $q\in Q$, and $i\in\mathbb{N}$, use  $b_{q}^i$ to represent a Boolean variable. We now define
$$\enforce_Q(q,i) = b^i_{q} \wedge \bigwedge_{p\in Q\setminus\{q\}}\neg b^{i}_{p}$$
$$\enforce_Q(q, q',i) = b^{i}_{q}\wedge \neg b^{i+1}_{q'} \wedge \bigwedge_{p\in Q\setminus\{q\}} \neg b^i_{p} \wedge \bigwedge_{p'\in Q\setminus\{q'\}} \neg b^{i+1}_{p'}$$
We omit the subscript $Q$ when the context is clear.\end{definition}

We define the following formula that checks whether an unsafe region is reachable after exactly $k$ steps of discrete transition in a hybrid system.
\begin{definition}[$k$-Step Reachability, Invariant-Free Case]
Suppose $H$ is invariant-free, and $U$ a subset of its state space represented by $\unsafe$. The $\lrf$-formula $\reach_{H,U}(k,M)$ is defined as:
\begin{eqnarray*}
%\reach^{k,M}(H,U) &:=&
& &\exists^X \vec x_{0} \exists^X\vec x_{0}^t\cdots \exists^X \vec x_{k}\exists^X\vec x_{k}^t\exists^{[0,M]}t_0\cdots \exists^{[0,M]}t_k.\\
& &\bigvee_{q\in Q} \Big(\init_{q}(\vec x_{0})\wedge \flow_{q}(\vec x_{0}, \vec x_{0}^t, t_0)\wedge \enforce(q,0)\Big)\\%\wedge (b_{q_i}\wedge \bigwedge_{q\neq q_i} \neg b_{q})
\wedge & & \bigwedge_{i=0}^{k-1}\bigg( \bigvee_{q, q'\in Q} \Big(\jump_{q\rightarrow q'}(\vec x_{i}^t, \vec x_{i+1})\wedge \enforce(q,q',i)\\
& & \hspace{4.7cm}\wedge\flow_{q'}(\vec x_{i+1}, \vec x_{i+1}^t, t_{i+1})\wedge \enforce(q',i+1)\Big)\bigg)\\
\wedge & & \bigvee_{q\in Q} \unsafe_q(\vec x_{k}^t).
\end{eqnarray*}
\end{definition}
Intuitively, the trajectories start with some initial state satisfying $\init_q(\vec x_{0})$ for some $q$. In each step, it follows $\flow_q(\vec x_{i}, \vec x_{i}^t, t)$ and makes a continuous flow from $\vec x_i$ to $\vec x_i^t$ after time $t$. When $H$ makes a $\jump$ from mode $q'$ to $q$, it resets variables following $\jump_{q'\rightarrow q}(\vec x_{k}^t, \vec x_{k+1})$. The auxiliary $\enforce$ formulas ensure that picking $\jump_{q\rightarrow q'}$ in the $i$-the step enforces picking $\flow_q'$ in the $(i+1)$-th step.
When the invariants are not trivial, we need to ensure that for all the time points along a continuous flow, the invariant condition holds. Thus, we need to universally quantify over time. This is a fact that has been previously discussed, for instance, in~\cite{DBLP:conf/fmcad/CimattiMT12}. However, if we allow nondeterministic flows, the situation is more complicated, which has not been discovered in existing work. We give the encoding for systems with only deterministic flows first, as follows:
\begin{definition}[$k$-Step Reachability, Nontrivial Invariant and Deterministic Flow]\label{br2}
Suppose $H$ contains invariants and only deterministic flow
, and $U$ a subset of its state space represented by $\unsafe$. In this case, the $\lrf$-formula $\reach_{H,U}(k,M)$ is defined as:
\begin{eqnarray*}
& &\exists^X \vec x_{0} \exists^X\vec x_{0}^t\cdots \exists^X \vec x_{k}\exists^X\vec x_{k}^t \exists^{[0,M]}t_0\cdots \exists^{[0,M]}t_k.\\
& &\bigvee_{q\in Q} \Big(\init_{q}(\vec x_{0})\wedge \flow_{q}(\vec x_{0}, \vec x_{0}^t, t_0)\wedge \enforce(q,0)\\
& &\hspace{5cm} \wedge \forall^{[0,t_0]}t\forall^X\vec x\;(\flow_{q}(\vec x_{0}, \vec x, t)\rightarrow \inv_{q}(\vec x))\Big) \\
\wedge & &\bigwedge_{i=0}^{k-1}\bigg( \bigvee_{q, q'\in Q} \Big(\jump_{q\rightarrow q'}(\vec
x_{i}^t, \vec x_{i+1})\wedge \flow_{q'}(\vec x_{i+1}, \vec x_{i+1}^t, t_{i+1})\wedge \enforce(q,q',i)\\
& & \hspace{1.5cm}\wedge\enforce(q',i+1)\wedge \forall^{[0,t_{i+1}]}t\forall^X\vec x\;(\flow_{q'}(\vec x_{i+1}, \vec x,
t)\rightarrow \inv_{q'}(\vec x)) )\Big)\bigg)\\
\wedge & &\bigvee_{q\in Q} (\unsafe_q(\vec x_{k}^t)\wedge \enforce(q,k)).
\end{eqnarray*}
\end{definition}
The extra universal quantifier for each continuous flow expresses the requirement that for all the time points between the initial and ending time point ($t\in[0,t_i+1]$) in a flow, the continuous variables $\vec x$ must take values that satisfy the invariant conditions $\inv_q(\vec x)$.

