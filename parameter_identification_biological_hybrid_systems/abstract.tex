\begin{abstract} % word limit = 250
\section{Motivation:}
Biological systems can possess multiple operational modes with specific nonlinear dynamics in each mode. Hybrid automata and its variants are often used to model and analyze the dynamics of such systems. Highly nonlinear hybrid models are difficult to analysis and usually have many parameters. An important problem is to identify parameter values using which the model can reach certain states of interests.
\section{Results:}
We present a parameter identification framework using $\delta$-complete decision procedures, which can solve satisfiability modulo theories (SMT) problems over the reals with a wide range of nonlinear functions, including ordinary differential equations (ODEs). We have demonstrated our method on two highly nonlinear hybrid systems: the prostate cancer progression model and the cardiac cellular action potential model. The results show that the parameter identification framework is convenient and efficient for performing model selection and personalized therapy optimization. We have also identified crucial parameter ranges related to cardiac disorders.
\section{Availability:} The source code is freely available at \\
\href{http://dreal.cs.cmu.edu/dreach.html}{http://dreal.cs.cmu.edu/dreach.html}

\section{Contact:} liubing@cs.cmu.edu

\section{Supplementary information:} Supplementary data are available at \emph{Bioinformatics} online.

\end{abstract}

%% [TODO] 


%% old
%Biological systems can possess multiple operational modes with specific nonlinear dynamics in each mode. Hybrid automata and its variants are often used to model and analyze the dynamics of such systems. An important problem is to identify parameters for a hybrid model so that the model can reach certain states. To tackle this problem, we present a framework using $\delta$-complete decision procedures, which can solve SMT problems over the reals with a wide range of nonlinear functions, including ODEs. We demonstrate our methods on two highly nonlinear biological hybrid systems. The results show that our method scales and can obtain biological insights that are consistent with experimental observations 