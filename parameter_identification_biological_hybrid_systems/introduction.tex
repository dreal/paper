\section{Introduction}

Computational modeling and analysis methods are playing a crucial role in understanding the complex dynamics of biological systems \citep{liu12jbcb}. In this paper we address the parameter synthesis problem for hybrid models of biological systems.
This problem amounts to finding sets of parameter values for which a model satisfies some precise 
behavioral constraints, such as time series or reachability properties. We focus on hybrid 
continuous/discrete models, since one of the key aspects of many biological systems is their differing 
behavior in various `discrete' modes. For example, it is well-known that the five stages of the cell 
cycle are driven by the activation of different signaling pathways. Hence, hybrid system models are often 
used in systems biology (see, \eg, 
\citep{chen04,tomlin04,Hu04,ye08,aihara10,antoniotti03,lincoln04,baldazzi11}).

Hybrid systems combine discrete control computation with continuous-time evolution. The state space 
of a hybrid system is defined by a finite set of continuous variables and modes. In each mode, the
continuous evolution ({\em flow}) of the system is usually given by the solution of ordinary differential
equations (ODEs). At any given time a hybrid system dwells in one of its modes and each variable 
evolves accordingly to the flow in the mode. Jump conditions control the switch to another mode,
possibly followed by a `reset' of the continuous variables. Thus, the temporal dynamics of a 
hybrid system is piecewise continuous.

Hybrid models of biological systems often involve many parameters such as rate constants of 
biochemical reactions, initial conditions, and threshold values in jump conditions. Generally, only 
a few rate constants will be available or can be measured experimentally --- in the latter case
the rate constants are obtained by fitting the model to experimental observations. Furthermore, 
it is also crucial to figure out what initial conditions or jump conditions may lead to an unsafe
state of the system, especially when studying hybrid systems used to inform clinical 
therapy \citep{tanaka10}. All such questions fall within the \textit{parameter synthesis} problem,
which is extremely difficult for hybrid systems. Even simple reachability questions for hybrid 
systems with linear (differential) dynamics are undecidable \citep{henzinger96}.
Therefore, the parameter synthesis problem needs to be relaxed in a sound manner in order to solve
it algorithmically --- this is the approach we shall follow.


In this paper, we tackle the parameter synthesis problem using $\delta$-complete procedures \cite{gao12a} 
for deciding first-order formula with arbitrary computable real functions, including solutions 
of Lipschitz-continuous ODEs \citep{gao12b}. Such procedures may return answers with
one-sided $\delta$-bounded errors, thereby overcoming undecidability issues (note that the maximum 
allowable error $\delta$ is an arbitrarily small positive rational). In our approach we describe 
the set of states of interest as a first-order logic formula and perform bounded model checking \cite{BMC}
to determine reachability of these states. We then adapt an interval constrains propagation based algorithm
to explore the parameter space and identify the sets of resulting parameters. 
%
We show the applicability of our method by carrying out a thorough case study characterized by 
highly nonlinear hybrid models. We discriminate two cardiac cell action potential 
models \cite{fenton98,orovio08} in terms of cell-type specificity and identify parameter ranges for which a cardiac cell may lose excitability.
%The results show that our method can obtain biological insights that are consistent with experimental  observations, scales to complex systems. In particular, the cardiac model studied is out of the scope of state-of-the-art tools.
The results show that our method scales to complex systems and can gain interesting biological insights that would be difficult to obtain using state-of-the-art tools such as HyTech? \cite{hytech} and Breach? \cite{breach}.
%
% Bing: please help to address the following comments 
% reviewer#1 "You state that the cardiac model studied is out of the scope of state of the art tools. This is a rather strong statement. Please note that a lot of work exist in traditional modeling and simulation (even outside the area of systems biology), as parameter estimation is one of the essential tasks in all simulation studies. So please be more concrete which state of the art tools you have in mind, and where it has been shown that those do not suffice for the problem at hand.



\hide{
\vspace{-2ex}
\paragraph{{\bf Our Contribution.}}
To summarize, in this paper we tackle parameter synthesis for hybrid biological models 
and we make the following contributions:
\vspace{-1ex}
\begin{itemize}
	\item we show how to encode the parameter synthesis problem as a bounded $\delta$-satisfiable
	      formula;
	\item we present a novel parameter synthesis technique based on $\delta$-complete procedures and
	      bounded model checking;
	\item we showcase the performance of our method by analyzing two hybrid models featuring 
	      highly nonlinear behavior which is out of the scope of state-of-the-art tools.
\end{itemize}
}

%\vspace{-2ex}
\paragraph{Related Work.}
A survey of modeling and analysis of biological systems using hybrid models can be found in \cite{luca08}.
Analyzing the properties of biochemical networks using formal verification techniques is being actively
pursued by a number of researchers, for which we refer to Brim's {\em et al.} recent 
survey \cite{BrimSFM13}.
Of particular interest in our context are parameter synthesis methods for qualitative behavior 
specifications (\eg, temporal logic formulas). The method introduced in \cite{rovergene} can deal 
with parameter synthesis for piecewise affine linear systems. For ODEs, 
Donz\'{e} {\em et al.}~\cite{donze} explore the parameter space using adaptive sampling and simulation, 
while Palaniappan {\em et al.} \cite{liu13} use a statistical 
model checking approach. Other techniques perform a sweep of the entire (bounded) parameter space, 
after it has been discretized \cite{Calzone06,Donaldson08}. Randomized optimization techniques were used
for parameter estimation in stochastic hybrid systems \cite{Koutroumpas08}, while identification
techniques for affine systems were used in \cite{Cinquemani08}.
The techniques above can handle nonlinear hybrid systems only through sampling and 
simulation, and so are incomplete. Our approach is instead $\delta$-complete, so that if a model 
is found to be unfeasible, then this is correct (see Section 2 for more details).

% Bing: do we need to address the following comment? 
% Reviewer#1 "You state that all the other approaches can only handle hybrid systems by sampling and simulation. However, if I understand the appendix correctly, your approach also uses numerical simulation to compute a function, see Page 13. Referring to the delta function I am wondering whether this is related to the problem of the step sizes of numerical integration algorithms and thus the problem e.g., encountered in threshold crossing in hybrid simulation and traditionally addressed by bi-section methods (up to a certain error epsilon), see also your algorithm 1. It seems the language puts together two things that are often kept separately in modeling and simulation, i.e., a model of the dynamics and some requirements referring to the model behavior (its goal), maybe you can comment on this. BTW: how does your approach relate to simulation-based model checking of deterministic models as e.g. proposed by Francois Fages?"



\hide{
The rest of the paper is organized as follows. The next section introduces the necessary 
theoretical background on $\delta$-complete decision procedures. 
In Section we formulate the parameter synthesis problem for hybrid automata. 
In Section 2.3, we present our novel techniques for synthesizing parameters using 
$\delta$-complete decision procedures. In the subsequent section we present a case study in which
we analyze in depth a hybrid model of the cardiac cell action potential.
In the final section, we summarize our results and discuss future work.
}

%%% Local Variables:
%%% TeX-master: "main"
%%% End:
