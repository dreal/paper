A wide range of static analysis is done on continuous programs, i.e., programs that implement continuous functions, usually over the real numbers. The interesting cases involve highly nonlinear functions. The
most basic cases are the ones in the standard C library that
implements elementary functions using floating point arithmetic. We
propose a method for verifying these programs that combines testing,
numerical analysis, and SMT solving, which aim to provide full
coverage in verification. The method can be understood as a type of
concolic testing, but the emphasis is on continuous functions and
numerical properties make it worthwhile to study specific techniques.
We generate point inputs to obtain preliminary coverage, and then use
SMT formulas to verify the behavior of the implementations around the
neighborhood of the tested points. In this way the algorithm can
completely cover any bounded interval of interest. We present a
comprehensive study of the floating point implementation of nonlinear
elementary functions in the Embedded GNU C library, and report serious
bugs in the library through the method, and as well as verified
intervals that guarantee correct implementations.

%%% Local Variables:
%%% mode: latex
%%% TeX-master: "main"
%%% End:
