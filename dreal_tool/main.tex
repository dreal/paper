\documentclass[envcountsect]{llncs}
%\usepackage{stmaryrd,amsmath,amssymb,newlfont,graphicx,caption,verbatim}
\usepackage{amsmath,amssymb,newlfont,graphicx,caption,verbatim,hyperref}
\usepackage[ruled,lined,boxed,commentsnumbered,linesnumbered]{algorithm2e}
\newcommand{\dom}{\mathrm{dom}}
\newtheorem{notation}[theorem]{Notation}
\newcommand{\len}{\mathit{len}}
\newcommand{\poly}{\mathsf{poly}}

%\setlength{\textwidth}{5.7in}
%\setlength{\textheight}{8.2in}
%\setlength{\topmargin}{0in}
%\setlength{\oddsidemargin}{.4in}
%\setlength{\evensidemargin}{.4in}

\title{{\sf dReal}: An SMT Solver for Nonlinear Theories over the Reals\thanks{This research was sponsored by the National Science Foundation grants no. DMS1068829, no. CNS0926181 and no. CNS0931985, the GSRC under contract no. 1041377, the Semiconductor Research Corporation under contract no.~2005TJ1366, and the Office of Naval Research under award no.~N000141010188.}}
\author{Sicun Gao \and Soonho Kong \and Edmund M. Clarke}
\institute{Carnegie Mellon University, Pittsburgh, PA 15213}


\begin{document}
\maketitle

\begin{abstract}
We describe the open-source tool {\sf dReal}, an SMT solver for nonlinear formulas over the reals. The tool can handle various nonlinear real functions such as polynomials, trigonometric functions, exponential functions, etc. {\sf dReal} implements the framework of $\delta$-complete decision procedures: It returns either {\sf unsat} or {\sf $\delta$-sat} on input formulas, where $\delta$ is a numerical error bound specified by the user. {\sf dReal} also produces certificates of correctness for both {\sf $\delta$-sat} (a solution) and {\sf unsat} answers (a proof of unsatisfiability).  
%{\sf dReal} integrates Interval Constraint Propagation methods in the DPLL(T) framework to handle nonlinearity, and is designed to be easily extendable with other numerical algorithms.
\end{abstract}

\section{Introduction}

SMT formulas over the real numbers can encode a wide range of problems in theorem proving and formal verification. Such formulas are very hard to solve when nonlinear functions are involved. Our recent work on {$\delta$-complete decision procedures} provided a new framework for this problem~\cite{DBLP:conf/cade/GaoAC12,DBLP:conf/lics/GaoAC12}. We say a decision procedure is {\em $\delta$-complete} for a set $S$ of SMT formulas, where $\delta$ is a positive rational number, if for any $\varphi$ from $S$, the procedure returns one of the following:
\begin{itemize}
 \item {\sf unsat}: $\varphi$ is unsatisfiable.
 \item {\sf $\delta$-sat}: $\varphi^{\delta}$ is satisfiable.
\end{itemize}
Here, $\varphi^{\delta}$ is a syntactic variant of $\varphi$ that encodes a notion of numerical perturbation on logic formulas~\cite{DBLP:conf/cade/GaoAC12}. With such relaxation, $\delta$-complete decision procedures can fully exploit the power of scalable numerical algorithms to solve nonlinear problems, and at the same time provide suitable correctness guarantees for many correctness-critical problems. {\sf dReal} implements this framework. It solves SMT problems over the reals with nonlinear functions, such as polynomials, sine, exponentiation, logarithm, etc. The tool is open-source\footnote{{\sf dReal} is available at \url{http://dreal.cs.cmu.edu}.}, built on {\sf opensmt}~\cite{DBLP:conf/tacas/BruttomessoPST10} for the
high-level DPLL(T) framework, and {\sf
realpaver}~\cite{DBLP:journals/toms/GranvilliersB06} for the Interval
Constraint Propagation algorithm. It returns {\sf unsat} or {\sf $\delta$-sat}
on input formulas, and the user can obtain certificates (proof of
unsatisfiability or solution) for the answers.

In this paper we describe the usage, design, and some results of the tool.

\paragraph{Related Work.} 

SMT solving for nonlinear formulas over the reals has gained much attention in recent years and many tools are now available. The symbolic approaches include Cylindrical Decomposition~\cite{collins}, with significant recent improvement~\cite{PassmoreJ09,DBLP:conf/cade/JovanovicM12}, and Gr\"obner bases~\cite{PlatzerQR09}. A drawback of symbolic algorithms is that it is restricted to arithmetic, namely polynomial constraints, with the exception of~\cite{metitarski}. On the other hand, many practical solvers incorporate scalable numerical computations. Examples of numerical algorithms that have been exploited include optimization algorithms~\cite{BorrallerasLNRR09,DBLP:conf/fmcad/NuzzoPSS10}, interval-based algorithms~\cite{HySAT,DBLP:conf/atva/EggersFH08,DBLP:conf/fmcad/Gao10}, Bernstein polynomials~\cite{bern}, and linearization algorithms~\cite{cordic}. All solvers show promising results on various nonlinear benchmarks. Our goal is to provide an open-source platform for the rigorous combination of numerical and symbolic algorithms under the framework of $\delta$-complete decision procedures~\cite{DBLP:conf/cade/GaoAC12}. 

\section{Usage}

\subsection{Input Format}

We accept formulas in the standard SMT-LIB 2.0 format~\cite{BarST-SMT-10} with extensions. In addition to nonlinear arithmetic (polynomials), we allow transcendental functions such as $\sin, \tan, \arcsin$, $\arctan$, $\exp$, $\log$, pow, $\sinh$. More nonlinear functions (for instance, solution of differential equations) can be added when needed, by providing the corresponding numerical evaluation algorithms. Floating-point numbers are allowed as constants in the formula. 

Bound information on variables can be declared using a list of simple atomic formulas. For instance ``{\tt (assert (< 0 x))}", which sets $x\in (0, +\infty)$ at parsing time. Also, the user can set the precision by writing
``{\tt (set-info :precision 0.0001)}." The default precision is $10^{-3}$, and can be set through command line.
\begin{example}\label{example} The following is an example input file.
It is taken from the Flyspeck project~\cite{DBLP:conf/dagstuhl/Hales05}. (Filename {\tt flyspeck/172.smt2}. Flyspeck {\tt ID (6096597438b)})
 \begin{verbatim}
(set-logic QF_NRA)
(set-info :precision 0.001)
(declare-fun x () Real)
(assert (<= 3.0 x))
(assert (<= x 64.0))
(assert (not (> (- (* 2.0 3.14159265) (* 2.0 (* x (arcsin (* (cos
0.797) (sin (/ 3.14159265 x))))))) (+ (- 0.591 (* 0.0331 x))
(+ (* 0.506 (/ (- 1.26 1.0) (- 1.26 1.0))) 1.0)))))
(check-sat)
(exit)
\end{verbatim}
\end{example}

\subsection{Command Line Options}

After building, {\tt dReal} can be simply used through:
\begin{verbatim}
dReal [--verbose] [--proof] [--precision <double>] <filename>
\end{verbatim}
The default output is {\sf unsat} or {\sf delta-sat}. When the flags are
enabled, the following output will be provided.
\begin{itemize}
 \item If {\tt --verbose} is set, then the solver will output the detailed
decision traces along with the solving process. 
\item If {\tt --proof} is set, the solver produces an addition file
``{\tt filename.proof}'' upon termination, and provides the following
information.
\begin{itemize}
\item If the answer is {\sf delta-sat},
then {\tt filename.proof} contains a witnessing solution, plugged into a
$\delta$-perturbation of the original formula, such
that the correctness can be easily checked externally. 

\item If the answer is {\sf unsat}, then {\tt filename.proof} contains a trace
of the solving steps, which can be verified as a proof tree that establishes the
unsatisfiability of the formula.  
\end{itemize}
\item The {\tt --precision} flag gives the option of overwriting the
default precision, and the one set in the benchmark.
\end{itemize}
When the {\tt --proof} flag is set, the solver produces a file that certifies
the answer. In the {\tt delta-sat} case, the solution is plugged in the
formula, and its correctness can be checked externally. For the {\tt unsat}
cases, we provide a proof checker that verifies the proof. It can be used with the following command:
\begin{verbatim}
proofcheck [--timeout <int>] <filename>
\end{verbatim}
The proof checker will create a new folder called {\tt filename.extra}, which
contains auxiliary files needed. It is possible for the proof
checking procedure to produce a large number of new files, so setting a
 timeout is important. By default, the timeout is {\tt 30min}. The proof
checker will return either ``{\sf proof verified}'' or ``{\sf timeout}''.

\begin{example}
With default parameters, {\sf dReal} solves the formula in
Example~\ref{example} in {\tt 10ms}, returning {\sf unsat}, on a machine with
a 32-core 2.3GHz AMD Opteron Processor and 94GB of RAM. We then run {\sf proofcheck} on the same machine. The proof checker returns
``{\sf proof verified}'' in 10.08s, after making 8 branching steps and
checking 77 axioms.
\end{example}

\section{Design}

\subsection{The $\delta$-Decision Problem}\label{lbl:deltas}

The standard decision problem is undecidable for SMT formulas
over the reals with trigonometric functions. Instead, we proposed to focus on the so-called $\delta$-decision problem, which relaxes the standard decision problem. Let $\delta$ be any positive rational number. On a given SMT formula $\varphi$, we ask for one of the
following answers:
\begin{itemize}
 \item {\sf unsat}: $\varphi$ is unsatisfiable.
 \item {\sf $\delta$-sat}: $\varphi^{\delta}$ is satisfiable.
\end{itemize}
When the two cases overlap, either answer can be returned. Here,
$\varphi^{\delta}$ is called the $\delta$-perturbation (or $\delta$-weakening)
of $\varphi$, which is formally defined as follows.
\begin{definition}[$\delta$-Weakening~\cite{DBLP:conf/cade/GaoAC12}]
Let $\delta\in \mathbb{Q}^+\cup\{0\}$ be a constant and $\varphi$ be a
$\Sigma_1$-sentence in a standard form $\varphi:= \exists^{\vec I}\vec
x\;(\bigwedge_{i=1}^m (\bigvee_{j=1}^{k_i}
f_{ij}(\vec x)= 0))$. The $\delta$-weakening of $\varphi$ defined as:
$\varphi^{\delta}:= \exists^{\vec I} \vec x\;(\bigwedge_{i=1}^m(\bigvee_{j=1}^k
|f_{ij}(\vec x)|\leq \delta)).$
\end{definition}
Solving the $\delta$-decision problem is as useful as the standard one for many
problems. For instance, suppose we perform bounded
model checking on hybrid systems, and encode safety properties as an SMT
formula $\varphi$. Then following standard model checking techniques, if we
decide that $\varphi$ is {\sf unsat}, then the system is indeed ``safe'' with in
some bounds; if we decide that $\varphi$ is {\sf $\delta$-sat}, then the system
would become ``unsafe'' under some $\delta$-perturbation on the system. In this
way, when $\delta$ is reasonably small, we have essentially taken into account
the robustness properties of the system, and can justifiably conclude that the
system is unsafe in practice.

\subsection{DPLL$\langle$ICP$\rangle$}

Interval Constraint Propagation (ICP)~\cite{handbookICP} is a constraint solving
algorithm that finds
solutions of real constraints using a ``branch-and-prune'' method, combining
interval arithmetic and constraint propagation. The idea is to use interval
extensions of functions to ``prune'' out sets of points that are not in the
solution set, and ``branch'' on intervals when such pruning can not be done,
until a small enough box that may contain a solution is found. In a DPLL(T) framework, ICP can be used as the theory solver that checks the
consistency of a set of theory atoms. We use
{\sf opensmt}~\cite{DBLP:conf/tacas/BruttomessoPST10} for the
general
DPLL(T) framework, and integrate {\sf realpaver}~\cite{DBLP:journals/toms/GranvilliersB06} which performs ICP. We now describe the design of the interface. A high-level structure of the theory solver is shown in Algorithm \ref{algo1}. 
\begin{algorithm}[h!]
\SetKwData{Left}{left}\SetKwData{This}{this}\SetKwData{Up}{up}
\SetKwFunction{Union}{Union}\SetKwFunction{FindCompress}{FindCompress}
\SetKwInOut{Input}{input}\SetKwInOut{Output}{output}
\Input{A conjunction of theory atoms, seen as constraints, $c_1(x_1,...,x_n),...,c_m(x_1,...,x_n)$, the initial interval bounds on all variables $B^0 = I^0_1\times \cdots \times I^0_n$, box stack $S=\emptyset$, and precision $\delta\in \mathbb{Q}^+$.}
\Output{{\sf $\delta$-sat}, or {\sf unsat} with learned conflict clauses.}
\BlankLine
$S.\mathrm{push}(B_0)$\;
\While{$S\neq \emptyset$}{\label{while}
$B\leftarrow S.\mathrm{pop}()$ \;
\While{$\exists 1\leq i \leq m, B\neq \mathrm{Prune}(B,c_i)$}{ 
%\For{$j\leftarrow 1$ \KwTo $m$}{
\CommentSty{//Pruning without branching, used as the assert() function.}
	$B\leftarrow\mathrm{Prune}(B, c_i)$\;
%	\If{$B=\emptyset$}{break\;}
}
\CommentSty{//The $\varepsilon$ below is computed from $\delta$ and the Lipschitz constants of functions beforehand.}\\
\If{$B\neq \emptyset$}
{\eIf{$\exists 1\leq i\leq n, |I_i|\geq \varepsilon$}
{$\{B_1,B_2\}\leftarrow \mathrm{Branch}(B, i)$;\CommentSty{  //Splitting on the intervals}
$S.\mathrm{push}(\{B_1,B_2\})$\;}{
return {\sf $\delta$-sat}; \CommentSty{   //Complete check() is successful.}}}
}
return {\sf unsat}\;
\caption{Theory Solving in DPLL$\langle$ICP$\rangle$}\label{algo1}
\end{algorithm}


\paragraph{Check and Assert.} For incomplete checks in the assert function,
we use the pruning operator provided in ICP to contract the interval assignments
on all the variables, by eliminating the boxes in the domain that do not contain
any solutions. At complete checks, we perform both pruning and branching, and
look for one interval solution of the system. That is, we prune and branch on
the interval assignment of all variables, and stop when either we have obtained
an interval vector that is smaller than the preset error bound, or when we have
traversed all the possible branching on the interval assignments.
\paragraph{Backtracking and Learning.} We maintain a stack of assignments on the variables,
which are mappings from variables to unions of intervals. When we reach a
conflict, we backtrack to the previous environment in the pushed stack. We also collect all the
constraints that have appeared in the pruning process leading to the conflict.
We then turn this subset of constraints into a learned clause and add it to the
original formula.
\paragraph{Witness for $\delta$-Satisfiabillity.} When the answer is {\sf $\delta$-sat} on $\varphi(\vec x)$, we provide a solution $\vec a\in \mathbb{R}^n$, such
that $\varphi^{\delta}(\vec a)$ is a ground formula that can be easily checked
to be true. It is important to note that the
solution witnesses $\delta$-satisfiability, instead of standard satisfiability
of the original formula. While the latter problem is undecidable, {\em any} point in the interval assignment returned by ICP can witness the satisfiability
of $\varphi^{\delta}$ when the intervals are smaller than an appropriate error
bound.
\paragraph{Proofs of Unsatisfiability.} When the answer is {\sf unsat}, we
produce a proof tree that can be verified to establish the validity of the
negation of the formula, i.e., $\forall \vec x \neg\varphi(\vec x)$. We
devised a simple first-order natural deduction system, and transform the
computation trace of the solving process into a proof tree. We then use
interval arithmetic and simple rules to check the correctness of the proof
tree. The proof check procedure recursively divide the problem into subproblems
with smaller domains. More details can be found in~\cite{cade2013extended}.

\section{Results}

Besides solving the standard benchmarks~\cite{DBLP:conf/cade/JovanovicM12} (data shown on the tool website), we managed to solve many challenging nonlinear benchmarks from the Flyspeck project~\cite{DBLP:conf/dagstuhl/Hales05} for the formal proof of the Kepler conjecture. The following is a typical formula:
\begin{align*}
\forall \vec x&\in [2, 2.51]^6.\ \Big( -\frac{\pi-4\arctan\frac{\sqrt
2}{5}}{12\sqrt2}\sqrt{\Delta(\vec x)}\\
&+\frac{2}{3}\sum_{i=0}^3\arctan\frac{\sqrt{\Delta(\vec x)}}{a_i(\vec x)}\leq
-\frac{\pi}{3}+4\arctan\frac{\sqrt 2}{5}\Big)
\end{align*}
where $a_i(\vec x)$ are quadratic and $\Delta(\vec x)$ is the
determinant of a nonlinear matrix. We solved 828 out of the 916 formulas (returning {\sf
  unsat}) with a timeout of 5 minutes and $\delta=10^{-3}$, without domain-specific heuristics. The proof traces of these formulas can be large. In Table~\ref{tbl:exp}, we list
some of the representative benchmarks to show scalability. Complete
tables are on the tool page.
\input{good_benchmarks.tex}

\bibliographystyle{abbrv}
\bibliography{tau_cade}
\end{document}
