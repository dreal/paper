% !TEX root = /Users/kquine/Dropbox/Research/Papers/2015/CPS-SMT-RTSS/cps-rtss.tex


\section{PALS Definitions}
\label{sec:pals-defs}




\begin{definition}
 For a typed machine $M_j = (D_{i_j}, S_j, D_{o_j}, \delta_{M_j})$ with 
 input set $D_{i_j} = D_{i_{j_1}} \times \cdots \times D_{i_{j_n}}$, an input adaptor
 $\mathit{adap}(j)=\{\alpha_i: D'_l \rightarrow D_{i_l}\}_{l\in\{1,\ldots, n_j\}}$
 is a family of functions to give desired input values
 $(\alpha_1(d_1),\ldots,\alpha_n(d_n))$ %for $M_j$'s input ports 
 from output $(d_1,\ldots,d_n) \in D_1 \times \cdots \times D_n$ of other machines.  
\end{definition}


\begin{definition}
A \emph{multirate machine ensemble} is 
given by
a tuple
$\mathcal{E} = (J_S ,  J_F, e, \{M_j\}_{j\in J_S \cup J_F}, E, \mathit{src}, \mathit{rate}, \mathit{adap})$
with:
\begin{inparaenum}[(i)]
	\item $J_S$ a set of slow machine indices;
	\item $J_F$ a set of fast machine indices ($J_S \cap J_F = \emptyset$);
	\item $e\not\in J_S \cup J_F$ the environment index;
 	\item $\{M_j\}_{j\in J_S \cup J_F}$  a family of typed machines;
	\item $E=(D^e_i, D^e_o)$ 
	the \emph{environment} with $D^{e}_i$  the \emph{input set} and 
	$D^{e}_o$  the \emph{output set};
	\item $\mathit{src}$  a \emph{wiring diagram} assigning to each input port its source output port, 
	 so that no connection exists between two ``fast'' machines;
	\item $\mathit{rate}$ assigning to a fast 
	machine its \emph{rate};
	%denoting how many times faster than the slow machines it is; 
	and
	\item $\mathit{adap}$ assigning to a machine its \emph{input adaptor}.
	%for  communication between components with different rates.
\end{inparaenum}
\end{definition}

\begin{definition}
The \emph{synchronous composition}
of  an ensemble
$\mathcal{E} = (J_S, J_F, e, \{M_j\}_{j\in J_s \cup J_F}, E, \mathit{src}, \mathit{rate}, \mathit{adap})$
is the typed machine
$ M_{\mathcal{E}} = ( D_i^{\mathcal{E}}, S^{\mathcal{E}},D_o^{\mathcal{E}}, \delta_{\mathcal{E}})$
where:
\begin{inparaenum}[(i)]
	\item $D_i^{\mathcal{E}} = D^e_o$ and  $D_o^{\mathcal{E}} = D^e_i$;
	\item $S^{\mathcal{E}} = (\Pi_{j\in J} S_j) \times (\Pi_{j\in J} D_{OF}^j)$, 
	where
	$D^j_{OF}$ stores the ``feedback outputs'' of subcomponents; and
	\item $ \delta_{\mathcal{E}} \subseteq 
	(D^{\mathcal{E}}_i\times S^{\mathcal{E}}) \:\times \: (S^{\mathcal{E}} \times D_o^{\mathcal{E}})$
	``combines''  the  transitions of the subcomponents
	 into a synchronous step (see \cite{pals-tcs}).
\end{inparaenum}
\end{definition}

\begin{definition}
The transition system for an ensemble $\mathcal{E}$ is a pair 
$\mathit{ts}(\mathcal{E}) = (S^{\mathcal{E}} \times \mathcal{D}_i^{\mathcal{E}}, \longrightarrow_{\mathcal{E}})$,
where $(\vec{s}_1,\vec{i}_1) \longrightarrow_{\mathcal{E}} (\vec{s}_2,\vec{i}_2)$
iff an ensemble in state $\vec{s}$ and with input
$\vec{i}$ from the environment has a transition to state $\vec{s'}$
(i.e., $\exists \vec{o}.\; ((\vec{i}, \vec{s}), (\vec{s'}, \vec{o})) \in \delta_{\mathcal{E}}$)
and the environment can generate output $\vec{i'}$ in the next step.
\end{definition}


\begin{definition}  
Consider $n$ controlled physical environments
$E_{M_i} = (C_i, {\vec{x}}_i, \Lambda_i)$
with $\vec{x}_i =  (x_{i_1}, \ldots,x_{i_{l_i}})$
for $i = 1,\ldots,n$.
Given a variable $t$ for time,
a \emph{time-invariant constraint}
is a logic formula of the form  
$(\forall t)\; \psi(\vec{x}_1(t), \vec{x}_2(t), \ldots, \vec{x}_n(t))$.
\end{definition}
