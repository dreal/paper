% !TEX root = /Users/kquine/Dropbox/Research/Papers/2015/CPS-SMT-RTSS/cps-rtss.tex


\section{Introduction}


%Virtually synchronous CPS, such as airplanes and cars, 
%are implemented as a network of distributed devices designed in a synchronous way.
\emph{Virtually synchronous} cyber physical systems (CPS)
consist of a number of distributed controllers that should logically act together in a synchronous way. 
The devices may operate at different frequencies, but the communication happens at their hyper-period boundary.
The devices %in the system 
may also interact with their % local 
\emph{physical environment}
whose continuous dynamics is modeled by ordinary differential equations
(e.g., avionics, aerospace,  automotive, robotics, etc). 
The design and analysis of such systems is very hard because of race conditions, clock skews, network delays, and execution times. % in addition to the usual difficulties for non-distributed CPS.
For this reason, 
PALS (physically asynchronous, but logically synchronous) 
\cite{pals-rtss09,mr-pals-journal,pals-tcs,al2012pattern} has been
developed to reduce the system complexity of a virtually  synchronous CPS,
provided that the network infrastructure gives \emph{bounds}
on computation times, network delays, and imprecisions of the local clocks.


However, 
formal analysis of virtually synchronous CPS with continuous dynamics is still very difficult.
Inputs or parameters of CPS can be any real numbers in a certain bound.
Different components will read their physical values at possibly different times 
according to their local clocks. 
The continuous dynamics of CPS often involves 
\emph{nonlinear} ordinary differential equations (ODE)
which may not provide exact solutions in general.
%Numerical simulations of ODEs cannot verify the correctness of the system.
Although %the 
PALS %methodology 
can greatly reduce the analysis effort 
caused by distributed real-time systems,
those difficulties raised by distributed hybrid systems
cannot be removed without using low-quality approximation on ODEs and local clocks.
%which prevents to precisely analyze such hybrid systems.



In this paper we present symbolic techniques using SMT solving to analyze 
distributed CPS along with PALS.
Formal analysis problems of distributed CPS, 
involving nonlinear ODEs and clock skews, 
are reduced to the satisfaction of SMT formulas over the real numbers and ODEs.
%
The satisfaction problem
(generally undecidable)
 is decidable up to any given precision $\delta > 0$ \cite{delta-comp,sat-ode}.
The number $\delta$, provided by the user, 
is the bound on numerical errors that is tolerable in the analysis.
By this approach, 
we can apply state-of-the-art SMT techniques that are %proven to be 
effective for analyzing \emph{discrete} controllers,
%combined with $\delta$-complete decision procedures
while precisely analyzing
the continuous dynamics of CPS
in a robust way under IO sampling or timing jitters %numerical perturbations 
by $\delta$.


One obstacle to SMT-based analysis is 
that it is not scalable for distributed hybrid systems with nonlinear ODEs.
%The number of physical variables considered is usually less than dozens, 
%and the temporal bound is typically less than hundreds. 
We cope with this scalability problem as follows.
First, the discrete part and the continuous part of the system are encoded as SMT formulas separately, 
and ODEs are considered \emph{only after} the discrete part of the system is fully analyzed.
Second, inductive SMT analysis is used to verify safety properties of distributed CPS for unbounded time.
Third, compositional SMT analysis is used to deal with distributed CPS with an arbitrary number of components. 
These methods are readily available for distributed CPS by simplifying the analysis problem using the PALS methodology. 
%Without PALS, finding appropriate inductive or compositional conditions would be very difficult for distributed CPS.




Another  obstacle to this approach is that 
SMT-based analysis 
is currently very inefficient
for \emph{distributed} hybrid systems with nonlinear ODEs. %, and often unfeasible in practice.
In distributed CPS, 
physical variables of different components can be ``physically'' correlated to each other.
For example, if we consider two adjacent rooms with thermostat controllers, the temperature of one room immediately affects the temperature of the other room. %unless they are perfectly insulated.
The behavior is specified as \emph{coupled} ODEs
in which variables evolve simultaneously over continuous time.
%and ODE solving needs to take into account all coupled variables at the same time.
%Therefore, 
Existing SMT approaches only use the non-modular encoding
for such distributed hybrid systems. 
%
The size of the SMT formula can be huge,
and  this leads to the \emph{formula explosion problem}
that makes such SMT-based analysis practically infeasible.

For this reason, 
we present a novel SMT framework
to  effectively encode formal analysis problems
of distributed hybrid systems with coupled ODEs \emph{in a modular way}.
%
For a composition $H_1 \parallel \cdots \parallel H_n$ of $n$ hybrid systems,
when each component $H_i$ has $m_i$ control modes,
the size of the SMT formula by the new modular encoding is proportional to $\sum_{i=1}^n k_i$,
whereas by the previous non-modular encoding the size is proportional to $\prod_i k_i$.
%
The key idea is to use uninterpreted real function symbols %and universal quantification
to logically decompose coupled variables in systems of ODEs in the SMT formula.

We provide an efficient SMT decision procedure for such formulas by the new encoding
for distributed hybrid systems.
We show that 
the satisfaction problems of the new logical theory 
can be reduced to ones in the standard theory of real numbers \emph{at no cost}.
%
We have implemented our new SMT techniques within the \textsf{dReal} SMT solver \cite{dReal},
and performed experiments on nontrivial distributed CPS.
%\begin{inparaenum}[(i)]
%	\item a controller for turning an airplane,
%	\item a controller for the attitude of a quadrotor UAV,
%	\item a steam boiler controller,
%	\item networked water tank controllers,
%	\item networked thermostat controllers, and
%	\item an automated highway system.
%\end{inparaenum}
The experimental results show that our techniques can dramatically increase 
the performance of SMT-based analysis for these systems
that involve multiple control modes and nonlinear ODEs.
% To the best of our knowledge, modular SMT encoding for 
%(coupled) nonlinear ODEs and its efficient decision procedure have never been developed for distributed hybrid systems.

%


%
%Using PALS and inductive/compositional analysis, together with PALS, 
%We have verified safety properties of these systems for unbounded time,
%and for any number of components for parameterized CPS (iv)--(vi).
%Such distributed CPS
%cannot be easily analyzed by other existing %SMT-based 
%approaches for hybrid systems.


The rest of the paper is organized as follows.
%
Section~\ref{sec:pals} briefly recalls PALS.
Section~\ref{sec:hybrid-pals} explains how PALS can be applied to distributed hybrid systems.
%involving nonlinear ODEs and clock skews. 
Section~\ref{sec:smt-encoding} shows how PALS models for distributed CPS
and their analysis problems (such as bounded reachability, inductive, and compositional analysis)
can be symbolically encoded as logic formulas.
Section~\ref{sec:smt-logic} presents a new SMT framework that allows 
modular encoding for distributed hybrid systems. % with coupled ODEs.  
Section~\ref{sec:case-studies} presents six case studies and their verification 
using our methods based on PALS and SMT.
Finally,
Section~\ref{sec:related-work} explains some related work,
and 
Section~\ref{sec:concl} presents some concluding remarks.



