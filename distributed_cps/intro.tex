% !TEX root = /Users/kquine/Dropbox/Research/Papers/2015/CPS-SMT-RTSS/cps-rtss.tex


\section{Introduction}

%% Paragraph 1
\emph{Virtually synchronous} distributed real-time systems consist of
a number of \emph{distributed} controllers that should logically
behave in a synchronous way. [[The devices may operate at different
frequencies, but  communication happens at their hyper-period
boundary.]] Designing and analyzing  such systems is very hard because
of race conditions, clock skews, network delays, execution times, and
the state space explosion caused by the interleavings. To overcome
these problems, 
the \emph{PALS} (physically asynchronous logically synchronous) methodology
\cite{pals-rtss09,mr-pals-journal,pals-tcs,al2012pattern} has been
developed to reduce the design and verification of a virtually
synchronous distributed real-time system to the much simpler task  of designing and
verifying the underlying \emph{synchronous} models,
provided that the  infrastructure can guarantee bounds
on computation times, network delays, and imprecisions of the local clocks.




% %OLD paragraph 1:
% %Virtually synchronous CPS, such as airplanes and cars, 
% %are implemented as a network of distributed devices designed in a synchronous way.
% \emph{Virtually synchronous} cyber physical systems (CPS)
% consist of a number of distributed controllers that should logically act together in a synchronous way. 
% The devices may operate at different frequencies, but the communication happens at their hyper-period boundary
% (e.g., avionics, aerospace,  automotive, robotics, etc). 
% The design and analysis of such systems is very hard because of race conditions, clock skews, network delays, and execution times. % in addition to the usual difficulties for non-distributed CPS.
% For this reason, 
% the PALS (physically asynchronous logically synchronous) methodology
% \cite{pals-rtss09,mr-pals-journal,pals-tcs,al2012pattern} has been
% developed to reduce the system complexity of a virtually  synchronous CPS,
% provided that the network infrastructure gives \emph{performance bounds}
% on computation times, network delays, and imprecisions of the local clocks.


%% New paragraph 2:
In this paper
we present \emph{Hybrid PALS} that extends  PALS 
to the important class of virtually synchronous \emph{cyber-physical
  systems} (CPSs)---where controllers also interact with their local \emph{physical environments},
whose \emph{continuous dynamics} can be  specified by using ordinary
differential equations (ODE)---which includes avionics, automotive, robotics, and medical systems. 
We prove a \emph{bisimulation equivalence} relating
distributed hybrid systems and their underlying synchronous models.
%This means that the design and analysis of distributed hybrid systems
%can be reduced to  those of the synchronous models.
%
The initial steps towards a hybrid extension of PALS were taken
in~\cite{hybrid-pals}. However, 
that work imposes strong  conditions on sampling and response times
of sensors and actuators, and, furthermore,  does not give 
a bisimulation equivalence. % (but only a trace equivalence). % between a synchronous model and a distributed hybrid model. 



% %% OLD paragraph 2:
% In this paper
% we present \emph{Hybrid PALS} that extends the PALS methodology 
% to virtually synchronous CPS given by \emph{distributed hybrid systems}.
% The devices also interact with their local \emph{physical environments},
% with continuous dynamics specified by using ordinary differential equations (ODE).
% We show that there exists a \emph{bisimulation equivalence} relating
% distributed hybrid systems and their underlying synchronous models.
% This means that the design and analysis of distributed hybrid systems
% can be reduced to  those of the synchronous models.
% %
% Hybrid PALS was first proposed in \cite{hybrid-pals}, but 
% the work imposes strong stringent conditions on sampling and response timings
% of sensors and actuators. Moreover, \cite{hybrid-pals} does not show
% a bisimulation equivalence (but only a trace equivalence). % between a synchronous model and a distributed hybrid model. 


%% NEW paragraph 3:
Verifying the synchronous Hybrid PALS models is still challenging:
inputs and parameter values can be any real number in certain
intervals; different components  read their physical values and give
actuator commands at  different times 
because of  local clock skews, and these timings cannot be abstracted
away; and the continuous dynamics  often involves 
\emph{nonlinear} ODEs
which in general may not have exact solutions.
%Numerical simulations of ODEs cannot verify the correctness of the system.
% Although the 
% PALS methodology 
% can greatly reduce the analysis effort 
% caused by distributed real-time systems,
% the difficulties raised by distributed hybrid systems
% cannot be removed without using low-quality approximation on ODEs and local clocks.
%which prevents to precisely analyze such hybrid systems.



% %%OLD paragraph 3:
% However, 
% formal analysis of virtually synchronous CPS with continuous dynamics is still very difficult.
% Inputs or parameters of CPS can be any real numbers in a certain bound.
% Different components will read their physical values at possibly different times 
% according to their local clocks. 
% The continuous dynamics of CPS often involves 
% \emph{nonlinear} ODEs
% which may not provide exact solutions in general.
% %Numerical simulations of ODEs cannot verify the correctness of the system.
% Although the 
% PALS methodology 
% can greatly reduce the analysis effort 
% caused by distributed real-time systems,
% the difficulties raised by distributed hybrid systems
% cannot be removed without using low-quality approximation on ODEs and local clocks.
% %which prevents to precisely analyze such hybrid systems.

%%% Polished paragraph 4:
This paper presents SMT solving techniques to address the challenges
of analyzing Hybrid PALS models of CPSs, where the formal analysis of
a distributed CPS involving ODEs and clock skews is reduced to
checking the satisfiability of SMT formulas over real numbers and ODEs.
This  satisfiability problem
 is decidable up to any user-given precision $\delta > 0$ \cite{delta-comp,sat-ode}.
%The number $\delta$, provided by the user, 
%is the bound on numerical errors that is tolerable in the analysis.
[[In this way, 
we can apply state-of-the-art SMT solving techniques that are %proven to be 
effective for analyzing \emph{discrete} controllers,
%combined with $\delta$-complete decision procedures
while  analyzing
the continuous dynamics of a CPS up to precision $\delta$.]] 


%%% OLD Paragraph 4
% We present symbolic techniques using SMT solving to analyze 
% distributed CPS along with PALS.
% %An SMT (satisfiability modulo theories) problem is to check the satisfiability of first-order formulas with respect to certain decidable logical theories
% Formal analysis problems of distributed CPS, 
% involving nonlinear ODEs and clock skews, 
% are reduced to the satisfaction of SMT formulas over the real numbers and ODEs.
% %
% The satisfiability problem
% (generally undecidable)
%  is decidable up to any given precision $\delta > 0$ \cite{delta-comp,sat-ode}.
% The number $\delta$, provided by the user, 
% is the bound on numerical errors that is tolerable in the analysis.
% By this approach, 
% we can apply state-of-the-art SMT techniques that are %proven to be 
% effective for analyzing \emph{discrete} controllers,
% %combined with $\delta$-complete decision procedures
% while precisely analyzing
% the continuous dynamics of CPS
% in a robust way under I/O sampling or timing jitters %numerical perturbations 
% by $\delta$.

%%% Polished Paragraph 5:
One problem with  SMT-based methods is 
that they do not scale well for distributed hybrid systems with nonlinear ODEs.
%The number of physical variables considered is usually less than dozens, 
%and the temporal bound is typically less than hundreds. 
We cope with this scalability problem as follows.
First, the discrete part and the continuous part of the system are
encoded separately, 
so that ODEs are considered \emph{only after} the discrete part of the
system has been  fully analyzed.
Second, in addition to bounded reachability analysis, \emph{inductive}
SMT analysis is used to verify safety properties,  without any time
bound. 
Third, compositional SMT analysis is used to analyze each part of a
distributed CPS by a divided-and-conquer approach. % an arbitrary
                                % number of components.  
%These methods are readily available for distributed CPS by simplifying the analysis problem using the PALS methodology. 
%Without PALS, finding appropriate inductive or compositional conditions would be very difficult for distributed CPS.

%%% OLD paragraph 5:
% One obstacle to SMT-based methods is 
% that it is not scalable for distributed hybrid systems with nonlinear ODEs.
% %The number of physical variables considered is usually less than dozens, 
% %and the temporal bound is typically less than hundreds. 
% We cope with this scalability problem as follows.
% First, the discrete part and the continuous part of the system are separately encoded in SMT, 
% so that ODEs are considered \emph{only after} the discrete part of the system is fully analyzed.
% Second, inductive SMT analysis, besides bounded reachability, is used to verify safety properties of distributed CPS for unbounded time.
% Third, compositional SMT analysis is used to analysis each part of distributed CPS by a divided-and-conquer approach. % an arbitrary number of components. 
% %These methods are readily available for distributed CPS by simplifying the analysis problem using the PALS methodology. 
% %Without PALS, finding appropriate inductive or compositional conditions would be very difficult for distributed CPS.


%%% Polished Paragraph 6:
%%Another  obstacle %to this approach 
%%is that  SMT-based analysis 
%%is currently inefficient
%%for \emph{distributed} hybrid systems with nonlinear ODEs. 
In distributed CPS, the 
physical states of different components can be  correlated to each other.
Consider, for example,  two adjacent rooms: the temperature of one room then immediately affects the
temperature of the other room. %unless they are perfectly insulated. 
The behavior is specified as \emph{coupled} ODEs
where variables evolve simultaneously over  time.
%and ODE solving needs to take into account all coupled variables at the same time.
%Therefore, 
Existing SMT approaches use  non-modular encodings
for such distributed hybrid systems. 
%
The size of the SMT formula can therefore be huge, which 
 leads to the \emph{formula explosion problem}
that makes  SMT-based analysis practically infeasible.

%%% OLD paragraph 6:
% Another  obstacle %to this approach 
% is that  SMT-based analysis 
% is currently inefficient
% for \emph{distributed} hybrid systems with nonlinear ODEs. 
% In distributed CPS, 
% physical states of different components can be physically correlated to each other.
% For example, if we consider two adjacent rooms with thermostat controllers, the temperature of one room immediately affects the temperature of the other room. %unless they are perfectly insulated.
% The behavior is specified as \emph{coupled} ODEs
% where variables evolve simultaneously over continuous time.
% %and ODE solving needs to take into account all coupled variables at the same time.
% %Therefore, 
% Existing SMT approaches use the non-modular encoding
% for such distributed hybrid systems. 
% %
% The size of the SMT formula can be huge,
% and  this leads to the \emph{formula explosion problem}
% that makes such SMT-based analysis practically infeasible.


%%% Polished stuff:
To overcome this problem, we present a new SMT framework
to  effectively encode formal analysis problems
of distributed hybrid systems with coupled ODEs \emph{in a modular
  way}.
%%%% You commented out the following sentence. Was there
%%%% something wrong???  Why not include it??
%For a composition $H_1 \parallel \cdots \parallel H_n$ of $n$ hybrid systems,
%where each component $H_i$ has $m_i$ control modes,
%the size of the SMT formula by the new modular encoding is proportional to $\sum_{i=1}^n k_i$,
%whereas by the previous non-modular encoding the size is proportional to $\prod_i k_i$.]
The key idea is to use equation ``names'' %uninterpreted real function symbols %and universal quantification
to logically decompose coupled variables in systems of ODEs in the SMT formula.
%We provide an efficient SMT decision procedure for such formulas by the new encoding
%for distributed hybrid systems.
We show that 
a satisfaction problem in  the new  theory 
can be reduced to one in the standard theory of the real numbers \emph{at no cost}.
%
We have implemented our SMT techniques within the \textsf{dReal} SMT
solver \cite{dReal}.
We have applied our techniques on a range of advanced CPSs, including
systems for: turning an airplane, 
%controlling the altitude of a quad-rotor UAV,
%controlling the water level in a power plant,  
 networked water tank
 controllers,  networked thermostat controllers, and
an automated highway system.
The experimental results show that these techniques can dramatically increase 
the performance of SMT-based analysis for nontrivial distributed CPS.
%that involve multiple control modes and nonlinear ODEs.
% To the best of our knowledge, modular SMT encoding for 
%(coupled) nonlinear ODEs and its efficient decision procedure have never been developed for distributed hybrid systems.

%%% OLD paragraph:
% For this reason, 
% we present a novel SMT framework
% to  effectively encode formal analysis problems
% of distributed hybrid systems with coupled ODEs \emph{in a modular way}.
% %For a composition $H_1 \parallel \cdots \parallel H_n$ of $n$ hybrid systems,
% %when each component $H_i$ has $m_i$ control modes,
% %the size of the SMT formula by the new modular encoding is proportional to $\sum_{i=1}^n k_i$,
% %whereas by the previous non-modular encoding the size is proportional to $\prod_i k_i$.
% The key idea is to use equation ``names'' %uninterpreted real function symbols %and universal quantification
% to logically decompose coupled variables in systems of ODEs in the SMT formula.
% %We provide an efficient SMT decision procedure for such formulas by the new encoding
% %for distributed hybrid systems.
% %We show that 
% The satisfaction problems of the new logical theory 
% can be reduced to ones in the standard theory of the real numbers \emph{at no cost}.
% %
% We have implemented our SMT technique within the \textsf{dReal} SMT solver \cite{dReal}.
% The experimental results show that our techniques can dramatically increase 
% the performance of SMT-based analysis for nontrivial distributed CPS.
% %that involve multiple control modes and nonlinear ODEs.
% % To the best of our knowledge, modular SMT encoding for 
% %(coupled) nonlinear ODEs and its efficient decision procedure have never been developed for distributed hybrid systems.

%\textbf{(Peter) I added the applications above from the commented-out list below..}

The rest of the paper is organized as follows.
%
Section~\ref{sec:pals} gives a background on  PALS.
Section~\ref{sec:hybrid-pals} introduces Hybrid PALS.
%involving nonlinear ODEs and clock skews. 
Section~\ref{sec:smt-encoding} shows how synchronous Hybrid PALS models 
and their verification problems (such as bounded reachability, inductive, and compositional analysis)
can be symbolically encoded as logical formulas.
Section~\ref{sec:smt-logic} presents a new SMT framework supporting
the 
modular encoding of  distributed hybrid systems. % with coupled ODEs.  
Section~\ref{sec:case-studies} gives an overview of the Hybrid PALS verification case studies.
Finally,
Section~\ref{sec:related-work} discusses related work,
and 
Section~\ref{sec:concl} gives some concluding remarks.

% experiments on nontrivial distributed CPS.
%\begin{inparaenum}[(i)]
%	\item a controller for turning an airplane,
%	\item a controller for the attitude of a quad-rotor UAV,
%	\item a steam boiler controller,
%	\item networked water tank controllers,
%	\item networked thermostat controllers, and
%	\item an automated highway system.
%\end{inparaenum}


