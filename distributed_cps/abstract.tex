% !TEX root = /Users/kquine/Dropbox/Research/Papers/2015/CPS-SMT-RTSS/cps-rtss.tex

\begin{abstract}
For effectively designing and analyzing \emph{virtually synchronous} cyber physical systems,
the PALS methodology 
has been proposed to 
reduce the combinatorial complexity
caused by the system's concurrency. 
However, cyber physical systems 
are often \emph{distributed hybrid systems}
in which distributed controllers communicate with each other via a network
and govern physical entities with continuous dynamics.
%
This paper presents Hybrid PALS to reduce 
the design and analysis of virtually synchronous hybrid systems
to those of the underlying synchronous model,
and shows a bisimulation equivalence between a synchronous model and a distributed hybrid model. 
%
We explain how various formal analysis problems of hybrid PALS models, 
such as bounded reachability and inductive analysis,
can be reduced to SMT solving over the real numbers with nonlinear ordinary differential equations up to an arbitrary precision.
%
Since such SMT-based analysis of 
distributed hybrid systems
are typically unfeasible due to the \emph{formula explosion problem},
we also proposes
 a novel SMT framework to effectively encode distributed hybrid systems in a modular way.
%The experimental results show that 
Our techniques greatly increase the performance of SMT analysis
for nontrivial distributed CPS.
\end{abstract}