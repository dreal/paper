% !TEX root = /Users/kquine/Dropbox/Research/Papers/2015/CPS-SMT-RTSS/cps-rtss.tex

\begin{abstract}
For effectively designing and analyzing \emph{virtually synchronous} cyber physical systems,
the PALS methodology 
has been proposed to 
reduce the combinatorial complexity
caused by the system's concurrency. 
However, cyber physical systems 
are often \emph{distributed hybrid systems}
in which distributed controllers communicate with each other  via a network
and govern physical entities with continuous dynamics.
%
Although
formal analysis problems of hybrid systems, involving nonlinear real functions 
and solutions of ordinary differential equations, can be reduced to SMT problems over the real numbers,
%up to an arbitrary precision,
such SMT-based analysis of \emph{distributed} hybrid systems
are typically unfeasible due to the \emph{formula explosion problem}.
%leading to large amounts of unnecessary computation.
%
This paper presents new efficient methods for analyzing distributed hybrid systems using SMT.
We present a novel logical framework to effectively encode distributed hybrid systems as SMT formulas in a modular way, and to allow compositional and inductive analysis together with the PALS methodology.
The experimental results show that our techniques dramatically increase the performance of SMT analysis
for nontrivial distributed hybrid systems.
\end{abstract}