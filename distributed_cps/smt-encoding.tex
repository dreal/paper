% !TEX root = /Users/kquine/Dropbox/Research/Papers/2015/CPS-SMT-RTSS/cps-rtss.tex


\section{Logical Encoding of Hybrid PALS Models}
\label{sec:smt-encoding}

Thanks to the behavioral equivalence (Theorem~\ref{thm:hybrid-pals}),
analysis problems of a distributed CPS $\MA(\mathcal{E}, T, \Gamma) \restriction_{\Pi} E_\mathcal{E}$
are reduced to those on the simpler synchronous model $\mathcal{E} \restriction_{\Pi} E_\mathcal{E}$.
%Directly analyzing such distributed systems is unfeasible in practice
%due to the combinatorial explosion even for simple discrete systems \cite{pals-tcs}.
However, it is difficult to analyze $\mathcal{E} \restriction_{\Pi} E_\mathcal{E}$ 
with (nonlinear)  continuous dynamics and skews of the local clocks.
% (bounded by $\epsilon > 0$).
The concrete values of local clocks are unknown since the values %of local clocks 
are determined on-the-fly by  \emph{clock synchronization} mechanisms \cite{pals-rtss09,lynch-book}.
Numerical simulation can no longer be used for nonlinear dynamics
because numerical errors can be accumulated.

This section explains how analysis problems of 
$\mathcal{E}\restriction_{\Pi}E_\mathcal{E}$ for \emph{any possible local clock}
can be symbolically encoded as logic formulas over the real numbers and ODEs.
The satisfiability of such formulas can then be automatically decided by SMT solving up to a given precision $\delta > 0$
(see Section~\ref{sec:smt-logic}).
%The idea is to express synchronous transitions in $\mathcal{E} \restriction_{\Pi} E_\mathcal{E}$ 
%as formulas of the form $\phi(\vec{i}, \vec{y} \shortmid \vec{y'}, \vec{o})$, where 
%\begin{inparaenum}[(i)]
%	\item $\vec{i}$ denotes the inputs, 
%	\item $\vec{y}$ denotes the states  at the beginning of the round at time $iT - \epsilon$,
%	\item $\vec{y'}$ denotes the states at the end of the round at time $(i+1)T - \epsilon$, and 
%	\item $\vec{o}$ denotes the outputs.
%\end{inparaenum}
%Using $\phi(\vec{i}, \vec{x} \shortmid \vec{x'}, \vec{o})$ as building blocks,
%we show formulas for bounded analysis, inductive analysis, 
%and compositional analysis of parameterized systems.


\subsection{Representing Environment-Restricted Controllers}
\label{sec:discrete-encoding}

A controller $M$ is naturally 
expressed as a logic formula over discrete domain
of the form $\phi_M(\vec{i}, \vec{y} \shortmid \vec{y'}, \vec{o})$,
including \emph{variables}
$\vec{i}$ denoting input, 
$\vec{y}$ denoting the current state, 
$\vec{y'}$ denoting the next state, and 
$\vec{o}$  denoting output.

\begin{definition}
For  $M = (D_i,S,D_o,\delta_M)$,
the formula $\phi_M$ is defined as
$\phi_{M}(\vec{d}_i, s \shortmid s',  \vec{d}_o)
\iff
( (\vec{d}_i, s), (s', \vec{d}_o) ) \in \delta_{M}$.
\end{definition}



A physical environment $E_M$ is expressed as 
a formula over the real numbers %and ODEs
of the form $\phi_{E_M}(\vec{a},u_0,u_t,\vec{v} \shortmid \vec{x})$,
including: \begin{inparaenum}[(i)]
	\item \emph{unary function symbols} $\vec{x}$ denoting the trajectories of $E_M$'s physical parameters $\vec{x}$,
	and
	\item \emph{variables} $\vec{a}$ denoting control commands, % from its controller $M$, 
		$u_0$ and $u_t$ respectively denoting 
		times at the beginning and the end of the trajectory duration, and 
		$\vec{v}$ denoting initial values of the trajectories $\vec{x}$ at time $u_0$.
\end{inparaenum}


If the continuous dynamics of $\vec{x}$ is specified as a system of ODEs
$\frac{\mathrm{d}\vec{x}}{\mathrm{d}t}= F_{\vec{a}}(\vec{x},t)$
for a control command $\vec{a}$, then the formula $\phi_{E_M}$ 
includes universal quantification over time along with solutions of the ODEs, such as a formula of the form:
\[
\bigvee \mathit{guard}(\vec{a}) \rightarrow
\big(
\forall t \in [u_0,u_t].\;
\vec{x}(t) = \vec{v} + \int_0^{t-u_0}  F_{\vec{a}}(\vec{x},t)\,\mathrm{d}t
\big)
\]
For example, 
the formula $\phi_{E_M}(\mathit{rate}_M,u_0,u_t,v_M \shortmid \alpha_M)$
for 
%an environment 
$E_M$ of a subcontroller $M$ in the airplane example is given by:
$\forall t \in [u_0, u_t].\; \alpha_M(t) = v_M + \int_0^{t-u_0}  \mathit{rate}_M \,\mathrm{d}t$.
%where the surface angle $\alpha_M$ follows the ODE
%$\frac{\mathrm{d}\alpha_M}{\mathrm{d}t} = \mathit{rate}_M$
%from the current angle $v_M$ in the time interval $[0,u]$.


\begin{definition}
For %a physical environment 
$E_M = (C, \vec{x}, \Lambda)$,
%the formula $\phi_{E_M}$ is defined as
$\phi_{E_M}(\vec{a},u_0,u_t,\vec{v} \shortmid \vec{x})$
iff
$((\vec{a},u_t-u_0,\vec{v}), \vec{\tau}) \in \Lambda$
and $\vec{x}(t) = \vec{\tau}(t - u_0)$ for $t \in [u_0, u_t]$.
\end{definition}


For an environment restriction $M \restriction E_M$,
its formula $\phi_{M \restriction E_M}$ 
has the form
$\phi_{M \restriction E_M}^{T,i}(\vec{i}, \vec{y}, \vec{v} \shortmid \vec{y'}, \vec{v'}, \vec{o})$,
including variables:
\begin{inparaenum}[(i)]
	\item $\vec{i}$ denoting input, 
	\item $(\vec{y},\vec{v})$ denoting a state  at the beginning of the round 
		(at time $iT - \epsilon$),
	\item $(\vec{y'},\vec{v'})$ denoting a state at the end of the round 
		(at time $(i+1)T - \epsilon$), and 
	\item $\vec{o}$ denoting output.
\end{inparaenum}
%
Instead of explicitly writing $iT - \epsilon$ and $(i+1)T - \epsilon$  in $\phi_{M \restriction E_M}$,
we use the ``$\epsilon$-shifted''  time axis for the interval $[iT, (i+1)T]$. %for the formula $\phi_{M \restriction E_M}$.
%
%A formula $\phi_{M \restriction E_M}$ 
%is basically the direct translation of its formal definition in Definition~\ref{def:env-res}.
But due to clock skews, 
the exact values of sampling time $u_I = (c_M(i)+t_I)-(iT-\epsilon)$
and response time $u_R = (c_M(i)+t_R)-(iT-\epsilon)$ are unpredictable. 
Because $iT - \epsilon < c_M(i) < iT + \epsilon$,
we represent those times as formulas 
$t_I < u_I < t_I + 2\epsilon$ and $t_R < u_R < t_R + 2\epsilon$
within $\phi_{M \restriction E_M}$.






\begin{definition}
For an environment restriction $M \restriction E_M$
with the interface projection functions $\pi=(\pi_T, \pi_R, \pi_I, \pi_C, \pi_O)$,
%the formula 
$\phi_{M \restriction E_M}^{T,i}(\vec{i}, \vec{y}, \vec{v} \shortmid \vec{y'}, \vec{v'}, \vec{o})$
is defined from Definition~\ref{def:env-res} by:
\begin{align*}
\begin{alignedat}{3}
\exists \vec{a},\vec{a'}, \vec{v}_I, \vec{v}_R, u_I, u_R, t_I, t_R.\;
&&&
\vec{a} = \pi_C(\vec{y})
\\
\wedge\;
\phi_{E_M}(\vec{a},iT,iT+u_I,\vec{v} \shortmid \vec{x})
&\;\;\wedge\;\;&&
\vec{v}_I = \vec{x}(iT+u_I)
\\
\wedge\;
\phi_{E_M}(\vec{a},iT+u_I,iT+u_R,\vec{v}_I \shortmid \vec{x})
&\;\;\wedge\;\;&&
\vec{v}_R = \vec{x}(iT+u_R)
\\
\wedge\;
\phi_M(\vec{i},\langle\vec{y},\pi_O(\vec{v}_I)\rangle \shortmid \langle\vec{y'},\pi_O(\vec{v'})\rangle, \vec{o})
&\;\;\wedge\;\;&&
\vec{a'} = \pi_C(\vec{y'})
\\
\wedge\;
\phi_{E_M}(\vec{a'},iT+u_R, (i+1)T ,\vec{v}_R \shortmid \vec{x})
&\;\;\wedge\;\;&&
\vec{v'} = \vec{x}((i+1)T)
\\
\wedge\;
t_I = \pi_I(\vec{y})
&\;\;\wedge\;\;&&
t_R = \pi_R(\vec{y},\vec{i})
\end{alignedat}
\\
\begin{alignedat}{3}
\wedge\;
(t_I < u_I < t_I + 2\epsilon)
\;\wedge\;
(t_R < u_R < t_R + 2\epsilon)
\end{alignedat}
\end{align*}
%
Notice that a local clock $c_M$ (and the round number $\pi_T$)
are abstracted from the formula.
That is,
$\phi_{M \restriction E_M}(\vec{i}, s, \vec{v} \shortmid s', \vec{v'}, \vec{o})$
iff
$( (\vec{i}, (s,\vec{v})), ((s',\vec{v'}), \vec{o}) ) \in \delta_{M \restriction E_M}$
for some $c_M$.
\end{definition}




\subsection{Representing Hybrid Ensembles}

%Recall that 
A synchronous composition  of %an ensemble 
$\mathcal{E}$ is a single machine $M_\mathcal{E}$
in which all machines perform their transitions in lock step.
Each fast machine $f$  performs $k = \mathit{rate}(f)$  internal transitions in one step.
%
A formula 
$\phi_{(M \restriction E_M)^{\times k}}^{T, i}$
for the $k$-step \emph{deceleration} of an environment restriction 
$M \restriction E_M$
is defined by sequentially 
composing
$k$ subformulas $\phi_{M \restriction E_M}^{T/k,ik},\ldots,\phi_{M \restriction E_M}^{T/k,(ik+k-1)}$.
These formulas are respectively related to the $k$ equal \emph{subintervals} $[(ik+n-1)T/k-\epsilon,(ik+n)T/k-\epsilon]$ 
for $n=1,\ldots,k$.

\begin{definition}
Given $M \restriction E_M$ and $k \in \mathbb{N}$,
the formula
$\phi_{(M \restriction E_M)^{\times k}}^{T, i}(\langle\vec{i_1},\ldots,\vec{i_k}\rangle, \vec{y_0}, \vec{v_0} \shortmid \vec{y}_{k}, \vec{v}_{k}, \langle\vec{o_1},\ldots,\vec{o_k}\rangle)$ is:
\[
%(\exists \vec{y_1},\ldots, \vec{y}_{k-1}, \vec{v_1},\ldots, \vec{v}_{k-1})
\exists\{\vec{y_n}, \vec{v_n}\}_{n=1}^{k-1}.\,
\bigwedge_{n=1}^k \phi_{M \restriction E_M}^{T/k,(ik+n-1)}(\vec{i_n}, \vec{y}_{n-1}, \vec{v}_{n-1} \shortmid \vec{y_{n}}, \vec{v_{n}}, \vec{o}_n)
\]
\end{definition}

The wiring diagram of $\mathcal{E}$ is expressed by
a conjunction of %appropriate 
\emph{equalities} between variables denoting input and output ports.
Each equality corresponds to a connection in $\mathcal{E}$
(together with an input adaptor for machines with different rates).
%Since feedback outputs of one machine to itself or another machine 
%becomes input of their destinations in the next step,
%we use a separate set of variables for such output ports connected to other machines in $\mathcal{E}$.
%$\phi_\mathit{wire}$ is a conjunction of equalities, together with the input adaptors of $\{M_j\}_{j\in J_S\cup J_F}$,  given by the wiring diagram of $\mathcal{E}$.


\begin{definition}
%The formula 
$\phi_\mathit{wire}$ is the conjunction of the equalities:
\begin{inparaenum}[(i)]
	\item $\alpha_j^l(i_j^l) = i_e^n$
	for each connection from $\mathcal{E}$'s $n$-th input port to $M_j$'s $l$-th input port 
	with its input adaptor $\alpha_j^l$;
	\item $o_e^n = o_j^l$
	for each connection from $M_j$'s $l$-th output port to $\mathcal{E}$'s $n$-th output port; and
	\item $\alpha_j^l(i_j^l) = f_k^n$ and ${f_k^n}' = o_k^n$
	for each connection from $M_k$'s $n$-th output port to $M_j$'s $l$-th input port with its input adaptor $\alpha_j^l$,
	where $f_k^l$ denotes a feedback output from the previous step, 
	and ${f_k^l}'$ denotes one for the next step.
\end{inparaenum}
\end{definition}


For a hybrid ensemble $\mathcal{E} \restriction_{\Pi} E_\mathcal{E}$ 
of machines  $\{M_j\}_{j\in J_S\cup J_F}$ and their physical environments $\{E_{M_j}\}_{j\in J_S\cup J_F}$
($J_S$ denoting slow machines and $J_F$ denoting fast machines),
its formula has the form
$\phi_{\mathcal{E} \restriction_{\Pi} E_\mathcal{E}}^{T,i}(
	\vec{i}, \{\vec{z_j},\vec{f_j}\}_{j\in J_S\cup J_F}
	\shortmid 
	\{\vec{z'_j}, \vec{f_j'}\}_{j\in J_S\cup J_F}, \vec{o})$,
%
including variables:
\begin{inparaenum}[(i)]
	\item $\vec{i}$ denoting input, 
	\item $\vec{z_j}$ denoting a state of $M_j \restriction E_{M_j}$ at the beginning of the round 
		(at time $iT - \epsilon$),
	\item $\vec{f_j}$ denoting feedback outputs from the previous round,
	\item $\vec{z_j'}$ denoting a state of $M_j \restriction E_{M_j}$ at the end of the round 
		(at time $(i+1)T - \epsilon$), 
	\item $\vec{f_j'}$ denoting the feedback outputs for the next round,	and 
	\item $\vec{o}$ denoting output.
\end{inparaenum}


\begin{definition}
For a hybrid ensemble $\mathcal{E} \restriction_{\Pi} E_\mathcal{E}$, %of machines 
%$\{M_j\}_{j\in J_S\cup J_F}$, their physical environments $\{E_{M_j}\}_{j\in J_S \cup J_F}$
%and the time-invariant constraint $(\forall t.\; \psi)$,
its formula
$\phi_{\mathcal{E} \restriction_{\Pi} E_\mathcal{E}}^{T,i}(
	\vec{i}, \{\vec{y_j}, \vec{v_j},\vec{\mathit{f}_j}\}_{j\in J_S\cup J_F}
	\shortmid 
	\{\vec{y'_j}, \vec{v'_j}, \vec{\mathit{f}_j'}\}_{j\in J_S\cup J_F}, \vec{o})$ is:
\begin{align*}
%(\exists \{\vec{i_j},\vec{o_j}\}_{j\in J_S\cup J_F})\;
\exists \{\vec{i_j},\vec{o_j}\}_{j\in J_S\cup J_F}.\;
\textstyle\bigwedge_{s \in J_S}
\big(\phi_{M_s \restriction E_{M_s}}^{T,i}(\vec{i_s}, \vec{y_s}, \vec{v_s} \shortmid \vec{y'_s}, \vec{v'_s}, \vec{o}_s)\big)
\phantom{.}
\\
\wedge\;
\textstyle\bigwedge_{f \in J_F}
\big(\phi_{(M_f \restriction E_{M_f})^{\times\mathit{rate}(j)}}^{T,i}(\vec{i_f}, \vec{y_f}, \vec{v_f} \shortmid \vec{y'_f}, \vec{v'_f}, \vec{o}_f)\big)
\phantom{.}
\\
\wedge\;
\phi_\mathit{wire}(\vec{i},\vec{o},\{\vec{i_j},\vec{o_j},\vec{\mathit{f}_j},\vec{\mathit{f}_j'}\}_{j\in J_S\cup J_F})
\;\wedge\;
(\forall t.\; \psi).
\end{align*}
\end{definition}

By construction, a formula $\phi_{\mathcal{E} \restriction_{\Pi} E_\mathcal{E}}^{T,i}$ is satisfiable
iff there is a corresponding transition of the synchronous composition $M_{\mathcal{E} \restriction_{\Pi} E_\mathcal{E}}$
from $iT - \epsilon$ to $(i+1)T - \epsilon$ for \emph{some} local clocks.
Therefore, by  the bisimulation equivalence (Theorem~\ref{thm:hybrid-pals}):

\begin{theorem}\label{thm:pals-encoding}
$\phi_{\mathcal{E} \restriction_{\Pi} E_\mathcal{E}}^{T,i}$ is satisfiable 
iff for some local clocks, there exists a corresponding \emph{stable transition} in a distributed hybrid system  
$\MA(\mathcal{E}, T, \Gamma) \restriction_{\Pi} E_\mathcal{E}$   for $[iT - \epsilon,(i+1)T - \epsilon]$.
\end{theorem}

\subsection{Representing Formal Analysis Problems}

Our goal is to verify safety properties
of a distributed CPS $\MA(\mathcal{E}, T, \Gamma) \restriction_{\Pi} E_\mathcal{E}$.
In general, a safety property of the system can be expressed as a formula of the form 
$\forall t.  \mathit{safe}(\vec{z},t)$
with respect to state variables $\vec{z}$ and time variable $t$.


For bounded analysis
to verify %whether $\MA(\mathcal{E}, T, \Gamma) \restriction_{\Pi} E_\mathcal{E}$
a safety property %$\forall t.  \mathit{safe}(\vec{z},t)$
up to a given bound $n \in \mathbb{N}$
(i.e., for the interval $[0,nT - \epsilon]$),
we encode its \emph{bounded counterexamples}. %of $\forall t.  \mathit{safe}(\vec{z},t)$
%up to the bound $n$.
If the formula is unsatisfiable (i.e., no counterexample), 
then the system satisfies the safety property in $[0,nT - \epsilon]$
\emph{for any local clocks} by Theorem~\ref{thm:pals-encoding}.



\begin{definition}
A bounded analysis problem for $\forall t.  \mathit{safe}(\vec{z},t)$
%of $\MA(\mathcal{E}, T, \Gamma) \restriction_{\Pi} E_\mathcal{E}$ 
up to $n$ with an initial condition $\mathit{init}(\vec{z})$
is expressed by:
\begin{align*}
\exists \vec{z}_0, \{\vec{z}_k, \vec{i}_k, \vec{o}_k, t_k\}_{k=1}^{n}.\;
\textstyle\bigwedge_{k=1}^{n}
\big(\phi_{\mathcal{E} \restriction_{\Pi} E_\mathcal{E}}^{T,k}(
	\vec{i}_k, \vec{z}_{k-1}
	\shortmid 
	\vec{z}_{k}, \vec{o}_k)\big)
\phantom{.}
\\
\wedge\;
\mathit{init}(\vec{z}_0)
\;\wedge\;
\textstyle\bigvee_{k=1}^{n}
\big(
(k-1)T < t_{k} \leq T
\wedge
\neg\mathit{safe}(\vec{z}_k, t_{k}) 
\big).
\end{align*}
\end{definition}


For unbounded verification of a safety property,
we encode its inductive proof as logic formulas.
The idea is to find an inductive condition $\mathit{ind}(\vec{z})$ 
for synchronous transitions
that corresponds to the beginning of a global round,
and to show that $\mathit{ind}(\vec{z})$ implies %the safety property 
$\mathit{safe}(\vec{z},t)$ 
for that round.

\begin{definition} 
An inductive analysis problem for $\forall t.  \mathit{safe}(\vec{z},t)$
with an inductive condition $\mathit{ind}(\vec{z})$ is expressed by:
\begin{multline*}
\forall \vec{z}, \vec{z'}, \vec{i}, \vec{o}, t \in [0,T].\;
\big(\mathit{ind}(\vec{z})  \rightarrow \mathit{safe}(\vec{z},t)\big)
\\
\wedge\;
\big(
(\mathit{ind}(\vec{z})\;\wedge\;\phi_{\mathcal{E} \restriction_{\Pi} E_\mathcal{E}}^{T,0}(\vec{i}, \vec{z}\shortmid \vec{z'}, \vec{o}))
\rightarrow
\mathit{ind}(\vec{z'})\big)
.
\end{multline*}
\end{definition}

We also encode a divide-and-conquer proof
to verify a safety property by compositional analysis. % as formulas.
An input condition $\mathit{c}_\mathit{in}(\vec{i},\vec{z},t)$
and an output condition $\mathit{c}_\mathit{out}(\vec{o},\vec{z},t)$ 
during a global round
are identified for synchronous transitions
in such a way 
that $\mathit{c}_\mathit{in}(\vec{i},\vec{z},t)$ implies %the safety property 
$\mathit{safe}(\vec{z},t)$ for that round.
%(time variable $t$ is included because of time-invariant constraints).
%This method is very useful for parameterized systems (see Section~\ref{sec:case-studies}).


\begin{definition} A compositional analysis for $\forall t.  \mathit{safe}(\vec{z},t)$
with I/O conditions $\mathit{c}_\mathit{in}(\vec{i},\vec{z},t)$ and $\mathit{c}_\mathit{out}(\vec{o},\vec{z},t)$
is expressed by:
\begin{multline*}
\forall \vec{z}, \vec{z'}, \vec{i}, \vec{o}, t \in [0,T].\;
\big(\mathit{c}_\mathit{in}(\vec{i},\vec{z},t) \rightarrow \mathit{safe}(\vec{z},t) \big)
\phantom{.}
\\
\wedge\;
\big(
(\mathit{c}_\mathit{in}(\vec{i},\vec{z},t)
\;\wedge\;
\phi_{\mathcal{E} \restriction_{\Pi} E_\mathcal{E}}^{T,0}(
	\vec{i}, \vec{z}
	\shortmid 
	\vec{z'}, \vec{o})
)
\rightarrow
\mathit{c}_\mathit{out}(\vec{o},\vec{z'},t)\big).
\end{multline*}
\end{definition}

The validity of such formulas for inductive or compositional  analysis can also be automatically checked 
using SMT solving
by checking the unsatisfiability of their \emph{negations}.









