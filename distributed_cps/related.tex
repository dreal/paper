% !TEX root = /Users/kquine/Dropbox/Research/Papers/2015/CPS-SMT-RTSS/cps-rtss.tex


\section{Related Work}
\label{sec:related-work}

%many important advances have been made by
%various approaches, including \cite{DBLP:journals/jlp/FranzleTE10,DBLP:conf/cav/FrehseGDCRLRGDM11,DBLP:journals/tac/AlthoffK14,frehse2005phaver,DBLP:conf/icons/HerdeEFT08,DBLP:conf/rtss/ChenAS12,DBLP:conf/aaai/CimattiMT12,platzer2008differential}.

%PALS was originally proposed to reduce the system complexity of 
%single-rate virtually synchronous distributed real-time systems \cite{pals-rtss09,pals-tcs},
%and then extended for multirate systems \cite{mr-pals-journal,al2012pattern} and hybrid systems \cite{ftscs-journal,hybrid-pals}.

%%% Peter: I have no clue what the following section was doing in the
%%% paper ...
% Hybrid automata techniques are not appropriate for hybrid PALS models.
% Hybrid PALS models are basically time triggered models, which are not perfectly synchronized due to clock skews,
% whereas composing hybrid automata requires perfect synchronization of several events.

%%% (Peter): I try to add some PALS rel work. 
PALS~\cite{pals-rtss09,mr-pals-journal,pals-tcs,al2012pattern} targets
distributed real-time systems, whose absence of continuous behaviors
means that continuous behaviors and local clocks do not need to be
taken into account in the synchronous models, which can therefore be
verified by any explicit-state model checker. In contrast, Hybrid PALS
synchronous models must take both clock skews and continuous behaviors
into account and hence cannot be analyzed by explicit-state
techniques. The first steps to add continuous behaviors to PALS were
taken in~\cite{hybrid-pals}. The present work presents a more general
model, where the relative sampling and actuating times are system
parameters,  and also provides a crucial bisimulation result between
the 
synchronous and the distributed model. But the main difference is in the
verification part. \textbf{Kyungmin, add stuff here!!!}


Because of the difficulty of handling SMT
formulas over the reals with nonlinear functions, 
SMT-solving-based verification is a fairly new direction for nonlinear hybrid
systems. The research
direction is initiated in~\cite{ratschan2007safety}, which uses constraint
solving algorithms for handling nonlinear reachability problems. Two
main lines of work that explicitly formulate problems as SMT formulas
are based on the HySAT/iSAT
solver~\cite{DBLP:journals/fmsd/FranzleH07,eggers2008sat}
and the MathSAT
solver~\cite{DBLP:conf/aaai/CimattiMT12,DBLP:conf/fmcad/CimattiMT12}.
But efficient encoding of networks of hybrid systems has not been
investigated in existing work along these lines. 

On the other hand,
for reachable set computation, 
%the work on 
SpaceEX~\cite{DBLP:conf/cav/FrehseGDCRLRGDM11}
has involved methods for handling networks restricted to linear/multiaffine hybrid systems,
and 
Flow*~\cite{DBLP:conf/cav/ChenAS13} proposed an approach for verifying
nonlinear hybrid systems using Taylor model flowpipe construction. 
%
dReach~\cite{dReach}
encodes reachability problems of hybrid systems to SMT problems and
solves them using %a $\delta$-decision procedures, 
dReal~\cite{dReal}.
%
Our approach is different from dReach, 
since dReach \emph{explicitly} enumerates all mode paths of a hybrid automaton
to generate many small SMT formulas for these paths.



%Multirate PALS ?? ( Abdullah's work, other synchronizer, virtually synchronization, ...)
%SMT and \texttt{dReal} ??


