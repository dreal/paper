\documentclass[12pt]{article}
%\documentclass{llncs}
\usepackage{amsmath, amssymb, fancyvrb, multirow, color, relsize}

%%%%%%%%%%%%%%%%%%%%%%%%%%%%%%%%%%%%%
%%%%%%%%%%%% TECH REPORT %%%%%%%%%%%%
%%%%%%%%%%%%%%%%%%%%%%%%%%%%%%%%%%%%%
%Check if we are compiling under latex or pdflatex
\ifx\pdftexversion\undefined
  \usepackage[dvips]{graphics}
\else
  \usepackage[pdftex]{graphics}
\fi

\usepackage{fullpage,cmu-titlepage2}
\usepackage{times}
\usepackage{rotating}
\usepackage{tabularx}
\usepackage[
%backref controls whether the bibliography has a list of backreferences
%to where the citaiton is used
%backref,
pageanchor=true,
plainpages=false,
pdfpagelabels, % makes the status line page labels the same as latex ones
bookmarks,
bookmarksnumbered,
pdfborder=0 0 0,
pdfpagemode=UseOutlines]{hyperref}
\usepackage{draftstamp}
\usepackage{subfigure}
\newcommand{\singlespace}{\renewcommand{\baselinestretch}{1.0}\normalsize}
%\draftstamp{\today}{DRAFT}

%%%%%%%%%%%%%%%%%%%%%%%%%%%%%%%%%%%%%
%%%%%%%%%%%%%%%%%%%%%%%%%%%%%%%%%%%%%


\newcommand{\rup}[1]{\ensuremath{\mathrm{#1}^{\uparrow}}}
\newcommand{\rdn}[1]{\ensuremath{\mathrm{#1}^{\downarrow}}}
\newcommand{\rzr}[1]{\ensuremath{\mathrm{#1}^{\to 0}}}
\newcommand{\rne}[1]{\ensuremath{\mathrm{#1}^{\cdot}}}
\newcommand{\dreal}{\textsf{dReal}}

\title{\mbox{Floating-point Bugs in Embedded GNU C Library}}

\author{Soonho Kong \and Sicun Gao \and Edmund M. Clarke}
%\institute{Carnegie Mellon University, Pittsburgh, PA 15213}
\date{\today}

\abstract{
We report serious bugs in floating-point computations for evaluating elementary
functions in the Embedded GNU C Library. For instance, the sine
function can return values larger than $10^{53}$ in certain rounding
modes. Further investigation also exposed faulty implementations in
the most recent version of the library, which seemingly fixed some
bugs, but only by discarding user-specified rounding-mode
requirements.
}

\keywords{Floating-point computation, GNU C Library, Nonlinear
  Arithmetic}

\trnumber{CMU-CS-13-130}
%\arpasupport{fox}
\support{This research was sponsored by the National Science Foundation grants
no. CNS1330014, no. CNS0926181 and no. CNS0931985, the GSRC under
contract no. 1041377, the Semiconductor Research Corporation under
contract no. 2005TJ1366, and the Office of Naval Research under award
no. N000141010188.}

\begin{document}
\renewcommand*{\thepage}{title-\arabic{page}}
\maketitle
\renewcommand*{\thepage}{\arabic{page}}


\section{Introduction}\label{sec:intro}

We have found floating-point bugs in Linux systems using Embedded
GLIBC (EGLIBC) version 2.16 or older. EGLIBC is a variant of the GNU C
Library (GLIBC) which is used as the default implementation in many
distributions including Debian, Ubuntu, and their variants.

The following C program computes the value of $\sin(-2.437592)$ in
double-precision after setting the rounding direction to upward
($+\infty$).

\begin{Verbatim}[numbers=left, frame=single, fontsize=\relsize{-1}]
#include <math.h>
#include <fenv.h>
#include <stdio.h>

int main() {
    double x = -2.437592;
    fesetround(FE_UPWARD);
    printf("sin(%f)=%f\n", x, sin(x));
    return 0;
}
\end{Verbatim}

The IEEE754 standard~\cite{IEEE:1985:AIS} does not specify correct
rounding methods on elementary functions such as the exponential,
logarithm, and trigonometric functions. Programmers and engineers
usually expect the program to print out an approximated value around
$\sin(-2.437592) \simeq -0.64727239229$ with an ``acceptable'' amount
error. However, they all should agree that the result be in the range
between -1 and 1, even in the worst case.

However, a surprising result appears if we compile and execute the
program in a machine running Ubuntu 12.04 LTS (or any system with
EGLIBC-2.15). The value is greater than $10^{53}$ and it should not be
a return value of sine function in any sense.

\begin{Verbatim}[frame=single, fontsize=\relsize{-1}]
$ gcc exp_bug.c -lm && ./a.out
sin(-2.437592)=191561981424936943059347927032148030287313979209416704.00000
\end{Verbatim}
% $

Here is another C program computing $\cosh(3.113408)$ with directed
rounding toward $+\infty$. This example is more interesting because it
shows different results on Intel and AMD machines, and both of the
results have serious problems.

\begin{Verbatim}[numbers=left, frame=single, fontsize=\relsize{-1}]
#include <math.h>
#include <fenv.h>
#include <stdio.h>

int main() {
    double x = 3.113408;
    fesetround(FE_UPWARD);
    printf("cosh(%f) = %f\n", x, cosh(x));
    return 0;
}
\end{Verbatim}

In a machine with Intel Core i7 CPU, the program outputs
\textit{inf} while a machine with AMD Opteron processor produces
$-160.191709$.

\begin{Verbatim}[frame=single, fontsize=\relsize{-1}]
[INTEL] $ gcc cosh_bug.c -lm && ./a.out
cosh(3.113408) = inf
[AMD]   $ gcc cosh_bug.c -lm && ./a.out
cosh(3.113408) = -160.191709
\end{Verbatim}
% $

Note that $\cosh(3.113408) \simeq 11.2710174432$ and the both results
\textit{inf} and $-160.191709$ are simply wrong. Moreover, each of the
wrong results have significant implications:

\begin{itemize}
\item Intel (\textit{inf}): It has a contagious effect in subsequent
  computations. \textit{inf} is a special value in the IEEE754
  standard which indicates an overflow in a computation. If one of
  subexpressions is evaluated to \textit{inf}, then in general the
  main expression also becomes infinity ($+\infty$ or $-\infty$) or
  NaN (Not a Number).
\item AMD (-160.191709): Mathematically, $\cosh(x)$ is greater than or
  equal to $1$ for all $x \in \mathbb{R}$. As a result, programmers
  and engineers write algorithms based on the invariant $\forall x. \
  1 \le \cosh(x)$. This result, $-160.191709$, breaks the invariant
  and could cause an unexpected behavior.
\end{itemize}


\section{Floating-point Bugs in EGLIBC ($\le 2.16$)}\label{sec:bugs}

We have tested the following math functions in C standard library:

\[
\mathtt{sin}, \mathtt{cos}, \mathtt{tan}, \mathtt{acos},
\mathtt{asin}, \mathtt{atan}, \mathtt{cosh}, \mathtt{sinh},
\mathtt{tanh}, \mathtt{exp}, \mathtt{log}, \mathtt{log10},
\mathtt{sqrt}.
\]

\begin{table}
  \centering
  \caption{Experiment Setup}
  \begin{tabular}{l|c|c||l|c|c}
    Function&               Domain&               Range& Function&    Domain &                    Range \\
    \hline
    \hline
    sin&    $[-10^{306},  10^{306}]$& $[-1.0,     1.0]$   & acos & $[-1.0,    1.0]$     & $[-\infty, +\infty]$\\
    cos&    $[-10^{306},  10^{306}]$& $[-1.0,     1.0]$   & asin & $[-1.0,    1.0]$     & $[-\infty, +\infty]$\\
    tan&    $[-10^{306},  10^{306}]$& $[-\infty, +\infty]$& atan & $[-1.0,    1.0]$     & $[-\infty, +\infty]$\\
    cosh&   $[-500, 500]$         & $[1.0,     +\infty]$& exp  & $[-100, 100]$        & $[0.0,     +\infty]$\\
    sinh&   $[-500, 500]$         & $[-\infty, +\infty]$& log  & $[10^{-306}, 10^{306}]$& $[-\infty, +\infty]$\\
    tanh&   $[-100, 100]$         & $[-1.0,     1.0]$   & log10& $[10^{-306}, 10^{306}]$& $[-\infty, +\infty]$\\
    sqrt&   $[0.0, 10^{306}]$      & $[0.0,     +\infty]$&      &                      &
  \end{tabular}
  \label{tbl:exp_setup}
\end{table}

For each function $f$, we take $100,000$ random numbers from a subset
of function $f$'s domain. Table~\ref{tbl:exp_setup} shows each
function's sampling domain and range. We pick the sampling domain
carefully so that the result of the computation can be fit in a
double-precision variable. We consider the four rounding modes
supported by C99 standard~\cite{ISO:C99}:
\[
\cdot \text{ (nearest)}, \
\to\!\!0 \text{ (toward zero)}, \
\uparrow \text{ (toward $+\infty$)}, \
\downarrow \text{ (toward $-\infty$)}.
\]

\begin{table}[!h]
  \centering
  \caption{Experimental results on Intel and AMD machines:
    $\mathrm{f}^{\uparrow}$, $\mathrm{f}^{\downarrow}$, and
    $\mathrm{f}^{\to 0}$ indicate a function f with rounding mode toward
    $+\infty$, toward $-\infty$, and toward $0$ respectively.
    ``Inconsistent'' denotes the number of cases in which the difference
    of two results are larger than $2^{20}$ ULP (Unit of Least Precision).
    ``Incorrect'' denotes the number of cases in which
    $\mathrm{f}_{\mathrm{C}}(x)$ is out of $f$'s mathematical range.
    ``$\pm\infty$'' denotes the number of cases in which
    $\mathrm{f}_{\mathrm{C}}(x)$ is either $-\infty$ or $+\infty$.
  }
  \begin{tabular}{l||r|r|r|r||r|r|r|r}
\multirow{2}{*}{Function}&  \multicolumn{4}{c}{Intel}& \multicolumn{4}{|c}{AMD}\\
\cline{2-9}
       &   Inconsistent& Incorrect&$\pm\infty$& Total&  Inconsistent& Incorrect&$\pm\infty$& Total\\
    \hline\hline
    \rup{sin}&  10055&    446&   0&  10501&   9853&   450&   0&  10303\\
    \rdn{sin}&   9619&    497&   0&  10116&  10009&   450&   0&  10459\\
    \rzr{sin}&  10087&    436&   0&  10523&   9904&   423&   0&  10327\\
    \rup{cos}&  10097&    434&   0&  10531&   9880&   423&   0&  10303\\
    \rdn{cos}&   9815&    442&   0&  10257&   9910&   461&   0&  10371\\
    \rzr{cos}&   9737&    444&   0&  10181&   9913&   441&   0&  10354\\
    \rup{tan}&  12218&      0&   0&  12218&  12452&     0&   0&  12452\\
    \rdn{tan}&  12387&      0&   0&  12387&  12378&     0&   0&  12378\\
    \rzr{tan}&  12486&      0&   0&  12486&  12506&     0&   0&  12506\\
   \rup{cosh}&  18768&  30139& 935&  49842&  37091& 12295& 291&  49677\\
   \rdn{cosh}&  49766&      0&   0&  49766&  49673&     0&   0&  49673\\
   \rzr{cosh}&  49713&      0&   0&  49713&  49772&     0&   0&  49772\\
   \rup{sinh}&  47807&      0&   0&  47807&  47451&     0& 266&  47717\\
   \rdn{sinh}&  47493&      0&   0&  47493&  47676&     0&   0&  47676\\
   \rzr{sinh}&  47911&      0&   0&  47911&  48046&     0&   0&  48046\\
   \rup{tanh}&   3107&      0&   0&   3107&   3242&     0&   0&   3242\\
   \rdn{tanh}&   3135&      0&   0&   3135&   3268&     0&   0&   3268\\
    \rup{exp}&  47386&   2536&   0&  49922&  37883& 11708& 124&  49715\\
    \rdn{exp}&  49646&      0&   0&  49646&  49978&     0&   0&  49978\\
    \rzr{exp}&  50022&      0&   0&  50022&  49836&     0&   0&  49836
  \end{tabular}
  \label{tbl:exp_result}
\end{table}

For each sample $x$ and for each rounding mode $rnd$, we compute two
values $f^{\mathrm{rnd}}_{C}(x)$ and
$f^{\mathrm{rnd}}_{\mathrm{MPFR}}(x)$ where $f_{C}$ is a function $f$
in C standard library and $f_{\mathrm{MPFR}}$ is a function $f$ in the
GNU MPFR library. MPFR supports arbitrary-precision floating-point
computation and we use it as a reference implementation to have a
comparison. The correctness of MPFR is, of course, another issue and
we do not discuss it here. In the experiments, we use 256-bit
precision for MPFR.

We have the following expectations for the two values:
\begin{itemize}
\item Consistency: The difference between $f^{\mathrm{rnd}}_{C}(x)$
  and $f^{\mathrm{rnd}}_{\mathrm{MPFR}}(x)$ should not be too large.
  In this experiment, we set the threshold of $2^{20}$ ULP (Unit of
  Least Precision) which is the spacing between two adjacent
  floating-point numbers. Note that IEEE754 double-precision format
  has 53 bits of precision and $2^{20}$ ULP implies that it loses
  $20$-bit precision out of $53$. If $\lvert f^{\mathrm{rnd}}_{C}(x) -
  f^{\mathrm{rnd}}_{\mathrm{MPFR}}(x) \rvert > 2^{20} \mathrm{ULP}$,
  we label the case as ``inconsistent''.
\item Correctness: The value of $f^{\mathrm{rnd}}_{C}(x)$ should be in
  the range of the mathematical function $f$. For instance,
  $\sin^{\mathrm{rnd}}_{C}(x)$ has to be between -1.0 and 1.0 no matter
  how imprecise it is.
\end{itemize}

We run the experiments\footnote{Source code is available at
  https://github.com/soonhokong/fp-test} on two machines -- one with
Intel Core i7-2600 CPU (8-core, 3.40GHz) and another with AMD Opteron
Processor 6134 (32-core, 2.30GHz). Both of them are running Ubuntu
12.04 LTS in which uses EGLIBC-2.15 for the C standard library
implementation. We use MPFR-3.1.1 and g++-4.8.1 C++ compiler in the
experiments.

The experimental results are summarized in table~\ref{tbl:exp_result}.
\begin{enumerate}
\item The implementations of $\mathrm{sin}$, $\mathrm{cos}$,
  $\mathrm{tan}$, $\mathrm{cosh}$, $\mathrm{sinh}$, $\mathrm{tanh}$
  and $\mathrm{exp}$ functions have severe problems when used with
  non-default rounding modes (toward $\infty$, toward $-\infty$, and
  toward $0$). It is also not rare to have the problematic case in
  practice. For instance, $\rup{cosh}$ function gives wrong results
  almost 50\% of cases (49,842 out of 100,000).
\item We have not observed any problem under the default rounding mode
  (toward nearest representable number). Also the implementations of
  $\mathrm{acos}, \mathrm{asin}, \mathrm{atan}, \mathrm{tanh},
  \mathrm{log}, \mathrm{log_{10}}$, and $\mathrm{sqrt}$ functions pass
  our tests.
\end{enumerate}

\section{Patch in EGLIBC-2.17}\label{sec:fix}

In EGLIBC-2.17, they provided a patch for the problem. The following
is a part of the new implementation of sine function (IEEE754
double-precision)\footnote{Available at
  \url{http://www.eglibc.org/cgi-bin/viewvc.cgi/branches/eglibc-2_17/libc/sysdeps/ieee754/dbl-64/s_sin.c?view=markup}}:

\begin{Verbatim}[numbers=left, frame=single, firstnumber=101,
  commandchars=\\\{\}, fontsize=\relsize{-1}, label={\texttt{eglibc-2.17/libc/sysdeps/ieee754/dbl-64/s\_sin.c}}, labelposition=topline  ]
__sin(double x)\{
        double xx,res,t,cor,y,s,c,sn,ssn,cs,ccs,xn,a,da,db,eps,xn1,xn2;
#if 0
        double w[2];
#endif
        mynumber u,v;
        int4 k,m,n;
#if 0
        int4 nn;
#endif
        double retval = 0;

        \textbf{SET_RESTORE_ROUND_53BIT (FE_TONEAREST);}
\end{Verbatim}

We find that the patch does not really fix the problem. At line 113,
it resets the rounding mode to ``round to nearest'' and compute the
value of $\sin(x)$ while ignoring the user-specified rounding mode. We
found a case in which the value of $\rup{\sin}_{C}(x)$ is smaller than
the value of $\rne{\sin}_{\mathrm{MPFR}}(x)$, which violates the
semantics of ``toward $+\infty$'' rounding mode:
\begin{align*}
            \rup{\sin}_{C}(-3.93799) & = 0.7148414480838297\underline{66879659928236}\\
   \rne{\sin}_{\mathrm{MPFR}}(-3.93799) & = 0.7148414480838297\underline{71665831705916}
\end{align*}

\section{Conclusion}\label{sec:conclusion}
We report serious bugs in floating-point computations for evaluating
elementary functions in the Embedded GNU C Library. It is not a
negligible numerical error but either a significant error ($2^{20}$
ULP) or mathematically incorrect result (i.e. $\sin(x) > 10^{53}$)
which can trigger severe problems in the following computations.
Moreover, the chances of having these results are not rare at all as
we have shown in section~\ref{sec:bugs}. The current fix does not
mitigate the problem but hides it.

\bibliographystyle{plain}
\bibliography{soon}

\end{document}

%%% Local Variables:
%%% mode: latex
%%% TeX-master: "fp_bugs_tech_report.tex"
%%% End:
